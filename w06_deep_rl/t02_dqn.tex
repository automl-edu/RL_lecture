% !TeX spellcheck = en_US
\documentclass[aspectratio=169]{../latex_main/tntbeamer}  % you can pass all options of the beamer class, e.g., 'handout' or 'aspectratio=43'
\usepackage{dsfont}
\usepackage{bm}
\usepackage[english]{babel}
\usepackage[T1]{fontenc}
%\usepackage[utf8]{inputenc}
\usepackage{graphicx}
\graphicspath{ {./figures/} }
\usepackage{algorithm}
\usepackage[ruled,vlined,algo2e,linesnumbered]{algorithm2e}
\usepackage{hyperref}
\usepackage{booktabs}
\usepackage{mathtools}

\usepackage{amsmath,amssymb}
\usepackage{latexsym}

\DeclareMathOperator*{\argmax}{arg\,max}
\DeclareMathOperator*{\argmin}{arg\,min}

\usepackage{pgfplots}
\pgfplotsset{compat=1.16}
\usepackage{tikz}
\usetikzlibrary{trees} 
\usetikzlibrary{shapes.geometric}
\usetikzlibrary{positioning,shapes,shadows,arrows,calc,mindmap}
\usetikzlibrary{positioning,fadings,through}
\usetikzlibrary{decorations.pathreplacing}
\usetikzlibrary{intersections}
\pgfdeclarelayer{background}
\pgfdeclarelayer{foreground}
\pgfsetlayers{background,main,foreground}
\tikzstyle{activity}=[rectangle, draw=black, rounded corners, text centered, text width=8em]
\tikzstyle{data}=[rectangle, draw=black, text centered, text width=8em]
\tikzstyle{myarrow}=[->, thick, draw=black]

% Define the layers to draw the diagram
\pgfdeclarelayer{background}
\pgfdeclarelayer{foreground}
\pgfsetlayers{background,main,foreground}

\input{./latex_main_old/macros}

% Requires XeLaTeX or LuaLaTeX
\usepackage{unicode-math}

\usepackage{fontspec}
%\setsansfont{Arial}
\setsansfont{RotisSansSerifStd}[ 
Path=./latex_main/fonts/,
Extension = .otf,
UprightFont = *-Regular,  % or *-Light
BoldFont = *-ExtraBold,  % or *-Bold
ItalicFont = *-Italic
]
\setmonofont{Cascadia Mono}[
Scale=0.8
]

% scale factor adapted; mathrm font added (Benjamin Spitschan @TNT, 2021-06-01)
%\setmathfont[Scale=1.05]{Libertinus Math}
%\setmathrm[Scale=1.05]{Libertinus Math}

% other available math fonts are (not exhaustive)
% Latin Modern Math
% XITS Math
% Libertinus Math
% Asana Math
% Fira Math
% TeX Gyre Pagella Math
% TeX Gyre Bonum Math
% TeX Gyre Schola Math
% TeX Gyre Termes Math

% Literature References
% #1 = Display Name
% #2 = Url (without \href)
\newcommand{\lit}[2]{\href{#2}{\footnotesize\color{black!60}[#1]}}

%%% Beamer Customization
%----------------------------------------------------------------------
% (Don't) Show sections in frame header. Options: 'sections', 'sections light', empty
\setbeamertemplate{headline}{empty}

% Add header logo for normal frames
\setheaderimage{
	% \includegraphics[height=\logoheight]{figures/TNT_darkv4.pdf}
	\includegraphics[height=\logoheight]{./latex_main/figures/luh_logo_rgb_0_80_155.pdf}
	% \includegraphics[height=\logoheight]{figures/logo_tntluh.pdf}
}

% Header logo for title page
\settitleheaderimage{
	% \includegraphics[height=\logoheight]{figures/TNT_darkv4.pdf}
	\includegraphics[height=\logoheight]{./latex_main/figures/luh_logo_rgb_0_80_155.pdf}
	% \includegraphics[height=\logoheight]{figures/logo_tntluh.pdf}
}

% Title page: tntdefault 
\setbeamertemplate{title page}[tntdefault]  % or luhstyle
% Add optional title image here
%\addtitlepageimagedefault{\includegraphics[width=0.65\textwidth]{figures/luh_default_presentation_title_image.jpg}}

% Title page: luhstyle
% \setbeamertemplate{title page}[luhstyle]
% % Add optional title image here
% \addtitlepageimage{\includegraphics[width=0.75\textwidth]{figures/luh_default_presentation_title_image.jpg}}

\author[Lindauer]{Marius Lindauer\\[1em]
	\includegraphics[height=\logoheight]{./latex_main/figures/luh_logo_rgb_0_80_155.pdf}\qquad
\includegraphics[height=\logoheight]{./latex_main/figures/TNT_darkv4}\qquad
\includegraphics[height=\logoheight]{./latex_main/figures/L3S.jpg}	}
\date{Winter Term 2021
}


%%% Custom Packages
%----------------------------------------------------------------------
% Create dummy content
\usepackage{blindtext}

% Adds a frame with the current page layout. Just call \layout inside of a frame.
\usepackage{layout}

\title[RL: Deep Reinforcement Learning]{RL: Deep}
\subtitle{DQN}



\begin{document}
	
	\maketitle

%----------------------------------------------------------------------
%----------------------------------------------------------------------
\begin{frame}[c]{RL with Function Approximation}
	
\begin{itemize}
	\item Represent state-action value function by $Q$-network with weights $\vect{w}$
	$$\hat{Q}(s,a;\vect{w}) \approx Q(s,a)$$
\end{itemize}

\centering
\includegraphics[width=0.7\textwidth]{../w05_function_approx/images/vfa.png}


\end{frame}
%-----------------------------------------------------------------------
%----------------------------------------------------------------------
\begin{frame}[c]{Recall: Incremental Model-Free Control Approaches}
	
\begin{itemize}
	\item Similar to policy evaluation, the true state-action value function for a state is unknown and so substitute a target value
	\item In Monte Carlo methods, use a return $G_t$ as a substitute target
	$$\Delta \vect{w} = \alpha(G_t - \hat{Q}(s_t,a_t; \vect{w})) \nabla_{\vect{w}} \hat{Q}(s_t, a_t; \vect{w}) $$
	\item For SARSA instead use a TD target $r+ \gamma \hat{Q}(s', a'; \vect{w})$ which leverages the current function approximations value
	$$\Delta \vect{w} = \alpha (r + \gamma \hat{Q}(s',a';\vect{w}) - \hat{Q}(s,a;\vect{w})) \nabla_{\vect{w}}\hat{Q}(s,a;\vect{w}) $$
	\item For Q-learning instead use a TD target $r + \gamma \max_{a'} \hat{Q}(s',a';\vect{w})$ which leverages the max of the current function approximations value
	$$\Delta \vect{w} = \alpha (r + \gamma \max_{a'} \hat{Q}(s',a';\vect{w}) - \hat{Q}(s,a;\vect{w})) \nabla_{\vect{w}}\hat{Q}(s,a;\vect{w}) $$
\end{itemize}
	
\end{frame}
%-----------------------------------------------------------------------
%----------------------------------------------------------------------
\begin{frame}[c]{Using these Ideas to do Deep RL in Atari}
	
\centering
\includegraphics[width=0.5\textwidth]{images/atari_deep_rl.png}

\begin{flushright}
	\small
	Image by David Silver
\end{flushright}
	
\end{frame}
%-----------------------------------------------------------------------
%----------------------------------------------------------------------
\begin{frame}[c]{Using these Ideas to do Deep RL in Atari}
	
\begin{itemize}
	\item End-to-end learning of values $Q(s, a)$ from pixels $s$
	\item Input state $s$ is stack of raw pixels from last $4$ frames
	\item Output is $Q(s, a)$ for $18$ joystick/button positions
	\item Reward is changed in score for that step
	\item Network architecture and hyperparameters fixed across all games
\end{itemize}

\centering
\includegraphics[width=0.6\textwidth]{images/atari_dqn_arch.png}

\begin{flushright}
	\footnotesize
	\vspace{-0.1cm}
	DQN source code: \url{sites.google.com/a/deepmind.com/dqn/}
\end{flushright}
	
\end{frame}
%-----------------------------------------------------------------------
%----------------------------------------------------------------------
\begin{frame}[c]{Q-Learning with Value Function Approximation}
	
	\begin{itemize}
		\item Minimize MSE loss by stochastic gradient descent
		\item Converges to the optimal $Q^*(s,a)$ using \alert{table lookup} representation
		\item But Q-learning with VFA can diverge
		\item Two of the issues causing problems:
		\begin{itemize}
			\item Correlations between samples violates i.i.d assumption of DNNs
			\item Non-stationary targets
		\end{itemize}
		\item Deep Q-learning (DQN) addresses both of these challenges by
		\begin{itemize}
			\item Experience replay
			\item Fixed Q-targets
		\end{itemize}
	\end{itemize}
	
\end{frame}
%-----------------------------------------------------------------------
%----------------------------------------------------------------------
\begin{frame}[c]{DQNs: Replay Buffer}
	
	\begin{itemize}
		\item To help remove correlations, store dataset (called a \alert{replay buffer}) $\mathcal{D}$ from prior experience
		\item To perform experience replay, repeat the following:
		\begin{enumerate}
			\item $(s,a,r,s')\sim \mathcal{D}$: sample experience tuple from the dataset
			\item Compute the target value for the sampled $s$: $r+\gamma \max_{a'} \hat{Q}(s',a';\vect{w})$
			\item Use stochastic gradient descent to update the network weights
			$$\Delta \vect{w} = \alpha (r + \gamma \max_{a'} \hat{Q}(s',a';\vect{w}) - \hat{Q}(s,a;\vect{w})) \nabla_{\vect{w}}\hat{Q}(s,a;\vect{w})$$
		\end{enumerate}
		\pause
		\item Remarks:
		\begin{itemize}
			\item Fixed sized buffer $\leadsto$ first-in--first-out scheme (as default implementation)
			\item heuristic trade-off between performing new episodes and sampling from the replay buffer
		\end{itemize}
			
		\medskip
		\pause
		\item[$\leadsto$] Can treat the target as a scalar, but the weights will get
		updated on the next round, changing the target value
	\end{itemize}
	
\end{frame}
%-----------------------------------------------------------------------
%----------------------------------------------------------------------
\begin{frame}[c]{DQNs: Fixed Q-Targets}
	\vspace{-1em}
	\begin{itemize}
		\item To help improve stability, fix the \alert{target weights} used in the target calculation for multiple updates
		\item Target network uses a different set of weights than the weights being updated
		\item Let parameters $\vect{w}^-$ be the set of weights used in the target and $\vect{w}$ be the weights that are being updated
		\item Slight change to the computation of target value:
		\begin{itemize}
			\item $(s,a,r,s')\sim \mathcal{D}$: sample experience tuple from the dataset
			\item Compute the target value for the sampled $s$: $r+\gamma \max_{a'} \hat{Q}(s',a';\vect{w}^-)$
			\item Use stochastic gradient descent to update the network weights
			$$\Delta \vect{w} = \alpha (r + \gamma \max_{a'} \hat{Q}(s',a';\vect{w}^-) - \hat{Q}(s,a;\vect{w})) \nabla_{\vect{w}}\hat{Q}(s,a;\vect{w})$$
		\end{itemize}
		\smallskip 
		\pause
		\item Remark:
		\begin{itemize}
			\item Hyperparameter how often you update $\vect{w}^-$
			\item Trade-off between updating too often ($\leadsto$ instability) and\\ too rarely ($\leadsto$ too old state information)
		\end{itemize}
	\end{itemize}
	
\end{frame}
%-----------------------------------------------------------------------
%----------------------------------------------------------------------
\begin{frame}[c]{DQN Summary}
	
	\begin{itemize}
		\item DQN uses experience replay and fixed Q-targets
		\item Store transition $(s_t, a_t, r_{t+1}, s_{t+1})$ in replay memory $\mathcal{D}$
		\item Sample random mini-batch of transitions $(s,a,r,s')$ from $\mathcal{D}$
		\item Compute Q-learning targets wrt old, fixed parameters $\vect{w}^-$
		\begin{itemize}
			\item Update $\vect{w}^-$ from time to time
		\end{itemize}
		\item Optimizes MSE between Q-network and Q-learning targets
		\item Uses stochastic gradient descent
	\end{itemize}
	
\end{frame}
%-----------------------------------------------------------------------
%-----------------------------------------------------------------------
\end{document}
