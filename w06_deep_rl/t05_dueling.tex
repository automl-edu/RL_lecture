% !TeX spellcheck = en_US
\documentclass[aspectratio=169]{../latex_main/tntbeamer}  % you can pass all options of the beamer class, e.g., 'handout' or 'aspectratio=43'
\usepackage{dsfont}
\usepackage{bm}
\usepackage[english]{babel}
\usepackage[T1]{fontenc}
%\usepackage[utf8]{inputenc}
\usepackage{graphicx}
\graphicspath{ {./figures/} }
\usepackage{algorithm}
\usepackage[ruled,vlined,algo2e,linesnumbered]{algorithm2e}
\usepackage{hyperref}
\usepackage{booktabs}
\usepackage{mathtools}

\usepackage{amsmath,amssymb}
\usepackage{latexsym}

\DeclareMathOperator*{\argmax}{arg\,max}
\DeclareMathOperator*{\argmin}{arg\,min}

\usepackage{pgfplots}
\pgfplotsset{compat=1.16}
\usepackage{tikz}
\usetikzlibrary{trees} 
\usetikzlibrary{shapes.geometric}
\usetikzlibrary{positioning,shapes,shadows,arrows,calc,mindmap}
\usetikzlibrary{positioning,fadings,through}
\usetikzlibrary{decorations.pathreplacing}
\usetikzlibrary{intersections}
\pgfdeclarelayer{background}
\pgfdeclarelayer{foreground}
\pgfsetlayers{background,main,foreground}
\tikzstyle{activity}=[rectangle, draw=black, rounded corners, text centered, text width=8em]
\tikzstyle{data}=[rectangle, draw=black, text centered, text width=8em]
\tikzstyle{myarrow}=[->, thick, draw=black]

% Define the layers to draw the diagram
\pgfdeclarelayer{background}
\pgfdeclarelayer{foreground}
\pgfsetlayers{background,main,foreground}

\input{./latex_main_old/macros}

% Requires XeLaTeX or LuaLaTeX
\usepackage{unicode-math}

\usepackage{fontspec}
%\setsansfont{Arial}
\setsansfont{RotisSansSerifStd}[ 
Path=./latex_main/fonts/,
Extension = .otf,
UprightFont = *-Regular,  % or *-Light
BoldFont = *-ExtraBold,  % or *-Bold
ItalicFont = *-Italic
]
\setmonofont{Cascadia Mono}[
Scale=0.8
]

% scale factor adapted; mathrm font added (Benjamin Spitschan @TNT, 2021-06-01)
%\setmathfont[Scale=1.05]{Libertinus Math}
%\setmathrm[Scale=1.05]{Libertinus Math}

% other available math fonts are (not exhaustive)
% Latin Modern Math
% XITS Math
% Libertinus Math
% Asana Math
% Fira Math
% TeX Gyre Pagella Math
% TeX Gyre Bonum Math
% TeX Gyre Schola Math
% TeX Gyre Termes Math

% Literature References
% #1 = Display Name
% #2 = Url (without \href)
\newcommand{\lit}[2]{\href{#2}{\footnotesize\color{black!60}[#1]}}

%%% Beamer Customization
%----------------------------------------------------------------------
% (Don't) Show sections in frame header. Options: 'sections', 'sections light', empty
\setbeamertemplate{headline}{empty}

% Add header logo for normal frames
\setheaderimage{
	% \includegraphics[height=\logoheight]{figures/TNT_darkv4.pdf}
	\includegraphics[height=\logoheight]{./latex_main/figures/luh_logo_rgb_0_80_155.pdf}
	% \includegraphics[height=\logoheight]{figures/logo_tntluh.pdf}
}

% Header logo for title page
\settitleheaderimage{
	% \includegraphics[height=\logoheight]{figures/TNT_darkv4.pdf}
	\includegraphics[height=\logoheight]{./latex_main/figures/luh_logo_rgb_0_80_155.pdf}
	% \includegraphics[height=\logoheight]{figures/logo_tntluh.pdf}
}

% Title page: tntdefault 
\setbeamertemplate{title page}[tntdefault]  % or luhstyle
% Add optional title image here
%\addtitlepageimagedefault{\includegraphics[width=0.65\textwidth]{figures/luh_default_presentation_title_image.jpg}}

% Title page: luhstyle
% \setbeamertemplate{title page}[luhstyle]
% % Add optional title image here
% \addtitlepageimage{\includegraphics[width=0.75\textwidth]{figures/luh_default_presentation_title_image.jpg}}

\author[Lindauer]{Marius Lindauer\\[1em]
	\includegraphics[height=\logoheight]{./latex_main/figures/luh_logo_rgb_0_80_155.pdf}\qquad
\includegraphics[height=\logoheight]{./latex_main/figures/TNT_darkv4}\qquad
\includegraphics[height=\logoheight]{./latex_main/figures/L3S.jpg}	}
\date{Winter Term 2021
}


%%% Custom Packages
%----------------------------------------------------------------------
% Create dummy content
\usepackage{blindtext}

% Adds a frame with the current page layout. Just call \layout inside of a frame.
\usepackage{layout}

\title[RL: Deep Reinforcement Learning]{RL: Deep}
\subtitle{Dueling Networks}



\begin{document}
	
	\maketitle

%----------------------------------------------------------------------
%----------------------------------------------------------------------
\begin{frame}[c]{Value \& Advantage Function}
	

\begin{itemize}
	\item Intuition: Different features might be needed either (a) to accurately represent value or\\ (b) to specify difference in actions
	\item For example
	\begin{itemize}
		\item Game score may help accurately predict $V(s)$
		\item But not necessarily in indicating relative action values $Q(s,a_1)$ vs $Q(s,a_2)$
	\end{itemize}
	\item Advantage function \lit{Baird 1993}\\
	$$A^\pi (s,a) = Q^\pi(s,a) - V^\pi(s) $$
\end{itemize}
	
\end{frame}
%-----------------------------------------------------------------------
%----------------------------------------------------------------------
\begin{frame}[c]{Dueling DQN~\lit{Wang et al. 2016}{https://arxiv.org/abs/1511.06581}}

\begin{center}
Standard DQN

\vspace{-2em}
\includegraphics[width=0.38\textwidth]{images/dueling_networks.png}

Plain DQN (above) vs Dueling DQN (below)
\end{center}

\begin{itemize}
	\item Above head predicts $V(s)$
	\item Heads below predicts $A(s,a_1)$, $A(s,a_2)$, $\ldots$
	\item Combination: $Q(s,a_1)$, $Q(s,a_2)$, $\ldots$
\end{itemize}
	
\end{frame}
%-----------------------------------------------------------------------
%----------------------------------------------------------------------
\begin{frame}[c]{Dueling DQN~\lit{Wang et al. 2016}{https://arxiv.org/abs/1511.06581}}
	
	\begin{itemize}
	\item Advantage function \lit{Baird 1993}\\
	$$A^\pi (s,a) = Q^\pi(s,a) - V^\pi(s) $$
	\item Consider a network that outputs $V(s; \vect{w}_1, \vect{w}_2)$ as well as advantage $A(s,a; \vect{w}_1, \vect{w}_3)$ where $\vect{w}_i$ are the weights of the different parts of the network
	\item To construct $Q$ could use\\ $$Q(s,a;\vect{w}_1, \vect{w}_2, \vect{w}_3) = V(s;\vect{w}_1, \vect{w}_2) + A(s,a;\vect{w}_1, \vect{w}_3)$$
	%\item Do we expect that this architecture will result in us learning a good estimate of true $V$ or $A$?
	\bigskip
	\pause
	\item Challenge: There doesn't have to be a unique advantage function

	\end{itemize}
	
\end{frame}
%-----------------------------------------------------------------------
%----------------------------------------------------------------------
\begin{frame}[c]{Uniqueness}
	
	\begin{itemize}
		\item Consider a network that outputs $V(s;\vect{w}_1, \vect{w}_2)$ as well as advantage $A(s,a; \vect{w}_1, \vect{w}_3)$
		\item To construct $Q$, could use $$Q(s,a;\vect{w}_1, \vect{w}_2, \vect{w}_3) = V(s;\vect{w}_1, \vect{w}_2) + A(s,a;\vect{w}_1, \vect{w}_3)$$
		\item Option 1: Force $Q(s,a) = V(s)$ for the best action suggested by the advantage:\\
		$$\hat{Q}(s,a;\vect{w}) = \hat{V}(s;\vect{w}) + \left( \hat{A}(s,a;\vect{w}) - \max_{a' \in \mathcal{A}} \hat{A}(s,a';\vect{w}) \right) $$
		\vspace{-1em}
		\begin{itemize}
			\item This helps to force the $V$ network to approximate $V$
		\end{itemize}
		\item Option 2: Use mean as baseline ($\leadsto$ more stable)\\
		$$\hat{Q}(s,a;\vect{w}) = \hat{V}(s;\vect{w}) + \left( \hat{A}(s,a;\vect{w}) - \frac{1}{|\mathcal{A}|} \sum_{a' \in \mathcal{A}} \hat{A}(s,a';\vect{w}) \right) $$
		\vspace{-1em}
		\begin{itemize}
			\item More stable often because averaging over all advantages instead of
			the advantage of the current max action.
		\end{itemize}
	\end{itemize}
	
\end{frame}
%-----------------------------------------------------------------------
%-----------------------------------------------------------------------
\end{document}
