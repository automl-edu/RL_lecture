% !TeX spellcheck = en_US

\documentclass[aspectratio=169]{tntbeamer}  % you can pass all options of the beamer class, e.g., 'handout' or 'aspectratio=43'
\usepackage{dsfont}
\usepackage{bm}
\usepackage[english]{babel}
\usepackage[T1]{fontenc}
%\usepackage[utf8]{inputenc}
\usepackage{graphicx}
\graphicspath{ {./figures/} }
\usepackage{algorithm}
\usepackage[ruled,vlined,algo2e,linesnumbered]{algorithm2e}
\usepackage{hyperref}
\usepackage{booktabs}
\usepackage{mathtools}

\usepackage{amsmath,amssymb}
\usepackage{latexsym}

\DeclareMathOperator*{\argmax}{arg\,max}
\DeclareMathOperator*{\argmin}{arg\,min}

\usepackage{pgfplots}
\pgfplotsset{compat=1.16}
\usepackage{tikz}
\usetikzlibrary{trees} 
\usetikzlibrary{shapes.geometric}
\usetikzlibrary{positioning,shapes,shadows,arrows,calc,mindmap}
\usetikzlibrary{positioning,fadings,through}
\usetikzlibrary{decorations.pathreplacing}
\usetikzlibrary{intersections}
\pgfdeclarelayer{background}
\pgfdeclarelayer{foreground}
\pgfsetlayers{background,main,foreground}
\tikzstyle{activity}=[rectangle, draw=black, rounded corners, text centered, text width=8em]
\tikzstyle{data}=[rectangle, draw=black, text centered, text width=8em]
\tikzstyle{myarrow}=[->, thick, draw=black]

% Define the layers to draw the diagram
\pgfdeclarelayer{background}
\pgfdeclarelayer{foreground}
\pgfsetlayers{background,main,foreground}

\input{./latex_main_old/macros}

% Requires XeLaTeX or LuaLaTeX
\usepackage{unicode-math}

\usepackage{fontspec}
%\setsansfont{Arial}
\setsansfont{RotisSansSerifStd}[ 
Path=./latex_main/fonts/,
Extension = .otf,
UprightFont = *-Regular,  % or *-Light
BoldFont = *-ExtraBold,  % or *-Bold
ItalicFont = *-Italic
]
\setmonofont{Cascadia Mono}[
Scale=0.8
]

% scale factor adapted; mathrm font added (Benjamin Spitschan @TNT, 2021-06-01)
%\setmathfont[Scale=1.05]{Libertinus Math}
%\setmathrm[Scale=1.05]{Libertinus Math}

% other available math fonts are (not exhaustive)
% Latin Modern Math
% XITS Math
% Libertinus Math
% Asana Math
% Fira Math
% TeX Gyre Pagella Math
% TeX Gyre Bonum Math
% TeX Gyre Schola Math
% TeX Gyre Termes Math

% Literature References
% #1 = Display Name
% #2 = Url (without \href)
\newcommand{\lit}[2]{\href{#2}{\footnotesize\color{black!60}[#1]}}

%%% Beamer Customization
%----------------------------------------------------------------------
% (Don't) Show sections in frame header. Options: 'sections', 'sections light', empty
\setbeamertemplate{headline}{empty}

% Add header logo for normal frames
\setheaderimage{
	% \includegraphics[height=\logoheight]{figures/TNT_darkv4.pdf}
	\includegraphics[height=\logoheight]{./latex_main/figures/luh_logo_rgb_0_80_155.pdf}
	% \includegraphics[height=\logoheight]{figures/logo_tntluh.pdf}
}

% Header logo for title page
\settitleheaderimage{
	% \includegraphics[height=\logoheight]{figures/TNT_darkv4.pdf}
	\includegraphics[height=\logoheight]{./latex_main/figures/luh_logo_rgb_0_80_155.pdf}
	% \includegraphics[height=\logoheight]{figures/logo_tntluh.pdf}
}

% Title page: tntdefault 
\setbeamertemplate{title page}[tntdefault]  % or luhstyle
% Add optional title image here
%\addtitlepageimagedefault{\includegraphics[width=0.65\textwidth]{figures/luh_default_presentation_title_image.jpg}}

% Title page: luhstyle
% \setbeamertemplate{title page}[luhstyle]
% % Add optional title image here
% \addtitlepageimage{\includegraphics[width=0.75\textwidth]{figures/luh_default_presentation_title_image.jpg}}

\author[Lindauer]{Marius Lindauer\\[1em]
	\includegraphics[height=\logoheight]{./latex_main/figures/luh_logo_rgb_0_80_155.pdf}\qquad
\includegraphics[height=\logoheight]{./latex_main/figures/TNT_darkv4}\qquad
\includegraphics[height=\logoheight]{./latex_main/figures/L3S.jpg}	}
\date{Winter Term 2021
}


%%% Custom Packages
%----------------------------------------------------------------------
% Create dummy content
\usepackage{blindtext}

% Adds a frame with the current page layout. Just call \layout inside of a frame.
\usepackage{layout}


\institute{Institut f\"ur Informationsverarbeitung}%\\ Leibniz Universit\"at Hannover}
%\title{TNT Beamer Template}
%\author{Suomynon A. Anonymous}
\date{}



\title[Reinforcement Learning: Deep Reinforcement Learning]{RL: Deep}
\subtitle{Dueling Networks}



\begin{document}
	
	\maketitle

%----------------------------------------------------------------------
%----------------------------------------------------------------------
\begin{frame}[c]{Value \& Advantage Function}
	

\begin{itemize}
	\item Intuition: Features need to accurate represent value may be different
	than those needed to specify difference in actions
	\item E.g.
	\begin{itemize}
		\item Game score may help accurately predict $V(s)$
		\item But not necessarily in indicating relative action values $Q(s,a_1)$ vs $Q(s,a_2)$
	\end{itemize}
	\item Advantage function \lit{Baird 1993}
	$$A^\pi (s,a) = Q^\pi(s,a) - V^\pi(s) $$
\end{itemize}
	
\end{frame}
%-----------------------------------------------------------------------
%----------------------------------------------------------------------
\begin{frame}[c]{Dueling DQN \litw{\href{https://arxiv.org/abs/1511.06581}{Wang et al. 2016}}}

\begin{center}
Standard DQN

\includegraphics[width=0.5\textwidth]{images/dueling_networks.png}

Dueling DQN
\end{center}

\begin{itemize}
	\item Above head predicts $V(s)$
	\item Heads below predicts $A(s,a_1)$, $A(s,a_2)$, $\ldots$
	\item Combination: $Q(s,a_1)$, $Q(s,a_2)$, $\ldots$
\end{itemize}
	
\end{frame}
%-----------------------------------------------------------------------
%----------------------------------------------------------------------
\begin{frame}[c]{Dueling DQN \litw{\href{https://arxiv.org/abs/1511.06581}{Wang et al. 2016}}}
	
	\begin{itemize}
	\item Advantage function \lit{Baird 1993}
	$$A^\pi (s,a) = Q^\pi(s,a) - V^\pi(s) $$
	\item Consider a network that outputs $V(s; \vec{w}_1, \vec{w}_2)$ as well as advantage $A(s,a; \vec{w}_1, \vec{w}_3)$ where $\vec{w}_i$ are the weights of the different parts of the network
	\item To construct $Q$ could use $$Q(s,a;\vec{w}_1, \vec{w}_2, \vec{w}_3) = V(s;\vec{w}_1, \vec{w}_2) + A(s,a;\vec{w}_1, \vec{w}_3)$$
	%\item Do we expect that this architecture will result in us learning a good estimate of true $V$ or $A$?
	\bigskip
	\pause
	\item Challenge: There doesn't have to be a unique advantage function

	\end{itemize}
	
\end{frame}
%-----------------------------------------------------------------------
%----------------------------------------------------------------------
\begin{frame}[c]{Uniqueness}
	
	\begin{itemize}
		\item Consider a network that outputs $V(s;\vec{w}_1, \vec{w}2)$ as well as advantage $A(s,a; \vec{w}_1, \vec{w}_3)$
		\item To construct $Q$ could use $$Q(s,a;\vec{w}_1, \vec{w}_2, \vec{w}_3) = V(s;\vec{w}_1, \vec{w}_2) + A(s,a;\vec{w}_1, \vec{w}_3)$$
		\item Option 1: Force $Q(s,a) = V(s)$ for the best action suggested by the advantage:
		$$\hat{Q}(s,a;\vec{w}) = \hat{V}(s;\vec{w}) + \left( \hat{A}(s,a;\vec{w} - \max_{a' \in \mathcal{A}} \hat{A}(s,a';\vec{w})) \right) $$
		\vspace{-1em}
		\begin{itemize}
			\item This helps to force the $V$ network to approximate $V$
		\end{itemize}
		\item Option 2: Use mean as baseline (more stable)
		$$\hat{Q}(s,a;\vec{w}) = \hat{V}(s;\vec{w}) + \left( \hat{A}(s,a;\vec{w} - \frac{1}{|\mathcal{A}|} \sum{a' \in \mathcal{A}} \hat{A}(s,a';\vec{w})) \right) $$
		\vspace{-1em}
		\begin{itemize}
			\item More stable often because averaging over all advantages instead of
			the advantage of the current max action.
		\end{itemize}
	\end{itemize}
	
\end{frame}
%-----------------------------------------------------------------------
%----------------------------------------------------------------------
\begin{frame}[c]{Practical Tips for DQN on Atari (from J. Schulman)}
	
	\begin{itemize}
		\item DQN is more reliable on some Atari tasks than others. Pong is a
		reliable task: if it doesn’t achieve good scores, something is wrong
		\item Large replay buffers improve robustness of DQN, and memory
		efficiency is key
		\begin{itemize}
			\item Use uint8 images, don’t duplicate data
		\end{itemize}
		\item Be patient. DQN converges slowly—for ATARI it’s often necessary to
		wait for 10-40M frames (couple of hours to a day of training on GPU)
		to see results significantly better than random policy
	\end{itemize}
	
\end{frame}
%-----------------------------------------------------------------------
%----------------------------------------------------------------------
\begin{frame}[c]{Practical Tips for DQN on Atari (from J. Schulman) cont.}
	
	\begin{itemize}
		\item Try Huberloss on Bellman error
		$$L(x) =  \begin{cases}
		\frac{x^2}{2} & \text{if } |x| \leq \delta\\
		\delta |x| - \frac{\delta^2}{2} & \text{otherwise}
		\end{cases}
		$$
		\item Consider trying Double DQN—significant improvement from small
		code change 
		\item To test out your data pre-processing, try your own skills at navigating
		the environment based on processed frames
		\item Always run at least two different seeds when experimenting
		\begin{itemize}
			\item [ML] I would rather recommend $4$ --- for final evaluation, even more!
		\end{itemize}
		\item Learning rate scheduling is beneficial. Try high learning rates in initial
		exploration period
		\item Try non-standard exploration schedules [later more]
	\end{itemize}
	
\end{frame}
%-----------------------------------------------------------------------
%-----------------------------------------------------------------------
\end{document}
