% !TeX spellcheck = en_US
\documentclass[aspectratio=169]{../latex_main/tntbeamer}  % you can pass all options of the beamer class, e.g., 'handout' or 'aspectratio=43'
\usepackage{dsfont}
\usepackage{bm}
\usepackage[english]{babel}
\usepackage[T1]{fontenc}
%\usepackage[utf8]{inputenc}
\usepackage{graphicx}
\graphicspath{ {./figures/} }
\usepackage{algorithm}
\usepackage[ruled,vlined,algo2e,linesnumbered]{algorithm2e}
\usepackage{hyperref}
\usepackage{booktabs}
\usepackage{mathtools}

\usepackage{amsmath,amssymb}
\usepackage{latexsym}

\DeclareMathOperator*{\argmax}{arg\,max}
\DeclareMathOperator*{\argmin}{arg\,min}

\usepackage{pgfplots}
\pgfplotsset{compat=1.16}
\usepackage{tikz}
\usetikzlibrary{trees} 
\usetikzlibrary{shapes.geometric}
\usetikzlibrary{positioning,shapes,shadows,arrows,calc,mindmap}
\usetikzlibrary{positioning,fadings,through}
\usetikzlibrary{decorations.pathreplacing}
\usetikzlibrary{intersections}
\pgfdeclarelayer{background}
\pgfdeclarelayer{foreground}
\pgfsetlayers{background,main,foreground}
\tikzstyle{activity}=[rectangle, draw=black, rounded corners, text centered, text width=8em]
\tikzstyle{data}=[rectangle, draw=black, text centered, text width=8em]
\tikzstyle{myarrow}=[->, thick, draw=black]

% Define the layers to draw the diagram
\pgfdeclarelayer{background}
\pgfdeclarelayer{foreground}
\pgfsetlayers{background,main,foreground}

\input{./latex_main_old/macros}

% Requires XeLaTeX or LuaLaTeX
\usepackage{unicode-math}

\usepackage{fontspec}
%\setsansfont{Arial}
\setsansfont{RotisSansSerifStd}[ 
Path=./latex_main/fonts/,
Extension = .otf,
UprightFont = *-Regular,  % or *-Light
BoldFont = *-ExtraBold,  % or *-Bold
ItalicFont = *-Italic
]
\setmonofont{Cascadia Mono}[
Scale=0.8
]

% scale factor adapted; mathrm font added (Benjamin Spitschan @TNT, 2021-06-01)
%\setmathfont[Scale=1.05]{Libertinus Math}
%\setmathrm[Scale=1.05]{Libertinus Math}

% other available math fonts are (not exhaustive)
% Latin Modern Math
% XITS Math
% Libertinus Math
% Asana Math
% Fira Math
% TeX Gyre Pagella Math
% TeX Gyre Bonum Math
% TeX Gyre Schola Math
% TeX Gyre Termes Math

% Literature References
% #1 = Display Name
% #2 = Url (without \href)
\newcommand{\lit}[2]{\href{#2}{\footnotesize\color{black!60}[#1]}}

%%% Beamer Customization
%----------------------------------------------------------------------
% (Don't) Show sections in frame header. Options: 'sections', 'sections light', empty
\setbeamertemplate{headline}{empty}

% Add header logo for normal frames
\setheaderimage{
	% \includegraphics[height=\logoheight]{figures/TNT_darkv4.pdf}
	\includegraphics[height=\logoheight]{./latex_main/figures/luh_logo_rgb_0_80_155.pdf}
	% \includegraphics[height=\logoheight]{figures/logo_tntluh.pdf}
}

% Header logo for title page
\settitleheaderimage{
	% \includegraphics[height=\logoheight]{figures/TNT_darkv4.pdf}
	\includegraphics[height=\logoheight]{./latex_main/figures/luh_logo_rgb_0_80_155.pdf}
	% \includegraphics[height=\logoheight]{figures/logo_tntluh.pdf}
}

% Title page: tntdefault 
\setbeamertemplate{title page}[tntdefault]  % or luhstyle
% Add optional title image here
%\addtitlepageimagedefault{\includegraphics[width=0.65\textwidth]{figures/luh_default_presentation_title_image.jpg}}

% Title page: luhstyle
% \setbeamertemplate{title page}[luhstyle]
% % Add optional title image here
% \addtitlepageimage{\includegraphics[width=0.75\textwidth]{figures/luh_default_presentation_title_image.jpg}}

\author[Lindauer]{Marius Lindauer\\[1em]
	\includegraphics[height=\logoheight]{./latex_main/figures/luh_logo_rgb_0_80_155.pdf}\qquad
\includegraphics[height=\logoheight]{./latex_main/figures/TNT_darkv4}\qquad
\includegraphics[height=\logoheight]{./latex_main/figures/L3S.jpg}	}
\date{Winter Term 2021
}


%%% Custom Packages
%----------------------------------------------------------------------
% Create dummy content
\usepackage{blindtext}

% Adds a frame with the current page layout. Just call \layout inside of a frame.
\usepackage{layout}

\title[RL: Deep Reinforcement Learning]{RL: Deep}
\subtitle{Double DQN}



\begin{document}
	
	\maketitle

%----------------------------------------------------------------------
%----------------------------------------------------------------------
\begin{frame}[c]{Recall: Double Q-Learning}
	
	\begin{itemize}
		\item Initialization:
		\begin{itemize}
			\item $Q_1(s,a)$ and $Q_2(s,a)$ for $\forall s \in S, a\in A$
			\item $t= 0$
			\item initial state $s_t = s_0$
		\end{itemize}
		\item Loop
		\begin{itemize}
			\item Select $a_t$ using $\epsilon$-greedy $\pi(s) \in \argmax_{a \in A} Q_1(s_t, a) + Q_2(s_t , a)$
			\item Observe $(r_t, s_{t+1})$
			\item With 50-50 probability either
			\begin{enumerate}
				\item $Q_1(s_t, a_t) \gets Q_1(s_t, a_t) + \alpha (r_t +\gamma \max_{a\in A} Q_2(s_{t+1}, a) - Q_1(s_t, a_t))$\\
				or
				\item $Q_2(s_t, a_t) \gets Q_2(s_t, a_t) + \alpha (r_t +\gamma \max_{a\in A} Q_1(s_{t+1}, a) - Q_2(s_t, a_t))$
			\end{enumerate}
			\item $t = t + 1 $
		\end{itemize}
	
		\bigskip
		\pause
		\item[$\leadsto$] reduces maximization bias
	\end{itemize}
	
\end{frame}
%-----------------------------------------------------------------------
%----------------------------------------------------------------------
\begin{frame}[c]{Double DQN \litw{\href{https://arxiv.org/pdf/1509.06461.pdf}{Hasselt et al. 2015}}}
	
	\begin{itemize}
		\item Extend this idea to DQN
		\item Current Q-network $\vec{w}$ is used to select actions 
		\item Older Q-network $\vec{w}^-$ is used to evaluate actions
		\item TD-error:
		$$r + \gamma \overbrace{\hat{Q}(s', \underbrace{\argmax_{a' \in A} \hat{Q}(s',a';\vec{w})}_{\text{Action selection: }\vec{w}};\vec{w}^-)}^{\text{Action evaluation: }\vec{w}^-} - Q(s,a;\vec{w})$$
		
		\pause
		\item Allows flipping between both weight sets frequently 
		\begin{itemize}
			\item alternatively, Polyak averaging:
					$$ w' \gets \tau w + (1 - \tau)w' $$
			\item $\tau$ is fairly small, e.g, $0.01$
		\end{itemize}
		\item Faster propagation of information compared to original DQN
	\end{itemize}
	
\end{frame}
%-----------------------------------------------------------------------
%----------------------------------------------------------------------
\begin{frame}[c]{Clipped Double DQN \litw{\href{https://arxiv.org/pdf/1802.09477.pdf}{Fujimoto et al. 2018}}}
	
	\begin{itemize}
		\item Extend this idea to DQN
		\item Again having two independent Q-networks with $\vec{w}_1$ and $\vec{w}_2$
		\item Take minimum action value for successor state
		\item TD-error:
		$$r + \gamma \min_{i=\{1,2\}}Q(s', \argmax_{a' \in A} Q(s', a'; \vec{w}); \vec{w}_i) - Q(s,a;\vec{w})$$
		\begin{itemize}
		\item Less overestimation of Q-values
		\item More stable learning targets
		\end{itemize}
	\end{itemize}
\end{frame}
%-----------------------------------------------------------------------
%-----------------------------------------------------------------------
\end{document}
