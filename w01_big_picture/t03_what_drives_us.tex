\documentclass[aspectratio=169]{../latex_main/tntbeamer}  % you can pass all options of the beamer class, e.g., 'handout' or 'aspectratio=43'
\usepackage{dsfont}
\usepackage{bm}
\usepackage[english]{babel}
\usepackage[T1]{fontenc}
%\usepackage[utf8]{inputenc}
\usepackage{graphicx}
\graphicspath{ {./figures/} }
\usepackage{algorithm}
\usepackage[ruled,vlined,algo2e,linesnumbered]{algorithm2e}
\usepackage{hyperref}
\usepackage{booktabs}
\usepackage{mathtools}

\usepackage{amsmath,amssymb}
\usepackage{latexsym}

\DeclareMathOperator*{\argmax}{arg\,max}
\DeclareMathOperator*{\argmin}{arg\,min}

\usepackage{pgfplots}
\pgfplotsset{compat=1.16}
\usepackage{tikz}
\usetikzlibrary{trees} 
\usetikzlibrary{shapes.geometric}
\usetikzlibrary{positioning,shapes,shadows,arrows,calc,mindmap}
\usetikzlibrary{positioning,fadings,through}
\usetikzlibrary{decorations.pathreplacing}
\usetikzlibrary{intersections}
\pgfdeclarelayer{background}
\pgfdeclarelayer{foreground}
\pgfsetlayers{background,main,foreground}
\tikzstyle{activity}=[rectangle, draw=black, rounded corners, text centered, text width=8em]
\tikzstyle{data}=[rectangle, draw=black, text centered, text width=8em]
\tikzstyle{myarrow}=[->, thick, draw=black]

% Define the layers to draw the diagram
\pgfdeclarelayer{background}
\pgfdeclarelayer{foreground}
\pgfsetlayers{background,main,foreground}

\input{./latex_main_old/macros}

% Requires XeLaTeX or LuaLaTeX
\usepackage{unicode-math}

\usepackage{fontspec}
%\setsansfont{Arial}
\setsansfont{RotisSansSerifStd}[ 
Path=./latex_main/fonts/,
Extension = .otf,
UprightFont = *-Regular,  % or *-Light
BoldFont = *-ExtraBold,  % or *-Bold
ItalicFont = *-Italic
]
\setmonofont{Cascadia Mono}[
Scale=0.8
]

% scale factor adapted; mathrm font added (Benjamin Spitschan @TNT, 2021-06-01)
%\setmathfont[Scale=1.05]{Libertinus Math}
%\setmathrm[Scale=1.05]{Libertinus Math}

% other available math fonts are (not exhaustive)
% Latin Modern Math
% XITS Math
% Libertinus Math
% Asana Math
% Fira Math
% TeX Gyre Pagella Math
% TeX Gyre Bonum Math
% TeX Gyre Schola Math
% TeX Gyre Termes Math

% Literature References
% #1 = Display Name
% #2 = Url (without \href)
\newcommand{\lit}[2]{\href{#2}{\footnotesize\color{black!60}[#1]}}

%%% Beamer Customization
%----------------------------------------------------------------------
% (Don't) Show sections in frame header. Options: 'sections', 'sections light', empty
\setbeamertemplate{headline}{empty}

% Add header logo for normal frames
\setheaderimage{
	% \includegraphics[height=\logoheight]{figures/TNT_darkv4.pdf}
	\includegraphics[height=\logoheight]{./latex_main/figures/luh_logo_rgb_0_80_155.pdf}
	% \includegraphics[height=\logoheight]{figures/logo_tntluh.pdf}
}

% Header logo for title page
\settitleheaderimage{
	% \includegraphics[height=\logoheight]{figures/TNT_darkv4.pdf}
	\includegraphics[height=\logoheight]{./latex_main/figures/luh_logo_rgb_0_80_155.pdf}
	% \includegraphics[height=\logoheight]{figures/logo_tntluh.pdf}
}

% Title page: tntdefault 
\setbeamertemplate{title page}[tntdefault]  % or luhstyle
% Add optional title image here
%\addtitlepageimagedefault{\includegraphics[width=0.65\textwidth]{figures/luh_default_presentation_title_image.jpg}}

% Title page: luhstyle
% \setbeamertemplate{title page}[luhstyle]
% % Add optional title image here
% \addtitlepageimage{\includegraphics[width=0.75\textwidth]{figures/luh_default_presentation_title_image.jpg}}

\author[Lindauer]{Marius Lindauer\\[1em]
	\includegraphics[height=\logoheight]{./latex_main/figures/luh_logo_rgb_0_80_155.pdf}\qquad
\includegraphics[height=\logoheight]{./latex_main/figures/TNT_darkv4}\qquad
\includegraphics[height=\logoheight]{./latex_main/figures/L3S.jpg}	}
\date{Winter Term 2021
}


%%% Custom Packages
%----------------------------------------------------------------------
% Create dummy content
\usepackage{blindtext}

% Adds a frame with the current page layout. Just call \layout inside of a frame.
\usepackage{layout}


\title[ML-RL: Big Picture]{RL: Introduction}
\subtitle{What drives us?}



\begin{document}
	
	\maketitle

%----------------------------------------------------------------------
%----------------------------------------------------------------------
\begin{frame}[c]{AutoML: Hyperparameters of an SVM}
	
	\centering
	\includegraphics[width=0.7\textwidth]{images/sklearn_svm_doc.png}
	
\end{frame}
%-----------------------------------------------------------------------	
%----------------------------------------------------------------------
\begin{frame}[c]{Hyperparameter Optimization}
	
	\begin{block}{Definition}
		Let 
		\begin{itemize}
			\item $\conf$ be the hyperparameters of an ML algorithm $\algo$ with domain $\pcs$,
			\pause
			\item $\dataset_{opt}$ be a dataset which is split into $\datasettrain$ and $\datasetval$ 
			\pause
			\item $\cost(\algo_{\conf}, \dataset_{train}, \dataset_{valid})$ denote the cost of $\algo_{\conf}$ trained on $\datasettrain$ and evaluated on $\datasetval$.
		\end{itemize}
		\pause
		The \emph{hyper-parameter optimization (HPO)} problem is to find a hyper-parameter configuration that minimizes this cost:
		\begin{equation}
		\optconf \in \argmin_{\conf \in \pcs} \cost(\algo_{\conf}, \dataset_{train}, \dataset_{valid}) \nonumber  
		\end{equation}
	\end{block}
	
\end{frame}
%-----------------------------------------------------------------------
%----------------------------------------------------------------------
\begin{frame}[c]{Reason I: AutoML for RL}
	
	\begin{itemize}
		\item RL algorithms also have many hyperparameters 
		\item Deep RL depends on the network architecture used 
		\item[$\leadsto$] Performance of RL depends on both\newline \lit{Henderson et al. 2019}{https://arxiv.org/pdf/1709.06560.pdf}, \lit{Engstrom et al. 2020}{https://arxiv.org/pdf/2005.12729.pdf}
		\pause
		\bigskip
		\item Hard to apply AutoML to RL because
		\begin{itemize}
			\item RL agents need a long time to really start learning 
			\item Learning of RL agents is very noisy $\leadsto$ very noisy signal for AutoML
		\end{itemize}
	\end{itemize}
	
\end{frame}
%-----------------------------------------------------------------------
%----------------------------------------------------------------------
\begin{frame}[c]{Reason II: Dynamic Algorithm Configuration}
	
	\vspace{-1em}
	\begin{itemize}
		\item Often we assume that an algorithm runs with some single settings
		\pause
		\item But some settings, e.g., learning rate, have to be dynamically adapted 
	\end{itemize}
	
	\pause
	
	\begin{block}{Definition}
		Let 
		\begin{itemize}
			\item $\conf$ be a hyperparameter configuration of an algorithm $\algo$,
			\pause
			\item $p(\dataset)$ be a probability distribution over datasets $\dataset \in \datasets$,
			\pause
			\item $\stateRL_t$ be a state description of $\algo$ solving $\dataset$ at time point $t$,
			\pause
			\item $\cost: \policies \times \datasets \to \perf$ be a cost metric assessing the \alert{cost of a conf. policy $\pi \in \Pi$} on $\dataset \in \datasets$
		\end{itemize}
		
		\pause
		the \emph{dynamic algorithm configuration problem (DAC)} is to obtain a configuration policy $\policy^* : \stateRL_t \times \dataset \mapsto \conf$ by optimizing its cost across a distribution of datasets:
		\begin{equation}
		\pi^* \in \argmin_{\policy \in \policies} \int_{\datasets} p(\dataset) c(\policy, \dataset) \diff \dataset \nonumber
		\end{equation}
	\end{block}
	
\end{frame}
%-----------------------------------------------------------------------	
%----------------------------------------------------------------------
\begin{frame}[c]{RL for Dynamic Algorithm Configuration}
	
	\begin{itemize}
		\item[$\leadsto$] We learn $\pi$ via RL! 
		\item We showed that:
		\begin{itemize}
			\item Dynamic Algorithm Configuration can be formulated as a RL problem~\lit{Biedenkapp et al. 2020}{https://www.tnt.uni-hannover.de/papers/data/1432/20-ECAI-DAC.pdf}
			\item Heuristics of planning solvers can be automatically and dynamically selected~\lit{Speck et al. 2020}{https://arxiv.org/abs/2006.08246}
			\item We can use a teacher (i.e., existing heuristics) to efficiently learn step size settings of CMA-ES~\lit{Shala et al. 2020}{https://ml.informatik.uni-freiburg.de/papers/20-PPSN-LTO-CMA.pdf}
			\item We can speed up learning by repeating actions~\lit{Biedenkapp et al. 2020}{https://www.tnt.uni-hannover.de/papers/data/1455/20-BIG-TempoRL.pdf}
			\item We can speed up learning by learning an efficient schedule of task instances~\lit{Eimer et al. 2020}{https://www.tnt.uni-hannover.de/papers/data/1454/space.pdf}
		\end{itemize}
	\end{itemize}

	
\end{frame}
%-----------------------------------------------------------------------	
%-----------------------------------------------------------------------
\end{document}
