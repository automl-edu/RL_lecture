% !TeX spellcheck = en_US

\documentclass[aspectratio=169]{tntbeamer}  % you can pass all options of the beamer class, e.g., 'handout' or 'aspectratio=43'
\usepackage{dsfont}
\usepackage{bm}
\usepackage[english]{babel}
\usepackage[T1]{fontenc}
%\usepackage[utf8]{inputenc}
\usepackage{graphicx}
\graphicspath{ {./figures/} }
\usepackage{algorithm}
\usepackage[ruled,vlined,algo2e,linesnumbered]{algorithm2e}
\usepackage{hyperref}
\usepackage{booktabs}
\usepackage{mathtools}

\usepackage{amsmath,amssymb}
\usepackage{latexsym}

\DeclareMathOperator*{\argmax}{arg\,max}
\DeclareMathOperator*{\argmin}{arg\,min}

\usepackage{pgfplots}
\pgfplotsset{compat=1.16}
\usepackage{tikz}
\usetikzlibrary{trees} 
\usetikzlibrary{shapes.geometric}
\usetikzlibrary{positioning,shapes,shadows,arrows,calc,mindmap}
\usetikzlibrary{positioning,fadings,through}
\usetikzlibrary{decorations.pathreplacing}
\usetikzlibrary{intersections}
\pgfdeclarelayer{background}
\pgfdeclarelayer{foreground}
\pgfsetlayers{background,main,foreground}
\tikzstyle{activity}=[rectangle, draw=black, rounded corners, text centered, text width=8em]
\tikzstyle{data}=[rectangle, draw=black, text centered, text width=8em]
\tikzstyle{myarrow}=[->, thick, draw=black]

% Define the layers to draw the diagram
\pgfdeclarelayer{background}
\pgfdeclarelayer{foreground}
\pgfsetlayers{background,main,foreground}

\input{./latex_main_old/macros}

% Requires XeLaTeX or LuaLaTeX
\usepackage{unicode-math}

\usepackage{fontspec}
%\setsansfont{Arial}
\setsansfont{RotisSansSerifStd}[ 
Path=./latex_main/fonts/,
Extension = .otf,
UprightFont = *-Regular,  % or *-Light
BoldFont = *-ExtraBold,  % or *-Bold
ItalicFont = *-Italic
]
\setmonofont{Cascadia Mono}[
Scale=0.8
]

% scale factor adapted; mathrm font added (Benjamin Spitschan @TNT, 2021-06-01)
%\setmathfont[Scale=1.05]{Libertinus Math}
%\setmathrm[Scale=1.05]{Libertinus Math}

% other available math fonts are (not exhaustive)
% Latin Modern Math
% XITS Math
% Libertinus Math
% Asana Math
% Fira Math
% TeX Gyre Pagella Math
% TeX Gyre Bonum Math
% TeX Gyre Schola Math
% TeX Gyre Termes Math

% Literature References
% #1 = Display Name
% #2 = Url (without \href)
\newcommand{\lit}[2]{\href{#2}{\footnotesize\color{black!60}[#1]}}

%%% Beamer Customization
%----------------------------------------------------------------------
% (Don't) Show sections in frame header. Options: 'sections', 'sections light', empty
\setbeamertemplate{headline}{empty}

% Add header logo for normal frames
\setheaderimage{
	% \includegraphics[height=\logoheight]{figures/TNT_darkv4.pdf}
	\includegraphics[height=\logoheight]{./latex_main/figures/luh_logo_rgb_0_80_155.pdf}
	% \includegraphics[height=\logoheight]{figures/logo_tntluh.pdf}
}

% Header logo for title page
\settitleheaderimage{
	% \includegraphics[height=\logoheight]{figures/TNT_darkv4.pdf}
	\includegraphics[height=\logoheight]{./latex_main/figures/luh_logo_rgb_0_80_155.pdf}
	% \includegraphics[height=\logoheight]{figures/logo_tntluh.pdf}
}

% Title page: tntdefault 
\setbeamertemplate{title page}[tntdefault]  % or luhstyle
% Add optional title image here
%\addtitlepageimagedefault{\includegraphics[width=0.65\textwidth]{figures/luh_default_presentation_title_image.jpg}}

% Title page: luhstyle
% \setbeamertemplate{title page}[luhstyle]
% % Add optional title image here
% \addtitlepageimage{\includegraphics[width=0.75\textwidth]{figures/luh_default_presentation_title_image.jpg}}

\author[Lindauer]{Marius Lindauer\\[1em]
	\includegraphics[height=\logoheight]{./latex_main/figures/luh_logo_rgb_0_80_155.pdf}\qquad
\includegraphics[height=\logoheight]{./latex_main/figures/TNT_darkv4}\qquad
\includegraphics[height=\logoheight]{./latex_main/figures/L3S.jpg}	}
\date{Winter Term 2021
}


%%% Custom Packages
%----------------------------------------------------------------------
% Create dummy content
\usepackage{blindtext}

% Adds a frame with the current page layout. Just call \layout inside of a frame.
\usepackage{layout}


\institute{Institut f\"ur Informationsverarbeitung}%\\ Leibniz Universit\"at Hannover}
%\title{TNT Beamer Template}
%\author{Suomynon A. Anonymous}
\date{}



\title[Reinforcement Learning: Big Picture]{RL: Policy Gradient}
\subtitle{The Big Picture}



\begin{document}
	
	\maketitle

%----------------------------------------------------------------------
%----------------------------------------------------------------------
\begin{frame}[c]{Policy-Based Reinforcement Learning}

\begin{itemize}
	\item In the last lecture we approximated the value or action-value function
	using parameters $\vec{w}$,
	$$V_{\vec{w}}(s) \approx V^\pi(s)$$
	$$Q_{\vec{w}}(s,a) \approx Q^\pi (s,a) $$
	\item A policy was generated directly from the value function
	\begin{itemize}
		\item e.g., using $\epsilon$-greedy
	\end{itemize}	
	\item Now, we will directly parametrize the policy, and will typically
	use $\theta$ to show parameterization:
	$$\pi_\theta (s,a) = \mathbb{P}[ a\mid s; \theta] $$
	\item Goal is to find a policy $\pi$ with the highest value function $V^\pi$
	\item We will focus again on model-free reinforcement learning
\end{itemize}

\end{frame}
%-----------------------------------------------------------------------
%----------------------------------------------------------------------
\begin{frame}[c]{Value-Based and Policy-Based RL}

	\begin{itemize}
		\item Value-based
		\begin{itemize}
			\item Learn Value function
			\item implicit policy (e.g., $\epsilon$-greedy)
		\end{itemize}
		\item Policy-based
		\begin{itemize}
			\item No explicit value function
			\item learnt policy
		\end{itemize}
		\item Actor-Critic
		\begin{itemize}
			\item Learn Value Function
			\item Learn Policy
		\end{itemize}
		
	\end{itemize}		
	
\end{frame}
%-----------------------------------------------------------------------
%----------------------------------------------------------------------
\begin{frame}[c]{Types of Policies to Search Over}
	
\begin{itemize}
	\item So far have focused on deterministic policies
	\item Now we are thinking about direct policy search in RL, will focus
	heavily on stochastic policies
\end{itemize}

\end{frame}
%-----------------------------------------------------------------------
%----------------------------------------------------------------------
\begin{frame}[c]{Example: Rock-Paper-Scissors}
	
	\begin{itemize}
		\item Two-player game of rock-paper-scissors
		\begin{itemize}
			\item Scissors beats paper
			\item Rock beats scissors
			\item Paper beats rock
		\end{itemize}
		\item Let state be history of prior actions (rock, paper and scissors) and if
		won or lost
		\item Is deterministic policy optimal? Why or why not?
		\item[$\leadsto$] stochastic (random) policy is the Nash equilibrium
	\end{itemize}

	
\end{frame}
%-----------------------------------------------------------------------
%----------------------------------------------------------------------
\begin{frame}[c]{Example: Aliased Gridword (1)}
	
	\begin{itemize}
		\item The agent \alert{cannot} differentiate the gray states
		\item Consider features of the following form (for all N, E, S, W)
		$$\phi(s,a) = 1\text{(s="wall to N", a = "move E")}$$
		\vspace{-1em}
		\begin{itemize}
			\item State representation is not Markov
		\end{itemize}
		\item Compare value-based RL, using an approximate value function
		$$Q_\theta (s,a) = f(\phi(s,a); \theta)$$
		\item To policy-based RL, using a parametrized policy
		$$\pi_\theta(s,a) = g(\phi(s,a); \theta) $$ 
	\end{itemize}
	
\end{frame}
%-----------------------------------------------------------------------
%----------------------------------------------------------------------
\begin{frame}[c]{Example: Aliased Gridworld (2)}
	
	\begin{itemize}
		\item Under aliasing, an optimal \alert{deterministic} policy will either
		\begin{itemize}
			\item Move W in both gray states 
			\item move E in both gray states
		\end{itemize}
		\item Either way, it can get stuck and never reach the money
		\item Value-based RL learns a near-deterministic policy
		\item So it will traverse the corridor for a long time
	\end{itemize}

\end{frame}
%-----------------------------------------------------------------------
%----------------------------------------------------------------------
\begin{frame}[c]{Example: Aliased Gridworld (3)}
	
	\begin{itemize}
		\item An optimal \alert{stochastic} policy will randomly move E or W in grey states
		$$\pi_\theta\text{(wall to N and S, move E)} = 0.5 $$
		$$\pi_\theta\text{(wall to N and S, move W)} = 0.5 $$
		\item It will reach the goal state in a few steps with high probability
		\item Policy-based RL can learn the optimal stochastic policy
	\end{itemize}
	
\end{frame}
%-----------------------------------------------------------------------
%-----------------------------------------------------------------------
\end{document}
