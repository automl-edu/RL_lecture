% !TeX spellcheck = en_US
\documentclass[aspectratio=169]{../latex_main/tntbeamer}  % you can pass all options of the beamer class, e.g., 'handout' or 'aspectratio=43'
\usepackage{dsfont}
\usepackage{bm}
\usepackage[english]{babel}
\usepackage[T1]{fontenc}
%\usepackage[utf8]{inputenc}
\usepackage{graphicx}
\graphicspath{ {./figures/} }
\usepackage{algorithm}
\usepackage[ruled,vlined,algo2e,linesnumbered]{algorithm2e}
\usepackage{hyperref}
\usepackage{booktabs}
\usepackage{mathtools}

\usepackage{amsmath,amssymb}
\usepackage{latexsym}

\DeclareMathOperator*{\argmax}{arg\,max}
\DeclareMathOperator*{\argmin}{arg\,min}

\usepackage{pgfplots}
\pgfplotsset{compat=1.16}
\usepackage{tikz}
\usetikzlibrary{trees} 
\usetikzlibrary{shapes.geometric}
\usetikzlibrary{positioning,shapes,shadows,arrows,calc,mindmap}
\usetikzlibrary{positioning,fadings,through}
\usetikzlibrary{decorations.pathreplacing}
\usetikzlibrary{intersections}
\pgfdeclarelayer{background}
\pgfdeclarelayer{foreground}
\pgfsetlayers{background,main,foreground}
\tikzstyle{activity}=[rectangle, draw=black, rounded corners, text centered, text width=8em]
\tikzstyle{data}=[rectangle, draw=black, text centered, text width=8em]
\tikzstyle{myarrow}=[->, thick, draw=black]

% Define the layers to draw the diagram
\pgfdeclarelayer{background}
\pgfdeclarelayer{foreground}
\pgfsetlayers{background,main,foreground}

\input{./latex_main_old/macros}

% Requires XeLaTeX or LuaLaTeX
\usepackage{unicode-math}

\usepackage{fontspec}
%\setsansfont{Arial}
\setsansfont{RotisSansSerifStd}[ 
Path=./latex_main/fonts/,
Extension = .otf,
UprightFont = *-Regular,  % or *-Light
BoldFont = *-ExtraBold,  % or *-Bold
ItalicFont = *-Italic
]
\setmonofont{Cascadia Mono}[
Scale=0.8
]

% scale factor adapted; mathrm font added (Benjamin Spitschan @TNT, 2021-06-01)
%\setmathfont[Scale=1.05]{Libertinus Math}
%\setmathrm[Scale=1.05]{Libertinus Math}

% other available math fonts are (not exhaustive)
% Latin Modern Math
% XITS Math
% Libertinus Math
% Asana Math
% Fira Math
% TeX Gyre Pagella Math
% TeX Gyre Bonum Math
% TeX Gyre Schola Math
% TeX Gyre Termes Math

% Literature References
% #1 = Display Name
% #2 = Url (without \href)
\newcommand{\lit}[2]{\href{#2}{\footnotesize\color{black!60}[#1]}}

%%% Beamer Customization
%----------------------------------------------------------------------
% (Don't) Show sections in frame header. Options: 'sections', 'sections light', empty
\setbeamertemplate{headline}{empty}

% Add header logo for normal frames
\setheaderimage{
	% \includegraphics[height=\logoheight]{figures/TNT_darkv4.pdf}
	\includegraphics[height=\logoheight]{./latex_main/figures/luh_logo_rgb_0_80_155.pdf}
	% \includegraphics[height=\logoheight]{figures/logo_tntluh.pdf}
}

% Header logo for title page
\settitleheaderimage{
	% \includegraphics[height=\logoheight]{figures/TNT_darkv4.pdf}
	\includegraphics[height=\logoheight]{./latex_main/figures/luh_logo_rgb_0_80_155.pdf}
	% \includegraphics[height=\logoheight]{figures/logo_tntluh.pdf}
}

% Title page: tntdefault 
\setbeamertemplate{title page}[tntdefault]  % or luhstyle
% Add optional title image here
%\addtitlepageimagedefault{\includegraphics[width=0.65\textwidth]{figures/luh_default_presentation_title_image.jpg}}

% Title page: luhstyle
% \setbeamertemplate{title page}[luhstyle]
% % Add optional title image here
% \addtitlepageimage{\includegraphics[width=0.75\textwidth]{figures/luh_default_presentation_title_image.jpg}}

\author[Lindauer]{Marius Lindauer\\[1em]
	\includegraphics[height=\logoheight]{./latex_main/figures/luh_logo_rgb_0_80_155.pdf}\qquad
\includegraphics[height=\logoheight]{./latex_main/figures/TNT_darkv4}\qquad
\includegraphics[height=\logoheight]{./latex_main/figures/L3S.jpg}	}
\date{Winter Term 2021
}


%%% Custom Packages
%----------------------------------------------------------------------
% Create dummy content
\usepackage{blindtext}

% Adds a frame with the current page layout. Just call \layout inside of a frame.
\usepackage{layout}

\title[RL: Analytic Gradient]{RL: Policy Search}
\subtitle{Analytic Gradient}



\begin{document}
	
	\maketitle

%----------------------------------------------------------------------
%----------------------------------------------------------------------
\begin{frame}[c]{Computing the Gradient Analytically}

\begin{itemize}
	\item We now compute the policy gradient analytically
	\item Assume policy $\pi_\theta$ is differentiable whenever it is non-zero\\
	 and we know the gradient $\nabla_\theta \pi_\theta (s,a)$, for example:
	 \begin{itemize}
	     \item softmax: $\pi_\theta (s,a) = \exp(\phi(s,a)^\intercal\theta) / \sum_{a} \exp(\phi(s,a)^\intercal\theta)$
	     \item Gaussian: $\pi(s,a) \sim \mathcal{N}(\mu(s), \sigma^2)$ with $\mu(s) = \phi(s)^\intercal \theta$
	     \item or a DNN
	 \end{itemize}
	\item Denote a state-action trajectory as 
	$$ \tau = (s_0, a_0, r_0, \ldots, s_{T-1}, a_{T-1}, r_{T-1}, s_T) $$
	\item Use $R(\tau) = \sum_{t=0}^{T} R(s_t, a_t)$ to be the sum of rewards for a trajectory~$\tau$
	\item[$\leadsto$] Focusing for now on $V(s_0; \theta) = \sum_\tau P(\tau; \theta) R(\tau)$ 
	
\end{itemize}

\end{frame}
%-----------------------------------------------------------------------
%----------------------------------------------------------------------
\begin{frame}[c]{Value of a Parameterized Policy}

\begin{itemize}
	\item Assume policy $\pi_\theta$ is differentiable whenever it is non-zero\\
	 and we know the gradient $\nabla_\theta \pi_\theta (s,a)$
	\item Recall policy value is $V(s_0; \theta) = \mathrm{E}_{\pi_\theta}\left[ \sum_{t} r(s_t, a_t); \pi_\theta, s_0 \right]$
        \begin{itemize}
	   \item where the expectation is taken over the states and actions visited by $\pi_\theta$
        \end{itemize}
	\item We can re-express this in multiple ways
	\begin{itemize}
	    \item $V(s_0;\theta) = \sum_a \pi_\theta (a \mid s_0) Q(s_0, a; \theta)$
	    \item $V(s_0; \theta) = \sum_\tau P(\tau; \theta) R(\tau)$ with $\tau$ being an trajectory
	\end{itemize}
	\item[$\leadsto$] We will focus on the latter definition.
\end{itemize}

\end{frame}
%-----------------------------------------------------------------------
%----------------------------------------------------------------------
\begin{frame}[c]{Likelihood Ratio Policy Gradient I }
	
	\begin{itemize}
		\item Denote a state-action trajectory as 
		$$ \tau = (s_0, a_0, r_0, \ldots, s_{T-1}, a_{T-1}, r_{T-1}, s_T) $$
		\item Use $R(\tau) = \sum_{t=0}^{T} R(s_t, a_t)$ to be the sum of rewards for $\tau$
		\item Policy value is 
		$$V(\theta) = \mathbb{E}\left[\sum_{t=0}^T R(s_t, a_t); \pi_\theta \right] = \sum_{\tau} P(\tau; \theta)R(\tau) $$
		where $P(\tau; \theta)$ is used to denote the probability over trajectories when executing policy $\pi_\theta$
		\item In this notation, our goal is to find the policy parameters $\theta^*$
		$$\theta^* \in \argmax_{\theta} V(\theta) = \argmax_{\theta}\sum_{\tau} P(\tau; \theta) R(\tau) $$
		
	\end{itemize}
	
\end{frame}
%-----------------------------------------------------------------------
%----------------------------------------------------------------------
\begin{frame}[c]{Likelihood Ratio Policy Gradient II}
	\vspace{-1.5em}
	\begin{itemize}
		\item Our goal is to find the policy parameters $\theta^*$
		$$\theta^* \in \argmax_{\theta} V(\theta) = \argmax_{\theta}\sum_{\tau} P(\tau; \theta) R(\tau) $$
		\item Take the gradient with respect to $\theta$:
		\small
		\begin{eqnarray*}
		\nabla_\theta V(\theta) &=& \nabla_\theta \sum_{\tau} P(\tau; \theta)R(\tau)\\ \pause
		&=& \sum_{\tau} \nabla_\theta P(\tau; \theta)R(\tau)\\ \pause
		&=& \sum_{\tau} \frac{P(\tau; \theta)}{P(\tau; \theta)} \nabla_\theta P(\tau; \theta)R(\tau)\\ \pause
		&=& \sum_{\tau} P(\tau; \theta)  R(\tau) \underbrace{\frac{\nabla_\theta P(\tau; \theta)}{P(\tau; \theta)}}_{\text{likelihood ratio}} \\ \pause
		&=& \sum_{\tau} P(\tau; \theta)  R(\tau) \nabla_\theta \log P(\tau; \theta)
		\end{eqnarray*} 
		
	\end{itemize}
	
\end{frame}
%-----------------------------------------------------------------------
%----------------------------------------------------------------------
\begin{frame}[c]{Likelihood Ratio Policy Gradient III}
	
	\begin{itemize}
		\item Our goal is to find the policy parameters $\theta^*$
		$$\theta^* \in \argmax_{\theta} V(\theta) = \argmax_{\theta}\sum_{\tau} P(\tau; \theta) R(\tau) $$
		\item Take the gradient with respect to $\theta$:
		$$ \nabla_\theta V(\theta) = \sum_{\tau} P(\tau; \theta)  R(\tau) \nabla_\theta \log P(\tau; \theta) $$
		\item Approximate with empirical estimate for $m$ sample trajectories under
		policy $\pi_\theta$:
		$$\nabla_\theta V(\theta) \approx \frac{1}{m} \sum_{i=1}^{m} R(\tau^{(i)}) \nabla_\theta \log P(\tau^{(i)}; \theta) $$
		
	\end{itemize}
	
\end{frame}
%-----------------------------------------------------------------------
%----------------------------------------------------------------------
\begin{frame}[c]{Decomposing the Trajectories Into States and Actions}
	
	\begin{itemize}
		\item Approximate with empirical estimate for $m$ sample trajectories under
		policy $\pi_\theta$:
		$$\nabla_\theta V(\theta) \approx \frac{1}{m} \sum_{i=1}^{m} R(\tau^{(i)}) \nabla_\theta \log P(\tau^{(i)}; \theta) $$
	\end{itemize}

{\footnotesize
\begin{eqnarray*}
\nabla_\theta \log P(\tau^{(i)}; \theta) &=& \nabla_\theta \log \left[ \rho(s_0) \prod_{t=0}^{T-1} \pi_\theta(a_t \mid s_t) P(s_{t+1} \mid s_t, a_t) \right]\\
&=& \nabla_\theta \left[ \log \rho(s_0) + \sum_{t=0}^{T-1} \log \pi_\theta(a_t \mid s_t) + \log P(s_{t+1}\mid s_t, a_t) \right]\\
&=& \sum_{t=0}^{T-1} \nabla_\theta \log \pi_\theta (a_t \mid s_t)
\end{eqnarray*}}
	
$\leadsto$ \alert{No dynamics model required!}
	
\end{frame}
%-----------------------------------------------------------------------
\end{document}
