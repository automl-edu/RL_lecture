% !TeX spellcheck = en_US
\documentclass[aspectratio=169]{../latex_main/tntbeamer}  % you can pass all options of the beamer class, e.g., 'handout' or 'aspectratio=43'
\usepackage{dsfont}
\usepackage{bm}
\usepackage[english]{babel}
\usepackage[T1]{fontenc}
%\usepackage[utf8]{inputenc}
\usepackage{graphicx}
\graphicspath{ {./figures/} }
\usepackage{algorithm}
\usepackage[ruled,vlined,algo2e,linesnumbered]{algorithm2e}
\usepackage{hyperref}
\usepackage{booktabs}
\usepackage{mathtools}

\usepackage{amsmath,amssymb}
\usepackage{latexsym}

\DeclareMathOperator*{\argmax}{arg\,max}
\DeclareMathOperator*{\argmin}{arg\,min}

\usepackage{pgfplots}
\pgfplotsset{compat=1.16}
\usepackage{tikz}
\usetikzlibrary{trees} 
\usetikzlibrary{shapes.geometric}
\usetikzlibrary{positioning,shapes,shadows,arrows,calc,mindmap}
\usetikzlibrary{positioning,fadings,through}
\usetikzlibrary{decorations.pathreplacing}
\usetikzlibrary{intersections}
\pgfdeclarelayer{background}
\pgfdeclarelayer{foreground}
\pgfsetlayers{background,main,foreground}
\tikzstyle{activity}=[rectangle, draw=black, rounded corners, text centered, text width=8em]
\tikzstyle{data}=[rectangle, draw=black, text centered, text width=8em]
\tikzstyle{myarrow}=[->, thick, draw=black]

% Define the layers to draw the diagram
\pgfdeclarelayer{background}
\pgfdeclarelayer{foreground}
\pgfsetlayers{background,main,foreground}

\input{./latex_main_old/macros}

% Requires XeLaTeX or LuaLaTeX
\usepackage{unicode-math}

\usepackage{fontspec}
%\setsansfont{Arial}
\setsansfont{RotisSansSerifStd}[ 
Path=./latex_main/fonts/,
Extension = .otf,
UprightFont = *-Regular,  % or *-Light
BoldFont = *-ExtraBold,  % or *-Bold
ItalicFont = *-Italic
]
\setmonofont{Cascadia Mono}[
Scale=0.8
]

% scale factor adapted; mathrm font added (Benjamin Spitschan @TNT, 2021-06-01)
%\setmathfont[Scale=1.05]{Libertinus Math}
%\setmathrm[Scale=1.05]{Libertinus Math}

% other available math fonts are (not exhaustive)
% Latin Modern Math
% XITS Math
% Libertinus Math
% Asana Math
% Fira Math
% TeX Gyre Pagella Math
% TeX Gyre Bonum Math
% TeX Gyre Schola Math
% TeX Gyre Termes Math

% Literature References
% #1 = Display Name
% #2 = Url (without \href)
\newcommand{\lit}[2]{\href{#2}{\footnotesize\color{black!60}[#1]}}

%%% Beamer Customization
%----------------------------------------------------------------------
% (Don't) Show sections in frame header. Options: 'sections', 'sections light', empty
\setbeamertemplate{headline}{empty}

% Add header logo for normal frames
\setheaderimage{
	% \includegraphics[height=\logoheight]{figures/TNT_darkv4.pdf}
	\includegraphics[height=\logoheight]{./latex_main/figures/luh_logo_rgb_0_80_155.pdf}
	% \includegraphics[height=\logoheight]{figures/logo_tntluh.pdf}
}

% Header logo for title page
\settitleheaderimage{
	% \includegraphics[height=\logoheight]{figures/TNT_darkv4.pdf}
	\includegraphics[height=\logoheight]{./latex_main/figures/luh_logo_rgb_0_80_155.pdf}
	% \includegraphics[height=\logoheight]{figures/logo_tntluh.pdf}
}

% Title page: tntdefault 
\setbeamertemplate{title page}[tntdefault]  % or luhstyle
% Add optional title image here
%\addtitlepageimagedefault{\includegraphics[width=0.65\textwidth]{figures/luh_default_presentation_title_image.jpg}}

% Title page: luhstyle
% \setbeamertemplate{title page}[luhstyle]
% % Add optional title image here
% \addtitlepageimage{\includegraphics[width=0.75\textwidth]{figures/luh_default_presentation_title_image.jpg}}

\author[Lindauer]{Marius Lindauer\\[1em]
	\includegraphics[height=\logoheight]{./latex_main/figures/luh_logo_rgb_0_80_155.pdf}\qquad
\includegraphics[height=\logoheight]{./latex_main/figures/TNT_darkv4}\qquad
\includegraphics[height=\logoheight]{./latex_main/figures/L3S.jpg}	}
\date{Winter Term 2021
}


%%% Custom Packages
%----------------------------------------------------------------------
% Create dummy content
\usepackage{blindtext}

% Adds a frame with the current page layout. Just call \layout inside of a frame.
\usepackage{layout}

\title[RL: Score Function]{RL: Policy Search}
\subtitle{Score Function}



\begin{document}
	
	\maketitle

%----------------------------------------------------------------------
%----------------------------------------------------------------------
\begin{frame}[c]{Likelihood Ratio + Score Function Policy Gradient}
	
	\begin{itemize}
		\item Putting everything together from the last slide deck
		\item Our goal is to find the policy parameters $\theta^*$
		$$\theta^* \in \argmax_{\theta} V(\theta) = \argmax_{\theta}\sum_{\tau} P(\tau; \theta) R(\tau) $$
		\item Approximate with empirical estimate for $m$ sample trajectories under
		policy $\pi_\theta$:
		\begin{eqnarray}
		\nabla_\theta V(\theta) &\approx& \frac{1}{m} \sum_{i=1}^{m} R(\tau^{(i)}) \nabla_\theta \log P(\tau^{(i)}; \theta) \nonumber\\
		&=& \frac{1}{m} \sum_{i=1}^{m} R(\tau^{(i)}) \sum_{t=0}^{T-1} \underbrace{\nabla_\theta \log \pi_\theta (a_t^{(i)} \mid s_t^{(i)})}_{\text{Score Function}}
		\end{eqnarray}
		\item[$\leadsto$] Do not need to know dynamics model!
		
	\end{itemize}
	
\end{frame}
%-----------------------------------------------------------------------
%----------------------------------------------------------------------
\begin{frame}[c]{Score Function Gradient Estimator: Intuition}
	
	\begin{itemize}
		\item Consider generic form of $R(\tau^{(i)}) \nabla_\theta \log P(\tau^{(i)}; \theta)$:\\
		$\hat{g}_i = f(x_i) \nabla_\theta \log p(x_i \mid \theta)$
		\item $f(x)$ measures how good the sample $x$ is
		\item Moving in the direction $\hat{g}_i$ pushes up to the logprob of the sample,\\ in proportion of how good it is
		\item Valid even if $f(x)$ is discontinuous; and unknown;\\ or sample space (containing $x$) is a discrete set
	\end{itemize}
	
\end{frame}
%-----------------------------------------------------------------------
%----------------------------------------------------------------------
\begin{frame}[c]{Score Function Gradient Estimator: Intuition}
	

\centering
$\hat{g}_i = f(x_i) \nabla_\theta \log p(x_i \mid \theta)$
\bigskip

\includegraphics[width=0.4\textwidth]{images/scoring_function_1.png}
\pause
\includegraphics[width=0.4\textwidth]{images/scoring_function_2.png}
	
\end{frame}
%-----------------------------------------------------------------------
%----------------------------------------------------------------------
\begin{frame}[c]{Policy Gradient Theorem}
	
The policy gradient theorem generalizes the likelihood ratio approach:
\begin{block}{Theorem}
For any differentiable policy $\pi_\theta$, the policy gradient is
$$\nabla_\theta V(\theta) = \mathbb{E}_{\pi_\theta} [\nabla_\theta \log \pi_\theta(s,a) Q^{\pi_\theta}(s,a)] $$

\end{block}
	
\end{frame}
%-----------------------------------------------------------------------
\end{document}
