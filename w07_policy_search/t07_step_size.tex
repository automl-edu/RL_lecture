% !TeX spellcheck = en_US
\documentclass[aspectratio=169]{../latex_main/tntbeamer}  % you can pass all options of the beamer class, e.g., 'handout' or 'aspectratio=43'
\usepackage{dsfont}
\usepackage{bm}
\usepackage[english]{babel}
\usepackage[T1]{fontenc}
%\usepackage[utf8]{inputenc}
\usepackage{graphicx}
\graphicspath{ {./figures/} }
\usepackage{algorithm}
\usepackage[ruled,vlined,algo2e,linesnumbered]{algorithm2e}
\usepackage{hyperref}
\usepackage{booktabs}
\usepackage{mathtools}

\usepackage{amsmath,amssymb}
\usepackage{latexsym}

\DeclareMathOperator*{\argmax}{arg\,max}
\DeclareMathOperator*{\argmin}{arg\,min}

\usepackage{pgfplots}
\pgfplotsset{compat=1.16}
\usepackage{tikz}
\usetikzlibrary{trees} 
\usetikzlibrary{shapes.geometric}
\usetikzlibrary{positioning,shapes,shadows,arrows,calc,mindmap}
\usetikzlibrary{positioning,fadings,through}
\usetikzlibrary{decorations.pathreplacing}
\usetikzlibrary{intersections}
\pgfdeclarelayer{background}
\pgfdeclarelayer{foreground}
\pgfsetlayers{background,main,foreground}
\tikzstyle{activity}=[rectangle, draw=black, rounded corners, text centered, text width=8em]
\tikzstyle{data}=[rectangle, draw=black, text centered, text width=8em]
\tikzstyle{myarrow}=[->, thick, draw=black]

% Define the layers to draw the diagram
\pgfdeclarelayer{background}
\pgfdeclarelayer{foreground}
\pgfsetlayers{background,main,foreground}

\input{./latex_main_old/macros}

% Requires XeLaTeX or LuaLaTeX
\usepackage{unicode-math}

\usepackage{fontspec}
%\setsansfont{Arial}
\setsansfont{RotisSansSerifStd}[ 
Path=./latex_main/fonts/,
Extension = .otf,
UprightFont = *-Regular,  % or *-Light
BoldFont = *-ExtraBold,  % or *-Bold
ItalicFont = *-Italic
]
\setmonofont{Cascadia Mono}[
Scale=0.8
]

% scale factor adapted; mathrm font added (Benjamin Spitschan @TNT, 2021-06-01)
%\setmathfont[Scale=1.05]{Libertinus Math}
%\setmathrm[Scale=1.05]{Libertinus Math}

% other available math fonts are (not exhaustive)
% Latin Modern Math
% XITS Math
% Libertinus Math
% Asana Math
% Fira Math
% TeX Gyre Pagella Math
% TeX Gyre Bonum Math
% TeX Gyre Schola Math
% TeX Gyre Termes Math

% Literature References
% #1 = Display Name
% #2 = Url (without \href)
\newcommand{\lit}[2]{\href{#2}{\footnotesize\color{black!60}[#1]}}

%%% Beamer Customization
%----------------------------------------------------------------------
% (Don't) Show sections in frame header. Options: 'sections', 'sections light', empty
\setbeamertemplate{headline}{empty}

% Add header logo for normal frames
\setheaderimage{
	% \includegraphics[height=\logoheight]{figures/TNT_darkv4.pdf}
	\includegraphics[height=\logoheight]{./latex_main/figures/luh_logo_rgb_0_80_155.pdf}
	% \includegraphics[height=\logoheight]{figures/logo_tntluh.pdf}
}

% Header logo for title page
\settitleheaderimage{
	% \includegraphics[height=\logoheight]{figures/TNT_darkv4.pdf}
	\includegraphics[height=\logoheight]{./latex_main/figures/luh_logo_rgb_0_80_155.pdf}
	% \includegraphics[height=\logoheight]{figures/logo_tntluh.pdf}
}

% Title page: tntdefault 
\setbeamertemplate{title page}[tntdefault]  % or luhstyle
% Add optional title image here
%\addtitlepageimagedefault{\includegraphics[width=0.65\textwidth]{figures/luh_default_presentation_title_image.jpg}}

% Title page: luhstyle
% \setbeamertemplate{title page}[luhstyle]
% % Add optional title image here
% \addtitlepageimage{\includegraphics[width=0.75\textwidth]{figures/luh_default_presentation_title_image.jpg}}

\author[Lindauer]{Marius Lindauer\\[1em]
	\includegraphics[height=\logoheight]{./latex_main/figures/luh_logo_rgb_0_80_155.pdf}\qquad
\includegraphics[height=\logoheight]{./latex_main/figures/TNT_darkv4}\qquad
\includegraphics[height=\logoheight]{./latex_main/figures/L3S.jpg}	}
\date{Winter Term 2021
}


%%% Custom Packages
%----------------------------------------------------------------------
% Create dummy content
\usepackage{blindtext}

% Adds a frame with the current page layout. Just call \layout inside of a frame.
\usepackage{layout}

\title[RL: Policy Gradient]{RL: Policy Search}
\subtitle{Step Size and Trust Region}


\begin{document}
	
	\maketitle

%----------------------------------------------------------------------
%----------------------------------------------------------------------
\begin{frame}[c]{Policy Gradient and Step Size}
	
    \begin{itemize}
        \item Goal: Each step of policy gradient yields an updated policy $\pi'$ whose value is greater than (or equal) to the prior policy $\pi$: $V^{\pi'} \geq V^\pi$
        \begin{itemize}
            \item Monotonic improvement
            \item Important in some applications
            \item PG often more stable than DQN
        \end{itemize} 
        \item Gradient descent approaches update the weights a small step in direction of gradient
        \item First order / linear approximation of the value function's dependence on the policy parameterization
        \item Locally a good approximation; further aways less good
    \end{itemize}

\end{frame}
%-----------------------------------------------------------------------
%----------------------------------------------------------------------
\begin{frame}[c]{Why are step sizes a big deal in RL?}
	
    \begin{itemize}
        \item Step size is important in any problem involving the optima of a function
        \item Supervised learning: Step too far $\leadsto$ next updates might fix it
        \item Reinforcement learning:
        \begin{itemize}
            \item Step too far $\leadsto$ bad policy
            \item Next batch: collected under bad policy
            \item \alert{Policy is determining data collection!} 
            \begin{itemize}
                \item Essentially controlling exploration and exploitation trade-off due to particular policy parameters and the stochasticity of the policy
            \end{itemize}
            \item May not be able to recover from a bad choice $\leadsto$ collapse in performance
        \end{itemize}
    \end{itemize}

\end{frame}
%-----------------------------------------------------------------------
%----------------------------------------------------------------------
\begin{frame}[c]{Policy Gradient Methods with Auto-Step-Size Selection}
	
    \begin{itemize}
        \item Can we automatically ensure the updated policy $\pi'$ has value greater than (or equal to) the prior policy $V^{\pi'} \geq V^\pi$?
        \item Consider this for the policy gradient setting, and hope to address this by modifying step size
    \end{itemize}

\end{frame}
%-----------------------------------------------------------------------
%----------------------------------------------------------------------
\begin{frame}[c]{Objetive Function}
	
    \begin{itemize}
        \item Goal: find policy parameters that maximize value function
        $$ V(\theta) = \mathbb{E}_{\pi_\theta} \left[ \sum_{t=0}^{\infty} \gamma^t r(s_t, a_t); \pi_\theta \right]$$
        \item have access to samples from the current policy $\pi_\theta$ 
        \otem Want to predict the value of a different policy ($\leadsto$ off-policy learning)
    \end{itemize}

\end{frame}
%-----------------------------------------------------------------------
%----------------------------------------------------------------------
\begin{frame}[c]{Objective Function}
	
    \begin{itemize}
        \item Goal: find policy parameters that maximize value function
        $$ V(\theta) = \mathbb{E}_{\pi_\theta} \left[ \sum_{t=0}^{\infty} \gamma^t r(s_t, a_t); \pi_\theta \right]$$
        \item Express value of $\Tilde{\pi}$ in terms of advantage of $\pi$
        \begin{eqnarray}
        V(\Tilde{\theta}) &=& V(\theta) + \mathbb{E}_{\pi{\Tilde{\theta}}} \left[ \sum_{t=0}^{\infty} \gamma^t A_\pi(s_t, a_t) \right]\\
        &=& V(\theta) + \sum_{s} \mu_{\Tilde{\pi}}(s) \sum_{a} \Tilde{\pi}(a \mid s) A_\pi(s_t, a_t) 
        \end{eqnarray}
        \item $ \mu_{\Tilde{\pi}}(s)$ is the discounted weighted frequency of state $s$ under policy $\Tilde{\pi}$
    \end{itemize}

\end{frame}
%-----------------------------------------------------------------------
%----------------------------------------------------------------------
\begin{frame}[c]{TODO}
	
    TODO	

\end{frame}
%-----------------------------------------------------------------------
%----------------------------------------------------------------------
\begin{frame}[c]{TODO}
	
    TODO	

\end{frame}
%-----------------------------------------------------------------------
%----------------------------------------------------------------------
\begin{frame}[c]{TODO}
	
    TODO	

\end{frame}
%-----------------------------------------------------------------------

%-----------------------------------------------------------------------
\end{document}
