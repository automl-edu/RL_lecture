% !TeX spellcheck = en_US

\documentclass[aspectratio=169]{tntbeamer}  % you can pass all options of the beamer class, e.g., 'handout' or 'aspectratio=43'
\usepackage{dsfont}
\usepackage{bm}
\usepackage[english]{babel}
\usepackage[T1]{fontenc}
%\usepackage[utf8]{inputenc}
\usepackage{graphicx}
\graphicspath{ {./figures/} }
\usepackage{algorithm}
\usepackage[ruled,vlined,algo2e,linesnumbered]{algorithm2e}
\usepackage{hyperref}
\usepackage{booktabs}
\usepackage{mathtools}

\usepackage{amsmath,amssymb}
\usepackage{latexsym}

\DeclareMathOperator*{\argmax}{arg\,max}
\DeclareMathOperator*{\argmin}{arg\,min}

\usepackage{pgfplots}
\pgfplotsset{compat=1.16}
\usepackage{tikz}
\usetikzlibrary{trees} 
\usetikzlibrary{shapes.geometric}
\usetikzlibrary{positioning,shapes,shadows,arrows,calc,mindmap}
\usetikzlibrary{positioning,fadings,through}
\usetikzlibrary{decorations.pathreplacing}
\usetikzlibrary{intersections}
\pgfdeclarelayer{background}
\pgfdeclarelayer{foreground}
\pgfsetlayers{background,main,foreground}
\tikzstyle{activity}=[rectangle, draw=black, rounded corners, text centered, text width=8em]
\tikzstyle{data}=[rectangle, draw=black, text centered, text width=8em]
\tikzstyle{myarrow}=[->, thick, draw=black]

% Define the layers to draw the diagram
\pgfdeclarelayer{background}
\pgfdeclarelayer{foreground}
\pgfsetlayers{background,main,foreground}

\input{./latex_main_old/macros}

% Requires XeLaTeX or LuaLaTeX
\usepackage{unicode-math}

\usepackage{fontspec}
%\setsansfont{Arial}
\setsansfont{RotisSansSerifStd}[ 
Path=./latex_main/fonts/,
Extension = .otf,
UprightFont = *-Regular,  % or *-Light
BoldFont = *-ExtraBold,  % or *-Bold
ItalicFont = *-Italic
]
\setmonofont{Cascadia Mono}[
Scale=0.8
]

% scale factor adapted; mathrm font added (Benjamin Spitschan @TNT, 2021-06-01)
%\setmathfont[Scale=1.05]{Libertinus Math}
%\setmathrm[Scale=1.05]{Libertinus Math}

% other available math fonts are (not exhaustive)
% Latin Modern Math
% XITS Math
% Libertinus Math
% Asana Math
% Fira Math
% TeX Gyre Pagella Math
% TeX Gyre Bonum Math
% TeX Gyre Schola Math
% TeX Gyre Termes Math

% Literature References
% #1 = Display Name
% #2 = Url (without \href)
\newcommand{\lit}[2]{\href{#2}{\footnotesize\color{black!60}[#1]}}

%%% Beamer Customization
%----------------------------------------------------------------------
% (Don't) Show sections in frame header. Options: 'sections', 'sections light', empty
\setbeamertemplate{headline}{empty}

% Add header logo for normal frames
\setheaderimage{
	% \includegraphics[height=\logoheight]{figures/TNT_darkv4.pdf}
	\includegraphics[height=\logoheight]{./latex_main/figures/luh_logo_rgb_0_80_155.pdf}
	% \includegraphics[height=\logoheight]{figures/logo_tntluh.pdf}
}

% Header logo for title page
\settitleheaderimage{
	% \includegraphics[height=\logoheight]{figures/TNT_darkv4.pdf}
	\includegraphics[height=\logoheight]{./latex_main/figures/luh_logo_rgb_0_80_155.pdf}
	% \includegraphics[height=\logoheight]{figures/logo_tntluh.pdf}
}

% Title page: tntdefault 
\setbeamertemplate{title page}[tntdefault]  % or luhstyle
% Add optional title image here
%\addtitlepageimagedefault{\includegraphics[width=0.65\textwidth]{figures/luh_default_presentation_title_image.jpg}}

% Title page: luhstyle
% \setbeamertemplate{title page}[luhstyle]
% % Add optional title image here
% \addtitlepageimage{\includegraphics[width=0.75\textwidth]{figures/luh_default_presentation_title_image.jpg}}

\author[Lindauer]{Marius Lindauer\\[1em]
	\includegraphics[height=\logoheight]{./latex_main/figures/luh_logo_rgb_0_80_155.pdf}\qquad
\includegraphics[height=\logoheight]{./latex_main/figures/TNT_darkv4}\qquad
\includegraphics[height=\logoheight]{./latex_main/figures/L3S.jpg}	}
\date{Winter Term 2021
}


%%% Custom Packages
%----------------------------------------------------------------------
% Create dummy content
\usepackage{blindtext}

% Adds a frame with the current page layout. Just call \layout inside of a frame.
\usepackage{layout}


\institute{Institut f\"ur Informationsverarbeitung}%\\ Leibniz Universit\"at Hannover}
%\title{TNT Beamer Template}
%\author{Suomynon A. Anonymous}
\date{}



\title[Meta-RL]{Meta Reinforcement Learning}
\subtitle{Self-paced and Teacher Guided Curriculum Learning\footnote{Based on a \href{https://lilianweng.github.io/lil-log/2020/01/29/curriculum-for-reinforcement-learning.html}{blog} by Lilian Weng}}


\begin{document}
	
	\maketitle

%----------------------------------------------------------------------
%----------------------------------------------------------------------
\begin{frame}[c]{Automatic Curriculum Generation = Self-Paced Learning}
	
	\begin{itemize}
		\item[$\leadsto$] In contrast to a hand-designed curriculum, the curriculum is constructed on the fly based on the needs of the policy learner.
	\end{itemize}
	
\end{frame}
%-----------------------------------------------------------------------
%----------------------------------------------------------------------
\begin{frame}[c]{Teacher-Guided Curriculum \litw{\href{https://arxiv.org/abs/1704.03003}{Graves et al. 2017}}}
	
	\begin{itemize}
		\item Model curriculum learning as an N-armed bandit problem
		\begin{itemize}
			\item Each arm is an env 
			\item Choose the arm that gives the most (meta-)reward
			\item non-stationary problem since policy changes over time and thus the reward distribution
		\end{itemize}
		\smallskip
		\item Two types of possible meta-reward signals
		\begin{itemize}
			\item Loss-driven progress tracks the learning progress
			\item Complex-driven progress tracks the model complexity in proportion to the models generalization performance\footnote{KL Divergence between posterior and prior distribution over network weights}
		\end{itemize}
	\end{itemize}
	
\end{frame}
%-----------------------------------------------------------------------
%----------------------------------------------------------------------
\begin{frame}[c]{Teacher-Guided Curriculum \litw{\href{https://arxiv.org/abs/1704.03003}{Graves et al. 2017}}}
	
	\begin{itemize}
		\item Model curriculum learning as an N-armed bandit problem
		\begin{itemize}
			\item Each arm is an env 
			\item Choose the arm that gives the most (meta-)reward
			\item non-stationary problem since policy changes over time and thus the reward distribution
		\end{itemize}
		\smallskip
		\pause
		\item Two types of possible meta-reward signals
		\begin{itemize}
			\item Loss-driven progress tracks the learning progress
			\item Complex-driven progress tracks the model complexity in proportion to the models generalization performance\footnote{KL Divergence between posterior and prior distribution over network weights}
		\end{itemize}
	\end{itemize}
	
\end{frame}
%-----------------------------------------------------------------------
%----------------------------------------------------------------------
\begin{frame}[c]{Teacher-Guided Curriculum \litw{\href{https://arxiv.org/abs/1704.03003}{Graves et al. 2017}}}
	
	\begin{itemize}
		\item Two types of learners:
		\begin{description}
			\item[Student] the target RL agent that returns its scores on a given task
			\begin{itemize}
				\item should learn fast
				\item should not forget solutions to previous tasks
			\end{itemize}
			\item[Teacher] the meta-agent that selects a task based on the student's scores
			\begin{itemize}
				\item State observations: list of scores on $N$ tasks
				\item Action: select a task
				\item Reward: Average score improvement across all tasks
			\end{itemize}
		\end{description}
	\end{itemize}

	$$r_t = \sum_{i=1}^N r_t^{(i)} - r_{t-1}^{(i)}$$	
	
	\pause
	\begin{itemize}
		\item Use $\epsilon$-greedy or Thompson sampling for solving the non-stationary teacher-problem
	\end{itemize}
	
\end{frame}
%-----------------------------------------------------------------------
%-----------------------------------------------------------------------
\end{document}
