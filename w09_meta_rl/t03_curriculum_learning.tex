% !TeX spellcheck = en_US

\documentclass[aspectratio=169]{tntbeamer}  % you can pass all options of the beamer class, e.g., 'handout' or 'aspectratio=43'
\usepackage{dsfont}
\usepackage{bm}
\usepackage[english]{babel}
\usepackage[T1]{fontenc}
%\usepackage[utf8]{inputenc}
\usepackage{graphicx}
\graphicspath{ {./figures/} }
\usepackage{algorithm}
\usepackage[ruled,vlined,algo2e,linesnumbered]{algorithm2e}
\usepackage{hyperref}
\usepackage{booktabs}
\usepackage{mathtools}

\usepackage{amsmath,amssymb}
\usepackage{latexsym}

\DeclareMathOperator*{\argmax}{arg\,max}
\DeclareMathOperator*{\argmin}{arg\,min}

\usepackage{pgfplots}
\pgfplotsset{compat=1.16}
\usepackage{tikz}
\usetikzlibrary{trees} 
\usetikzlibrary{shapes.geometric}
\usetikzlibrary{positioning,shapes,shadows,arrows,calc,mindmap}
\usetikzlibrary{positioning,fadings,through}
\usetikzlibrary{decorations.pathreplacing}
\usetikzlibrary{intersections}
\pgfdeclarelayer{background}
\pgfdeclarelayer{foreground}
\pgfsetlayers{background,main,foreground}
\tikzstyle{activity}=[rectangle, draw=black, rounded corners, text centered, text width=8em]
\tikzstyle{data}=[rectangle, draw=black, text centered, text width=8em]
\tikzstyle{myarrow}=[->, thick, draw=black]

% Define the layers to draw the diagram
\pgfdeclarelayer{background}
\pgfdeclarelayer{foreground}
\pgfsetlayers{background,main,foreground}

\input{./latex_main_old/macros}

% Requires XeLaTeX or LuaLaTeX
\usepackage{unicode-math}

\usepackage{fontspec}
%\setsansfont{Arial}
\setsansfont{RotisSansSerifStd}[ 
Path=./latex_main/fonts/,
Extension = .otf,
UprightFont = *-Regular,  % or *-Light
BoldFont = *-ExtraBold,  % or *-Bold
ItalicFont = *-Italic
]
\setmonofont{Cascadia Mono}[
Scale=0.8
]

% scale factor adapted; mathrm font added (Benjamin Spitschan @TNT, 2021-06-01)
%\setmathfont[Scale=1.05]{Libertinus Math}
%\setmathrm[Scale=1.05]{Libertinus Math}

% other available math fonts are (not exhaustive)
% Latin Modern Math
% XITS Math
% Libertinus Math
% Asana Math
% Fira Math
% TeX Gyre Pagella Math
% TeX Gyre Bonum Math
% TeX Gyre Schola Math
% TeX Gyre Termes Math

% Literature References
% #1 = Display Name
% #2 = Url (without \href)
\newcommand{\lit}[2]{\href{#2}{\footnotesize\color{black!60}[#1]}}

%%% Beamer Customization
%----------------------------------------------------------------------
% (Don't) Show sections in frame header. Options: 'sections', 'sections light', empty
\setbeamertemplate{headline}{empty}

% Add header logo for normal frames
\setheaderimage{
	% \includegraphics[height=\logoheight]{figures/TNT_darkv4.pdf}
	\includegraphics[height=\logoheight]{./latex_main/figures/luh_logo_rgb_0_80_155.pdf}
	% \includegraphics[height=\logoheight]{figures/logo_tntluh.pdf}
}

% Header logo for title page
\settitleheaderimage{
	% \includegraphics[height=\logoheight]{figures/TNT_darkv4.pdf}
	\includegraphics[height=\logoheight]{./latex_main/figures/luh_logo_rgb_0_80_155.pdf}
	% \includegraphics[height=\logoheight]{figures/logo_tntluh.pdf}
}

% Title page: tntdefault 
\setbeamertemplate{title page}[tntdefault]  % or luhstyle
% Add optional title image here
%\addtitlepageimagedefault{\includegraphics[width=0.65\textwidth]{figures/luh_default_presentation_title_image.jpg}}

% Title page: luhstyle
% \setbeamertemplate{title page}[luhstyle]
% % Add optional title image here
% \addtitlepageimage{\includegraphics[width=0.75\textwidth]{figures/luh_default_presentation_title_image.jpg}}

\author[Lindauer]{Marius Lindauer\\[1em]
	\includegraphics[height=\logoheight]{./latex_main/figures/luh_logo_rgb_0_80_155.pdf}\qquad
\includegraphics[height=\logoheight]{./latex_main/figures/TNT_darkv4}\qquad
\includegraphics[height=\logoheight]{./latex_main/figures/L3S.jpg}	}
\date{Winter Term 2021
}


%%% Custom Packages
%----------------------------------------------------------------------
% Create dummy content
\usepackage{blindtext}

% Adds a frame with the current page layout. Just call \layout inside of a frame.
\usepackage{layout}


\institute{Institut f\"ur Informationsverarbeitung}%\\ Leibniz Universit\"at Hannover}
%\title{TNT Beamer Template}
%\author{Suomynon A. Anonymous}
\date{}



\title[Curriculum RL]{Curriculum Reinforcement Learning\footnote{Based on a \href{https://lilianweng.github.io/lil-log/2020/01/29/curriculum-for-reinforcement-learning.html}{blog} by Lilian Weng}}


\begin{document}
	
	\maketitle

%----------------------------------------------------------------------
%----------------------------------------------------------------------
\begin{frame}[c]{From Easy to Hard}


\begin{itemize}
	\item We humans also learn step by step
	\begin{itemize}
		\item E.g., in math, we learn first basic arithmetics before we later learn complex derivatives and integrals
		\item E.g., in this lecture, we first talked about simple planning on MDPs, before we talked about complex meta-RL ideas
	\end{itemize}
	\smallskip
	\item \alert{Idea:} break down complex concepts into simpler concepts s.t. we can start from the easy ones and build up the complex one
	\item \alert{Challenge}: How can we design a curriculum starting from simple to hard tasks?
	\begin{itemize}
		\item A poorly designed curriculum might even harm learning.
	\end{itemize}
	\item \alert{Challenge}: How do we avoid catastrophic forgetting by training on another instance?
\end{itemize}


\end{frame}
%-----------------------------------------------------------------------
%----------------------------------------------------------------------
\begin{frame}[c]{Task-Specific Curriculum Learning~\litw{\href{https://dl.acm.org/doi/10.1145/1553374.1553380}{Bengio et al. 2009}}}
	
	
	\begin{enumerate}
		\item Cleaner Examples may yield better generalization faster.
		\item Introducing gradually more difficult examples speeds up online training.
	\end{enumerate}

	\begin{itemize}
		\item Results by \lit{\href{https://arxiv.org/abs/1410.4615}{Zaremba and Sutskever. 2014}} indicated that one should mix in easy tasks to not forget how to solve these.
	\end{itemize}

	
\end{frame}
%-----------------------------------------------------------------------
%----------------------------------------------------------------------
\begin{frame}[c]{How to Quantify Complexity/Difficulty of an Env?}
	
	\begin{itemize}
		\item Human specification
		\begin{itemize}
			\item parameterized generator for environments that allow to control the complexity by hand
			\item or we can measure meta-features of the different envs, e.g.
			\begin{itemize}
				\item Size of maze
				\item distance between start state and end state
				\item Fraction of floor space to walls
				\item ...
			\end{itemize}
			\item Note: We cannot quantify the complexity of the optimal policy for a given MDP, because we have to measure these before actually solving the MDP
		\end{itemize}
	\end{itemize}
	

\end{frame}
%-----------------------------------------------------------------------
%----------------------------------------------------------------------
\begin{frame}[c]{Automatic Curriculum Generation}
	
	\begin{itemize}
		\item In contrast to a hand-designed curriculum before training starts, the curriculum is generated on the fly for any instances in the domain
		\item Criteria for the ordering have to be provided or computed during runtime
		\item While criteria need to be computed during runtime, they do not need to be hand-designed
	\end{itemize}
	
\end{frame}

%-----------------------------------------------------------------------
%----------------------------------------------------------------------
\begin{frame}[c]{Example Criterion 1: Manual Difficulty Ratings}
	
	\begin{enumerate}
		\item Idea: define a function to rate env difficulty
		\item Example: POET~\lit{\href{https://arxiv.org/pdf/1901.01753.pdf}{Wang et al. 2019}}\footnote{Image source: \href{https://eng.uber.com/poet-open-ended-deep-learning/}{Uber AI}}
	\end{enumerate}
	\centering
	
	\includegraphics[scale=0.5]{images/poet.PNG}			
\end{frame}

%-----------------------------------------------------------------------
%----------------------------------------------------------------------
\begin{frame}[c]{Example Criterion 2: External Model}
	
	\begin{enumerate}
		\item Minimal loss wrt another policy being pretrained on other tasks as criterion
		\item Sort the instances s.t. they go from easy to hard wrt this pretrained agent~\lit{\href{https://arxiv.org/abs/1802.03796}{Weinshall et al. 2018}}
	\end{enumerate}
	
\end{frame}

%-----------------------------------------------------------------------
%-----------------------------------------------------------------------
\end{document}
