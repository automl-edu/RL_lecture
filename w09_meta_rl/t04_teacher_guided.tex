% !TeX spellcheck = en_US
\documentclass[aspectratio=169]{../latex_main/tntbeamer}  % you can pass all options of the beamer class, e.g., 'handout' or 'aspectratio=43'
\usepackage{dsfont}
\usepackage{bm}
\usepackage[english]{babel}
\usepackage[T1]{fontenc}
%\usepackage[utf8]{inputenc}
\usepackage{graphicx}
\graphicspath{ {./figures/} }
\usepackage{algorithm}
\usepackage[ruled,vlined,algo2e,linesnumbered]{algorithm2e}
\usepackage{hyperref}
\usepackage{booktabs}
\usepackage{mathtools}

\usepackage{amsmath,amssymb}
\usepackage{latexsym}

\DeclareMathOperator*{\argmax}{arg\,max}
\DeclareMathOperator*{\argmin}{arg\,min}

\usepackage{pgfplots}
\pgfplotsset{compat=1.16}
\usepackage{tikz}
\usetikzlibrary{trees} 
\usetikzlibrary{shapes.geometric}
\usetikzlibrary{positioning,shapes,shadows,arrows,calc,mindmap}
\usetikzlibrary{positioning,fadings,through}
\usetikzlibrary{decorations.pathreplacing}
\usetikzlibrary{intersections}
\pgfdeclarelayer{background}
\pgfdeclarelayer{foreground}
\pgfsetlayers{background,main,foreground}
\tikzstyle{activity}=[rectangle, draw=black, rounded corners, text centered, text width=8em]
\tikzstyle{data}=[rectangle, draw=black, text centered, text width=8em]
\tikzstyle{myarrow}=[->, thick, draw=black]

% Define the layers to draw the diagram
\pgfdeclarelayer{background}
\pgfdeclarelayer{foreground}
\pgfsetlayers{background,main,foreground}

\input{./latex_main_old/macros}

% Requires XeLaTeX or LuaLaTeX
\usepackage{unicode-math}

\usepackage{fontspec}
%\setsansfont{Arial}
\setsansfont{RotisSansSerifStd}[ 
Path=./latex_main/fonts/,
Extension = .otf,
UprightFont = *-Regular,  % or *-Light
BoldFont = *-ExtraBold,  % or *-Bold
ItalicFont = *-Italic
]
\setmonofont{Cascadia Mono}[
Scale=0.8
]

% scale factor adapted; mathrm font added (Benjamin Spitschan @TNT, 2021-06-01)
%\setmathfont[Scale=1.05]{Libertinus Math}
%\setmathrm[Scale=1.05]{Libertinus Math}

% other available math fonts are (not exhaustive)
% Latin Modern Math
% XITS Math
% Libertinus Math
% Asana Math
% Fira Math
% TeX Gyre Pagella Math
% TeX Gyre Bonum Math
% TeX Gyre Schola Math
% TeX Gyre Termes Math

% Literature References
% #1 = Display Name
% #2 = Url (without \href)
\newcommand{\lit}[2]{\href{#2}{\footnotesize\color{black!60}[#1]}}

%%% Beamer Customization
%----------------------------------------------------------------------
% (Don't) Show sections in frame header. Options: 'sections', 'sections light', empty
\setbeamertemplate{headline}{empty}

% Add header logo for normal frames
\setheaderimage{
	% \includegraphics[height=\logoheight]{figures/TNT_darkv4.pdf}
	\includegraphics[height=\logoheight]{./latex_main/figures/luh_logo_rgb_0_80_155.pdf}
	% \includegraphics[height=\logoheight]{figures/logo_tntluh.pdf}
}

% Header logo for title page
\settitleheaderimage{
	% \includegraphics[height=\logoheight]{figures/TNT_darkv4.pdf}
	\includegraphics[height=\logoheight]{./latex_main/figures/luh_logo_rgb_0_80_155.pdf}
	% \includegraphics[height=\logoheight]{figures/logo_tntluh.pdf}
}

% Title page: tntdefault 
\setbeamertemplate{title page}[tntdefault]  % or luhstyle
% Add optional title image here
%\addtitlepageimagedefault{\includegraphics[width=0.65\textwidth]{figures/luh_default_presentation_title_image.jpg}}

% Title page: luhstyle
% \setbeamertemplate{title page}[luhstyle]
% % Add optional title image here
% \addtitlepageimage{\includegraphics[width=0.75\textwidth]{figures/luh_default_presentation_title_image.jpg}}

\author[Lindauer]{Marius Lindauer\\[1em]
	\includegraphics[height=\logoheight]{./latex_main/figures/luh_logo_rgb_0_80_155.pdf}\qquad
\includegraphics[height=\logoheight]{./latex_main/figures/TNT_darkv4}\qquad
\includegraphics[height=\logoheight]{./latex_main/figures/L3S.jpg}	}
\date{Winter Term 2021
}


%%% Custom Packages
%----------------------------------------------------------------------
% Create dummy content
\usepackage{blindtext}

% Adds a frame with the current page layout. Just call \layout inside of a frame.
\usepackage{layout}

\title[Curriculum RL]{Curriculum Learning}
\subtitle{Teacher-Guided Curriculum\footnote{Based on a \href{https://lilianweng.github.io/lil-log/2020/01/29/curriculum-for-reinforcement-learning.html}{blog} by Lilian Weng}}


\begin{document}
	
	\maketitle

%----------------------------------------------------------------------
%----------------------------------------------------------------------
\begin{frame}[c]{Teacher-Guided Curriculum}
	
	\begin{itemize}
		\item Idea: Expert teacher can use its own knowledge to create a curriculum
		\item Question: who or what is an expert teacher?
		\pause
		\item Possible answers: "Common sense" methods, real-life expert decisions, a learner learning how to construct a curriculum, ?
	\end{itemize}
	
\end{frame}
%-----------------------------------------------------------------------
%-----------------------------------------------------------------------

\begin{frame}[c]{The Teacher-Student Setup \lit{Matiisen et al., 2017}{https://arxiv.org/pdf/1707.00183.pdf}}
	
	\begin{figure}
	\centering
	\includegraphics[scale= 0.5]{images/teacher_student.png}
	\caption{Teacher-Student interaction}
	\end{figure}
	
\end{frame}
%-----------------------------------------------------------------------
%-----------------------------------------------------------------------

\begin{frame}[c]{The Teacher}
	\begin{itemize}
		\item The teacher observes the student's reward $x_t$
		\item The action space consists of all instances $i$
		\item The teacher's reward is the change in the agent's reward:
			$$ r_t = x_t - x_{t-1}$$
	\end{itemize}
	
\end{frame}
%-----------------------------------------------------------------------
%-----------------------------------------------------------------------

\begin{frame}[c]{The Ideal Result \lit{Matiisen et al., 2017}{https://arxiv.org/pdf/1707.00183.pdf}
	
	\begin{figure}
	\centering
	\includegraphics[scale= 0.5]{images/ideal_teacher.png}
	\caption{Teacher-Student interaction}
	\end{figure}

\end{frame}

%-----------------------------------------------------------------------
%-----------------------------------------------------------------------

\begin{frame}[c]{Variations on the idea}

	\begin{itemize}
		\item The same method can be applied to continuously parameterized environments \lit{Portelas et al., 2019}{https://arxiv.org/pdf/1910.07224.pdf}
		\item The teacher can be a fail-safe in safety-critical applications \lit{Turchetta et al., 2020}{https://papers.nips.cc/paper/2020/file/8df6a65941e4c9da40a4fb899de65c55-Paper.pdf}
		\item Guided Policy Search \lit{Levine \& Koltun, 2013}{https://graphics.stanford.edu/projects/gpspaper/gps_full.pdf} uses an expert policy to sample trajectories (not necessarily across instances)
	\end{itemize}

\end{frame}

\end{document}
