% !TeX spellcheck = en_US
\documentclass[aspectratio=169]{../latex_main/tntbeamer}  % you can pass all options of the beamer class, e.g., 'handout' or 'aspectratio=43'
\usepackage{dsfont}
\usepackage{bm}
\usepackage[english]{babel}
\usepackage[T1]{fontenc}
%\usepackage[utf8]{inputenc}
\usepackage{graphicx}
\graphicspath{ {./figures/} }
\usepackage{algorithm}
\usepackage[ruled,vlined,algo2e,linesnumbered]{algorithm2e}
\usepackage{hyperref}
\usepackage{booktabs}
\usepackage{mathtools}

\usepackage{amsmath,amssymb}
\usepackage{latexsym}

\DeclareMathOperator*{\argmax}{arg\,max}
\DeclareMathOperator*{\argmin}{arg\,min}

\usepackage{pgfplots}
\pgfplotsset{compat=1.16}
\usepackage{tikz}
\usetikzlibrary{trees} 
\usetikzlibrary{shapes.geometric}
\usetikzlibrary{positioning,shapes,shadows,arrows,calc,mindmap}
\usetikzlibrary{positioning,fadings,through}
\usetikzlibrary{decorations.pathreplacing}
\usetikzlibrary{intersections}
\pgfdeclarelayer{background}
\pgfdeclarelayer{foreground}
\pgfsetlayers{background,main,foreground}
\tikzstyle{activity}=[rectangle, draw=black, rounded corners, text centered, text width=8em]
\tikzstyle{data}=[rectangle, draw=black, text centered, text width=8em]
\tikzstyle{myarrow}=[->, thick, draw=black]

% Define the layers to draw the diagram
\pgfdeclarelayer{background}
\pgfdeclarelayer{foreground}
\pgfsetlayers{background,main,foreground}

\input{./latex_main_old/macros}

% Requires XeLaTeX or LuaLaTeX
\usepackage{unicode-math}

\usepackage{fontspec}
%\setsansfont{Arial}
\setsansfont{RotisSansSerifStd}[ 
Path=./latex_main/fonts/,
Extension = .otf,
UprightFont = *-Regular,  % or *-Light
BoldFont = *-ExtraBold,  % or *-Bold
ItalicFont = *-Italic
]
\setmonofont{Cascadia Mono}[
Scale=0.8
]

% scale factor adapted; mathrm font added (Benjamin Spitschan @TNT, 2021-06-01)
%\setmathfont[Scale=1.05]{Libertinus Math}
%\setmathrm[Scale=1.05]{Libertinus Math}

% other available math fonts are (not exhaustive)
% Latin Modern Math
% XITS Math
% Libertinus Math
% Asana Math
% Fira Math
% TeX Gyre Pagella Math
% TeX Gyre Bonum Math
% TeX Gyre Schola Math
% TeX Gyre Termes Math

% Literature References
% #1 = Display Name
% #2 = Url (without \href)
\newcommand{\lit}[2]{\href{#2}{\footnotesize\color{black!60}[#1]}}

%%% Beamer Customization
%----------------------------------------------------------------------
% (Don't) Show sections in frame header. Options: 'sections', 'sections light', empty
\setbeamertemplate{headline}{empty}

% Add header logo for normal frames
\setheaderimage{
	% \includegraphics[height=\logoheight]{figures/TNT_darkv4.pdf}
	\includegraphics[height=\logoheight]{./latex_main/figures/luh_logo_rgb_0_80_155.pdf}
	% \includegraphics[height=\logoheight]{figures/logo_tntluh.pdf}
}

% Header logo for title page
\settitleheaderimage{
	% \includegraphics[height=\logoheight]{figures/TNT_darkv4.pdf}
	\includegraphics[height=\logoheight]{./latex_main/figures/luh_logo_rgb_0_80_155.pdf}
	% \includegraphics[height=\logoheight]{figures/logo_tntluh.pdf}
}

% Title page: tntdefault 
\setbeamertemplate{title page}[tntdefault]  % or luhstyle
% Add optional title image here
%\addtitlepageimagedefault{\includegraphics[width=0.65\textwidth]{figures/luh_default_presentation_title_image.jpg}}

% Title page: luhstyle
% \setbeamertemplate{title page}[luhstyle]
% % Add optional title image here
% \addtitlepageimage{\includegraphics[width=0.75\textwidth]{figures/luh_default_presentation_title_image.jpg}}

\author[Lindauer]{Marius Lindauer\\[1em]
	\includegraphics[height=\logoheight]{./latex_main/figures/luh_logo_rgb_0_80_155.pdf}\qquad
\includegraphics[height=\logoheight]{./latex_main/figures/TNT_darkv4}\qquad
\includegraphics[height=\logoheight]{./latex_main/figures/L3S.jpg}	}
\date{Winter Term 2021
}


%%% Custom Packages
%----------------------------------------------------------------------
% Create dummy content
\usepackage{blindtext}

% Adds a frame with the current page layout. Just call \layout inside of a frame.
\usepackage{layout}

\title[Meta-RL]{Meta Reinforcement Learning}
\subtitle{Challenges in Contextual RL}
\usepackage{todonotes}

\begin{document}
	
	\maketitle

%----------------------------------------------------------------------
%----------------------------------------------------------------------

\todo[inline]{Too soon -- we need to talk about how cRL helps us formalize the Meta-Learning problem better}
\todo[inline]{Talk about contextual policies and transfer learning}
\todo[inline]{Talk about formalizing Meta-Learning as a learning context variable across instances}
\todo[inline]{What are uses of this context variable -- What kind of instances exist out there}
\todo[inline]{What are uses of this context variable -- What kind of instances exist out there}



\begin{frame}[c]{Why do we need separate methods in cRL?}

\begin{itemize}
	\item Some instances are already very hard on their own
	\item Learning a policy across a task distribution is not trivial either
	\item Conventional policy updates aren't necessarily good for every instance in $\mathcal{I}$
	\item Having access to the context can help performance, but is not always guaranteed
	\item Optimal hyperparameter might vary depending on the current instance
\end{itemize}


\end{frame}
%-----------------------------------------------------------------------
%----------------------------------------------------------------------
\begin{frame}[c]{Learning a Task Distribution is Hard}
	
\begin{figure}
    \centering
    \includegraphics[width=0.75\textwidth]{./images/CARLPendulumEnv_mean_ep_rew_over_step_hidden.png}
    \caption{Context variations on CARLPendulum. For most context features, difficulty increases with variation.}
    \label{fig:my_label}
\end{figure}
	
\end{frame}
%-----------------------------------------------------------------------
%----------------------------------------------------------------------
\begin{frame}[c]{Uneven Learning}
	
\begin{figure}
    \centering
    \includegraphics[width=0.7\textwidth]{./images/ez_40_5_comb.png}
    \caption{Performance per Instance over time. Instances are ordered according to performance at the start of training (left), but performance shifts dramatically to only a few hard instances (right).}
    \label{fig:my_label}
\end{figure}
	
\end{frame}
%-----------------------------------------------------------------------
%----------------------------------------------------------------------
\begin{frame}[c]{Context Helps Learning~\lit{Benjamins et al. 2021}{https://arxiv.org/pdf/2110.02102.pdf}}

\begin{figure}
    \centering
    \includegraphics[width=0.75\textwidth]{./images/CARLPendulumEnv_mean_ep_rew_over_step_visibledtglmmax_speed.png}
    
    \label{fig:my_label}
\end{figure}
\centering Contextual RL in CARL

\end{frame}
%-----------------------------------------------------------------------
%----------------------------------------------------------------------

\begin{frame}[c]{Instability through Hyperparameters~\lit{Eimer et al. 2021}{http://www.tnt.uni-hannover.de/papers/data/1540/CARL_HPs_2021.pdf}}

\begin{figure}
    \centering
    \includegraphics[width=0.6\textwidth]{./images/pb2workers_CARLPendulumEnv_hidden_vs_visible.png}
    \label{fig:my_label}
\end{figure}
\centering Hyperparamter Tuning on CARLLunarLander. The best configuration for visible context outperforms hidden context in evaluation, but the instability is greatly increased.

\end{frame}

\begin{frame}[c]{Key Approaches in Meta-RL}
	
	\begin{itemize}
		\item Meta-Learning Hyperparameters
		\begin{itemize}
			\item Hyperparameter values for good performance
			\item Alternative optimization methods
		\end{itemize}
		\item Meta-Learning the Training Dynamics
		\begin{itemize}
			\item Credit Assignment for state-action pairs
			\item Problem-specific Exploration Strategies
		\end{itemize}
		\item Meta-Learning Policies
		\begin{itemize}
		    \item Multi-Task learning and Task Transfer
			\item Smarter Updates across tasks
			\item Curriculum Learning
		\end{itemize}
		\item Meta-Learning Representations
	\end{itemize}
	
	
\end{frame}
%-----------------------------------------------------------------------
%-----------------------------------------------------------------------
\end{document}
