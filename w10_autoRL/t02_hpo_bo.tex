% !TeX spellcheck = en_US
\documentclass[aspectratio=169]{../latex_main/tntbeamer}  % you can pass all options of the beamer class, e.g., 'handout' or 'aspectratio=43'
\usepackage{dsfont}
\usepackage{bm}
\usepackage[english]{babel}
\usepackage[T1]{fontenc}
%\usepackage[utf8]{inputenc}
\usepackage{graphicx}
\graphicspath{ {./figures/} }
\usepackage{algorithm}
\usepackage[ruled,vlined,algo2e,linesnumbered]{algorithm2e}
\usepackage{hyperref}
\usepackage{booktabs}
\usepackage{mathtools}

\usepackage{amsmath,amssymb}
\usepackage{latexsym}

\DeclareMathOperator*{\argmax}{arg\,max}
\DeclareMathOperator*{\argmin}{arg\,min}

\usepackage{pgfplots}
\pgfplotsset{compat=1.16}
\usepackage{tikz}
\usetikzlibrary{trees} 
\usetikzlibrary{shapes.geometric}
\usetikzlibrary{positioning,shapes,shadows,arrows,calc,mindmap}
\usetikzlibrary{positioning,fadings,through}
\usetikzlibrary{decorations.pathreplacing}
\usetikzlibrary{intersections}
\pgfdeclarelayer{background}
\pgfdeclarelayer{foreground}
\pgfsetlayers{background,main,foreground}
\tikzstyle{activity}=[rectangle, draw=black, rounded corners, text centered, text width=8em]
\tikzstyle{data}=[rectangle, draw=black, text centered, text width=8em]
\tikzstyle{myarrow}=[->, thick, draw=black]

% Define the layers to draw the diagram
\pgfdeclarelayer{background}
\pgfdeclarelayer{foreground}
\pgfsetlayers{background,main,foreground}

\input{./latex_main_old/macros}

% Requires XeLaTeX or LuaLaTeX
\usepackage{unicode-math}

\usepackage{fontspec}
%\setsansfont{Arial}
\setsansfont{RotisSansSerifStd}[ 
Path=./latex_main/fonts/,
Extension = .otf,
UprightFont = *-Regular,  % or *-Light
BoldFont = *-ExtraBold,  % or *-Bold
ItalicFont = *-Italic
]
\setmonofont{Cascadia Mono}[
Scale=0.8
]

% scale factor adapted; mathrm font added (Benjamin Spitschan @TNT, 2021-06-01)
%\setmathfont[Scale=1.05]{Libertinus Math}
%\setmathrm[Scale=1.05]{Libertinus Math}

% other available math fonts are (not exhaustive)
% Latin Modern Math
% XITS Math
% Libertinus Math
% Asana Math
% Fira Math
% TeX Gyre Pagella Math
% TeX Gyre Bonum Math
% TeX Gyre Schola Math
% TeX Gyre Termes Math

% Literature References
% #1 = Display Name
% #2 = Url (without \href)
\newcommand{\lit}[2]{\href{#2}{\footnotesize\color{black!60}[#1]}}

%%% Beamer Customization
%----------------------------------------------------------------------
% (Don't) Show sections in frame header. Options: 'sections', 'sections light', empty
\setbeamertemplate{headline}{empty}

% Add header logo for normal frames
\setheaderimage{
	% \includegraphics[height=\logoheight]{figures/TNT_darkv4.pdf}
	\includegraphics[height=\logoheight]{./latex_main/figures/luh_logo_rgb_0_80_155.pdf}
	% \includegraphics[height=\logoheight]{figures/logo_tntluh.pdf}
}

% Header logo for title page
\settitleheaderimage{
	% \includegraphics[height=\logoheight]{figures/TNT_darkv4.pdf}
	\includegraphics[height=\logoheight]{./latex_main/figures/luh_logo_rgb_0_80_155.pdf}
	% \includegraphics[height=\logoheight]{figures/logo_tntluh.pdf}
}

% Title page: tntdefault 
\setbeamertemplate{title page}[tntdefault]  % or luhstyle
% Add optional title image here
%\addtitlepageimagedefault{\includegraphics[width=0.65\textwidth]{figures/luh_default_presentation_title_image.jpg}}

% Title page: luhstyle
% \setbeamertemplate{title page}[luhstyle]
% % Add optional title image here
% \addtitlepageimage{\includegraphics[width=0.75\textwidth]{figures/luh_default_presentation_title_image.jpg}}

\author[Lindauer]{Marius Lindauer\\[1em]
	\includegraphics[height=\logoheight]{./latex_main/figures/luh_logo_rgb_0_80_155.pdf}\qquad
\includegraphics[height=\logoheight]{./latex_main/figures/TNT_darkv4}\qquad
\includegraphics[height=\logoheight]{./latex_main/figures/L3S.jpg}	}
\date{Winter Term 2021
}


%%% Custom Packages
%----------------------------------------------------------------------
% Create dummy content
\usepackage{blindtext}

% Adds a frame with the current page layout. Just call \layout inside of a frame.
\usepackage{layout}

\title[AutoRL]{AutoRL}
\subtitle{Bayesian Optimization}


\begin{document}
	
	\maketitle

%-----------------------------------------------------------------------


%-----------------------------------------------------------------------
\begin{frame}[c]{Global Blackbox Optimization}
    
    \begin{itemize}
        \item Consider the \alert{global optimization problem} of finding: 
        \[\conf^* \in \argmin_{\conf \in \confs} f(\conf)\]
        
\pause
        \item In the most general form, function $f$ is a \alert{blackbox function}:
    	
    	\begin{center}
\scalebox{0.5}{\input{images/intro_images/blackbox_HPO.tex}}
    	\end{center}

   %\end{itemize}
    
   \begin{itemize}
    	\item Only mode of interaction with $f$: querying $f$'s value at a given $\conf$ 
        \item Function $f$ may not be available in closed form, not convex, not differentiable, noisy, etc. 
   \end{itemize}
    	%}   
\medskip
\pause

        \item Let's discuss a \alert{Bayesian} approach for solving such blackbox optimization problems
\medskip
\pause
        \item Blackbox optimization can be used for hyperparameter optimization (HPO)
    %\end{itemize}
   	 	\begin{itemize}
         	\item Define \alert{$f(\conf) := \mathcal{L}( \mathcal{A}_{\conf}, \mathcal{D}_{train}, \mathcal{D}_{valid} )$}
        	\item Note: for formulations of HPO that go beyond blackbox optimization, see next lecture
        \end{itemize}
    \end{itemize}
\end{frame}

%----------------------------------------------------------------------
\begin{frame}[c]{Bayesian Optimization of a Blackbox Function in a Nutshell}

\bigskip
\bigskip
\bigskip

    \onslide<1->
    \begin{figure}
        \vspace{-1em}
        \centering
        \only<1>{
            \includegraphics[width=0.95\textwidth]{images/intro_images/IntroPlots_Obs.pdf}
        }\only<2>{
            \includegraphics[width=0.95\textwidth]{images/intro_images/IntroPlots_GP.pdf}
        }\only<3>{
            \includegraphics[width=0.95\textwidth]{images/intro_images/IntroPlots_Acqui.pdf}
        }\only<4->{
            \includegraphics[width=0.95\textwidth]{images/intro_images/IntroPlots_Complete.pdf}
        }
    \end{figure}
    
\end{frame}
%----------------------------------------------------------------------
\begin{frame}[c]{Acquisition Function}

\begin{itemize}
    \item Goal to trade-off exploration and exploitation
    \begin{itemize}
        \item \alert{Exploration:} Sample in unexplored areas with high uncertainty to check\\ whether there are good samples
        \item \alert{Exploitation:} Sample in the areas where you believe the best sample should be
    \end{itemize}
    \smallskip
    \item We don't know the best exploration-exploitation trade-off
    \item[$\leadsto$] There are different possible acquisition function for this trade-off
    \smallskip
    \item For example, lower-confidence bounds (LCB)
\end{itemize}

\begin{equation}
    \alpha_{\text{UCB}}(\conf) = \mu(\conf) - \kappa \sigma(\conf)
\end{equation}

with $\mu$ being the posterior mean and $\sigma$ the posterior variance at point $\conf$

\end{frame}

%----------------------------------------------------------------------

%----------------------------------------------------------------------
\begin{frame}[c]{Bayesian Optimization of a Blackbox Function in a Nutshell}
\vspace{-1em}
\begin{columns}[T]
\column{0.45\textwidth}
General approach
\begin{itemize}
    \item Fit a \alert{probabilistic model} to the collected function samples $\langle{}\conf, \cost(\conf)\rangle{}$
    \item Use the model to guide optimization, trading off \alert{exploration \vs{} exploitation}
\end{itemize}


\onslide<4->{
    \alert{Popular approach in the statistics literature} since
    \lit{Mockus. 1975}{http://link.springer.com/chapter/10.1007\%2F3-540-07165-2_55}
    \begin{itemize}
        \item Efficient in \#function evaluations
        \item Works when objective is \alert{nonconvex, noisy, has unknown derivatives, etc.}
        \item \alert{Convergence} results\\ \lit{Srinivas et al. 2009}{https://arxiv.org/pdf/0912.3995.pdf}; \lit{Bull. 2011}{http://www.jmlr.org/papers/v12/bull11a.html} \lit{de Freitas et al. 2012}{https://www.cs.ubc.ca/~nando/papers/BayesBandits.pdf}; \lit{Kawaguchi et al. 2015}{https://proceedings.neurips.cc/paper/2015/file/0ebcc77dc72360d0eb8e9504c78d38bd-Paper.pdf}
    \end{itemize}
}

\column{0.55\textwidth}
\onslide<1->
\begin{figure}
    \vspace{-1em}
    \centering
    \onslide<1->{\includegraphics[width=0.8\textwidth]{images/intro_images/IntroPlots_Iter2.pdf}}
    \onslide<2->{\includegraphics[width=0.8\textwidth]{images/intro_images/IntroPlots_Iter3.pdf}}
    \onslide<3->{\includegraphics[width=0.8\textwidth]{images/intro_images/IntroPlots_Iter4.pdf}}
\end{figure}
\end{columns}

\end{frame}

\begin{frame}[c]{Bayesian Optimization: Pseudocode}

\begin{center}
\begin{minipage}{0.75\textwidth}
\begin{algorithm}[H]
    \setcounter{AlgoLine}{0}
    \SetKwInOut{Require}{Require}
    \SetKwInOut{Result}{Result}
    
    \Require{Search space $\pcs$, 
    		cost function $\cost$, 
    		acquisition function $\acq$, predictive model $\surro$,
    		maximal number of function evaluations $\bobudget$}
    \Result{Best configuration $\hat{\conf}$
    (according to $\dataset$ or 
    $\surro$)}
    
	Initialize data $\iter[0]{\dataset}$ with initial observations\;% \leftarrow \varnothing$\; 
	 
    \For{$\bocount=1$ \KwTo $\bobudget$}{
		%\While{$B$ not exhausted} {
		Fit predictive model $\iter[\bocount]{\surro}$ on $\iter[\bocount-1]{\dataset}$\;
		
		Select next query point: $\bonextsample \in \argmax_{\conf \in \pcs} \acq(\conf; \iter[\bocount-1]{\dataset}, \iter[\bocount]{\surro})$\;
		
		Query $\bonextobs$\;
		
		Update data: $\iter[\bocount]{\dataset} \leftarrow \iter[\bocount-1]{\dataset} \cup \{\langle \bonextsample, \bonextobs \rangle \}$\;
	}
\end{algorithm}
\end{minipage}
\end{center}
\end{frame}
% %-----------------------------------------------------------------------
% \begin{frame}[c]{Bayesian Optimization: Origin of the Name}

% \begin{itemize}
%     \item Bayesian optimization uses \alert{Bayes' theorem}: 
%     	\begin{equation*}
%     	    P(A \vert B) = \frac{P(B \vert A) \times  P(A)}{P(B)}
%     	    \propto P(B \vert A) \times  P(A)
%     	\end{equation*} 
%     \item Bayesian optimization uses this to compute a posterior over functions: 
%         \begin{equation*}
%             P(\func \vert \dataset_{1:\bocount}) \propto P(\dataset_{1:\bocount} \vert \func) \times P(\func), \text{~~~~ where } \dataset_{1:\bocount} = \left \{ \conf_{1:\bocount}, \cost(\conf_{1:\bocount}) \right\}
%         \end{equation*} 
% \pause
% \vspace*{-0.5cm}
%     \item Meaning of the individual terms:
%         \begin{itemize}
%             \item $P(f)$ is the \alert{prior} over functions, which represents our belief about the space of possible objective functions \alert{before} we see any data
%             \item $\dataset_{1:\bocount}$ is the \alert{data} (or observations, evidence)
%             \item $P(\dataset_{1:\bocount} \vert \func)$ is the likelihood of the data given a function
%             \item $P(\func \vert \dataset_{1:\bocount})$ is the \alert{posterior} probability over functions given the data
%         \end{itemize}
%     \end{itemize}
% \end{frame}
% %-----------------------------------------------------------------------
\begin{frame}[c]{Bayesian Optimization: Advantages and Disadvantages}

\begin{columns}[T] % align columns
\begin{column}{.48\textwidth}


\begin{block}{Advantages}
\begin{itemize}
  \item Sample efficient 
  \item Can handle noise
  \item Native incorporation of priors 
  \item Does not require gradients 
  \item Theoretical guarantees
\end{itemize}
\end{block}

\end{column}%

\hfill%
\pause 
\begin{column}{.48\textwidth}

\begin{block}{Disadvantages}
\begin{itemize}
  \item Overhead because of model training in each iteration 
  \item Crucially relies on robust surrogate model
\end{itemize}
\end{block}

\end{column}
\end{columns}

\end{frame}



%-----------------------------------------------------------------------
\end{document}
