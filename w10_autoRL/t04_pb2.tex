% !TeX spellcheck = en_US
\documentclass[aspectratio=169]{../latex_main/tntbeamer}  % you can pass all options of the beamer class, e.g., 'handout' or 'aspectratio=43'
\usepackage{dsfont}
\usepackage{bm}
\usepackage[english]{babel}
\usepackage[T1]{fontenc}
%\usepackage[utf8]{inputenc}
\usepackage{graphicx}
\graphicspath{ {./figures/} }
\usepackage{algorithm}
\usepackage[ruled,vlined,algo2e,linesnumbered]{algorithm2e}
\usepackage{hyperref}
\usepackage{booktabs}
\usepackage{mathtools}

\usepackage{amsmath,amssymb}
\usepackage{latexsym}

\DeclareMathOperator*{\argmax}{arg\,max}
\DeclareMathOperator*{\argmin}{arg\,min}

\usepackage{pgfplots}
\pgfplotsset{compat=1.16}
\usepackage{tikz}
\usetikzlibrary{trees} 
\usetikzlibrary{shapes.geometric}
\usetikzlibrary{positioning,shapes,shadows,arrows,calc,mindmap}
\usetikzlibrary{positioning,fadings,through}
\usetikzlibrary{decorations.pathreplacing}
\usetikzlibrary{intersections}
\pgfdeclarelayer{background}
\pgfdeclarelayer{foreground}
\pgfsetlayers{background,main,foreground}
\tikzstyle{activity}=[rectangle, draw=black, rounded corners, text centered, text width=8em]
\tikzstyle{data}=[rectangle, draw=black, text centered, text width=8em]
\tikzstyle{myarrow}=[->, thick, draw=black]

% Define the layers to draw the diagram
\pgfdeclarelayer{background}
\pgfdeclarelayer{foreground}
\pgfsetlayers{background,main,foreground}

\input{./latex_main_old/macros}

% Requires XeLaTeX or LuaLaTeX
\usepackage{unicode-math}

\usepackage{fontspec}
%\setsansfont{Arial}
\setsansfont{RotisSansSerifStd}[ 
Path=./latex_main/fonts/,
Extension = .otf,
UprightFont = *-Regular,  % or *-Light
BoldFont = *-ExtraBold,  % or *-Bold
ItalicFont = *-Italic
]
\setmonofont{Cascadia Mono}[
Scale=0.8
]

% scale factor adapted; mathrm font added (Benjamin Spitschan @TNT, 2021-06-01)
%\setmathfont[Scale=1.05]{Libertinus Math}
%\setmathrm[Scale=1.05]{Libertinus Math}

% other available math fonts are (not exhaustive)
% Latin Modern Math
% XITS Math
% Libertinus Math
% Asana Math
% Fira Math
% TeX Gyre Pagella Math
% TeX Gyre Bonum Math
% TeX Gyre Schola Math
% TeX Gyre Termes Math

% Literature References
% #1 = Display Name
% #2 = Url (without \href)
\newcommand{\lit}[2]{\href{#2}{\footnotesize\color{black!60}[#1]}}

%%% Beamer Customization
%----------------------------------------------------------------------
% (Don't) Show sections in frame header. Options: 'sections', 'sections light', empty
\setbeamertemplate{headline}{empty}

% Add header logo for normal frames
\setheaderimage{
	% \includegraphics[height=\logoheight]{figures/TNT_darkv4.pdf}
	\includegraphics[height=\logoheight]{./latex_main/figures/luh_logo_rgb_0_80_155.pdf}
	% \includegraphics[height=\logoheight]{figures/logo_tntluh.pdf}
}

% Header logo for title page
\settitleheaderimage{
	% \includegraphics[height=\logoheight]{figures/TNT_darkv4.pdf}
	\includegraphics[height=\logoheight]{./latex_main/figures/luh_logo_rgb_0_80_155.pdf}
	% \includegraphics[height=\logoheight]{figures/logo_tntluh.pdf}
}

% Title page: tntdefault 
\setbeamertemplate{title page}[tntdefault]  % or luhstyle
% Add optional title image here
%\addtitlepageimagedefault{\includegraphics[width=0.65\textwidth]{figures/luh_default_presentation_title_image.jpg}}

% Title page: luhstyle
% \setbeamertemplate{title page}[luhstyle]
% % Add optional title image here
% \addtitlepageimage{\includegraphics[width=0.75\textwidth]{figures/luh_default_presentation_title_image.jpg}}

\author[Lindauer]{Marius Lindauer\\[1em]
	\includegraphics[height=\logoheight]{./latex_main/figures/luh_logo_rgb_0_80_155.pdf}\qquad
\includegraphics[height=\logoheight]{./latex_main/figures/TNT_darkv4}\qquad
\includegraphics[height=\logoheight]{./latex_main/figures/L3S.jpg}	}
\date{Winter Term 2021
}


%%% Custom Packages
%----------------------------------------------------------------------
% Create dummy content
\usepackage{blindtext}

% Adds a frame with the current page layout. Just call \layout inside of a frame.
\usepackage{layout}

\title[AutoRL]{AutoRL}
\subtitle{PBT Combined with Bayesian Optimization}


\begin{document}
	
	\maketitle


%----------------------------------------------------------------------
\begin{frame}[c]{Combining Population-based Training and Bayesian Optimization}

\begin{itemize}
	\item Bayesian Optimization (BO) is well known for its sample efficiency, but cannot change hyperparameters dynamically while training
        \item PBT can change hyperparameter dynamically, but is not very sample efficient.
	\pause
	\medskip
	\item \alert{Idea:} Can we use BO to guide PBT? 
	\pause
	\medskip
	\item[$\leadsto$] Less parallel compute resources are required (?)
	\pause
	\medskip
	\item[$\leadsto$] Scales better to higher dimensional spaces (?)
\end{itemize}

\end{frame}
%----------------------------------------------------------------------
%----------------------------------------------------------------------
\begin{frame}[c]{PBT + BO: Outline}

\begin{enumerate}
	\item Sample initial population
	\begin{itemize}
		\item Each population member is a combination of hyperparameter setting $\conf$ and (partially trained) model
	\end{itemize}
	\item Train population for a bit 
	\item Tournament selection to drop poorly performing population members
	\item Use \alert{Bayesian optimization} to select new hyperparameter settings
	\begin{itemize}
		\item Change the hyperparameter settings, but inherits the partially trained model (+ pertubation)
	\end{itemize}
	\item[$\leadsto$] New population consists of so-far best performing ones and new off-springs
	\item Go to 2.
\end{enumerate}


\end{frame}
%----------------------------------------------------------------------
%----------------------------------------------------------------------
\begin{frame}[c]{PBT  + BO = PB2 \lit{Parker-Holder et al. 2020}{https://arxiv.org/pdf/2002.02518.pdf}}

\begin{itemize}
	\item \alert{Challenge}: The cost depends on the previous $\confI{1}, \confI{2}. \ldots, \confI{t-1}$
	\pause
	\medskip
	\item BO-Surrogate model predicts the cost improvement over time:
\end{itemize}

\begin{equation}
\cost_{\text{PBT}}^{(t)}(\conf)= \frac{\cost^{(t)}(\conf) - \cost^{(t-1)}(\conf)}{\Delta t}\nonumber
\end{equation}

where $\cost^{(t)}(\conf)$ is the cost for a given hyperparameter setting at time step $t$.

\bigskip
\pause
\begin{itemize}
	\item Remark: Also add $\cost^{(t-1)}$ as an input to the BO-surrogate model to\\ ease the task of predicting the improvement
\end{itemize}

\end{frame}
%----------------------------------------------------------------------

%----------------------------------------------------------------------
\begin{frame}[c]{PB2: Advantages and Disadvantages}

\begin{columns}[T] % align columns
\begin{column}{.48\textwidth}


\begin{block}{Advantages}
\begin{itemize}
  \item Sample efficient ($\leadsto$ BO)
  \item Natively Parallel ($\leadsto$ PBT)
  \item Hyperparameters can be adjusted dynamically while training ($\leadsto$ PBT)
\end{itemize}
\end{block}

\end{column}%

\hfill%
\pause 
\begin{column}{.48\textwidth}

\begin{block}{Disadvantages}
\begin{itemize}
  \item (neglectable) overhead because of model training in each iteration 
  \item Crucially relies on robust surrogate model
  \item BO is used to solve a non-stationary problem and thus requires extrapolation capabilities
  \item A-priori unclear which hyperparameters need dynamical adaptation 
\end{itemize}
\end{block}

\end{column}
\end{columns}

\end{frame}
%----------------------------------------------------------------------

%-----------------------------------------------------------------------
\end{document}
