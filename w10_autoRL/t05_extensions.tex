% !TeX spellcheck = en_US
\documentclass[aspectratio=169]{../latex_main/tntbeamer}  % you can pass all options of the beamer class, e.g., 'handout' or 'aspectratio=43'
\usepackage{dsfont}
\usepackage{bm}
\usepackage[english]{babel}
\usepackage[T1]{fontenc}
%\usepackage[utf8]{inputenc}
\usepackage{graphicx}
\graphicspath{ {./figures/} }
\usepackage{algorithm}
\usepackage[ruled,vlined,algo2e,linesnumbered]{algorithm2e}
\usepackage{hyperref}
\usepackage{booktabs}
\usepackage{mathtools}

\usepackage{amsmath,amssymb}
\usepackage{latexsym}

\DeclareMathOperator*{\argmax}{arg\,max}
\DeclareMathOperator*{\argmin}{arg\,min}

\usepackage{pgfplots}
\pgfplotsset{compat=1.16}
\usepackage{tikz}
\usetikzlibrary{trees} 
\usetikzlibrary{shapes.geometric}
\usetikzlibrary{positioning,shapes,shadows,arrows,calc,mindmap}
\usetikzlibrary{positioning,fadings,through}
\usetikzlibrary{decorations.pathreplacing}
\usetikzlibrary{intersections}
\pgfdeclarelayer{background}
\pgfdeclarelayer{foreground}
\pgfsetlayers{background,main,foreground}
\tikzstyle{activity}=[rectangle, draw=black, rounded corners, text centered, text width=8em]
\tikzstyle{data}=[rectangle, draw=black, text centered, text width=8em]
\tikzstyle{myarrow}=[->, thick, draw=black]

% Define the layers to draw the diagram
\pgfdeclarelayer{background}
\pgfdeclarelayer{foreground}
\pgfsetlayers{background,main,foreground}

\input{./latex_main_old/macros}

% Requires XeLaTeX or LuaLaTeX
\usepackage{unicode-math}

\usepackage{fontspec}
%\setsansfont{Arial}
\setsansfont{RotisSansSerifStd}[ 
Path=./latex_main/fonts/,
Extension = .otf,
UprightFont = *-Regular,  % or *-Light
BoldFont = *-ExtraBold,  % or *-Bold
ItalicFont = *-Italic
]
\setmonofont{Cascadia Mono}[
Scale=0.8
]

% scale factor adapted; mathrm font added (Benjamin Spitschan @TNT, 2021-06-01)
%\setmathfont[Scale=1.05]{Libertinus Math}
%\setmathrm[Scale=1.05]{Libertinus Math}

% other available math fonts are (not exhaustive)
% Latin Modern Math
% XITS Math
% Libertinus Math
% Asana Math
% Fira Math
% TeX Gyre Pagella Math
% TeX Gyre Bonum Math
% TeX Gyre Schola Math
% TeX Gyre Termes Math

% Literature References
% #1 = Display Name
% #2 = Url (without \href)
\newcommand{\lit}[2]{\href{#2}{\footnotesize\color{black!60}[#1]}}

%%% Beamer Customization
%----------------------------------------------------------------------
% (Don't) Show sections in frame header. Options: 'sections', 'sections light', empty
\setbeamertemplate{headline}{empty}

% Add header logo for normal frames
\setheaderimage{
	% \includegraphics[height=\logoheight]{figures/TNT_darkv4.pdf}
	\includegraphics[height=\logoheight]{./latex_main/figures/luh_logo_rgb_0_80_155.pdf}
	% \includegraphics[height=\logoheight]{figures/logo_tntluh.pdf}
}

% Header logo for title page
\settitleheaderimage{
	% \includegraphics[height=\logoheight]{figures/TNT_darkv4.pdf}
	\includegraphics[height=\logoheight]{./latex_main/figures/luh_logo_rgb_0_80_155.pdf}
	% \includegraphics[height=\logoheight]{figures/logo_tntluh.pdf}
}

% Title page: tntdefault 
\setbeamertemplate{title page}[tntdefault]  % or luhstyle
% Add optional title image here
%\addtitlepageimagedefault{\includegraphics[width=0.65\textwidth]{figures/luh_default_presentation_title_image.jpg}}

% Title page: luhstyle
% \setbeamertemplate{title page}[luhstyle]
% % Add optional title image here
% \addtitlepageimage{\includegraphics[width=0.75\textwidth]{figures/luh_default_presentation_title_image.jpg}}

\author[Lindauer]{Marius Lindauer\\[1em]
	\includegraphics[height=\logoheight]{./latex_main/figures/luh_logo_rgb_0_80_155.pdf}\qquad
\includegraphics[height=\logoheight]{./latex_main/figures/TNT_darkv4}\qquad
\includegraphics[height=\logoheight]{./latex_main/figures/L3S.jpg}	}
\date{Winter Term 2021
}


%%% Custom Packages
%----------------------------------------------------------------------
% Create dummy content
\usepackage{blindtext}

% Adds a frame with the current page layout. Just call \layout inside of a frame.
\usepackage{layout}

\title[AutoRL]{AutoRL}
\subtitle{Increasing Efficiency of AutoRL}


\begin{document}
	
	\maketitle


%----------------------------------------------------------------------
\begin{frame}[c]{AutoRL tailored to RL}

\begin{itemize}
    \item Bayesian Optimization and PBT can be applied to all kinds of HPO problems
    \item RL has some special traits that, for example, supervised learning does not have
    \item How can we exploit these RL traits to have even more efficient AutoRL approaches?
\end{itemize}

\end{frame}
%----------------------------------------------------------------------
%----------------------------------------------------------------------
\begin{frame}[c]{Shared Replay Buffer}

\begin{columns}

\column{0.5\textwidth}

\centering
\includegraphics[width=0.65\textwidth]{images/searl.jpg}

\footnotesize
Source: \lit{Franke et al. 2021}{https://openreview.net/pdf?id=hSjxQ3B7GWq}

\column{0.5\textwidth}

\begin{itemize}
    \item Training several agents in parallel (e.g., PBT or PB2) implies that all of them acquire experience ($s_t, a_t, r_t, s_{t+1}$)
    \item Off-policy RL algorithms can make of these (e.g., DQN)
    \item Idea: Share replay buffer among all agents
    \begin{itemize}
        \item[$\leadsto$] Ensure that each agent plays at least an entire episode before making use of the shared replay buffer again
    \end{itemize}
\end{itemize}

\end{columns}


\end{frame}
%----------------------------------------------------------------------
%----------------------------------------------------------------------
\begin{frame}[c]{Growing Network Size}


\begin{itemize}
    \item Obviously, the DL architecture is important for training an agent\\ (e.g., see Table~1 in \lit{Stanic et al. 2022}{https://arxiv.org/pdf/2208.03374.pdf})
    \item Commonly, the network size (both policy and value networks) is fixed at the beginning of training
    \item However, it is unclear how complex a task is and thus the required complexity of the RL network
    \item Observation: At the beginning, RL agents learn simple tasks first
    \item Idea: Increase network complexity over time \lit{Franke et al. 2021}{https://openreview.net/pdf?id=hSjxQ3B7GWq}
    \begin{itemize}
        \item For example, Lamarckian operators allow growing a network without changing its predictions (e.g., widening layers or adding layers) \lit{Chen et al. 2015}{https://arxiv.org/abs/1511.05641}\lit{Elsken et al. 2019}{https://openreview.net/pdf?id=ByME42AqK7}
    \end{itemize}
    
\end{itemize}

\end{frame}
%----------------------------------------------------------------------
%----------------------------------------------------------------------
\begin{frame}[c]{Generalization to Similar Tasks}

\begin{columns}

\column{0.5\textwidth}

\centering
\includegraphics[width=0.65\textwidth]{images/cart-pole.png}

\column{0.5\textwidth}

\begin{itemize}
    \item Transferability across environments can be important~\lit{Zhang et al. 2021}{https://arxiv.org/abs/2102.13651} 
    \begin{itemize}
        \item After deploying RL agents can lead to slight variations of the environments (e.g., sim2real gap)
        \item It is not desirable to re-configure an RL agent for all minor environment changes
        \item Open question: When is re-configuration necessary depending on the environment change?
    \end{itemize}
    \item HPO is crucial for learning agents that can generalize to similar environments~\lit{Eimer et al. 2021}{https://www.tnt.uni-hannover.de/papers/data/1540/CARL_HPs_2021(1).pdf}
\end{itemize}

\end{columns}

\end{frame}
%----------------------------------------------------------------------
%----------------------------------------------------------------------
\begin{frame}[c]{Multi-fidelity Optimization for AutoRL}

\begin{columns}

\column{0.5\textwidth}

\centering
\includegraphics[width=0.8\textwidth]{images/mf_opt.jpg}

\column{0.5\textwidth}

\begin{itemize}
    \item Speeding up AutoML is possibly by making decisions about well-performing models after partial training (e.g., dataset subsets or training for a few epochs)
    \item Same idea can be applied to AutoRL; efficient decisions can be done on
    \begin{itemize}
        \item Evaluation on a few random seeds (both agent and environment)
        \item Partial training
        \item Evaluation (and training) on short episode lengths
        \item Evaluation on a few environment variations
    \end{itemize}
\end{itemize}

\end{columns}

\end{frame}
%----------------------------------------------------------------------
%----------------------------------------------------------------------
\begin{frame}[c]{Survey on AutoRL}

\begin{itemize}
    \item[$\leadsto$] \lit{Parker-Holder et al. 2022}{https://arxiv.org/abs/2201.03916} 
    \item Further ideas:
    \begin{itemize}
        \item Using meta-gradients
        \item Automated reward shaping
        \item Design of environments and curriculum learning
    \end{itemize}
    \item Holy grail: Learning to learn RL algorithms
\end{itemize}

\end{frame}
%----------------------------------------------------------------------

%-----------------------------------------------------------------------
\end{document}
