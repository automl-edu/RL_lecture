% !TeX spellcheck = en_US
\documentclass[aspectratio=169]{../latex_main/tntbeamer}  % you can pass all options of the beamer class, e.g., 'handout' or 'aspectratio=43'
\usepackage{dsfont}
\usepackage{bm}
\usepackage[english]{babel}
\usepackage[T1]{fontenc}
%\usepackage[utf8]{inputenc}
\usepackage{graphicx}
\graphicspath{ {./figures/} }
\usepackage{algorithm}
\usepackage[ruled,vlined,algo2e,linesnumbered]{algorithm2e}
\usepackage{hyperref}
\usepackage{booktabs}
\usepackage{mathtools}

\usepackage{amsmath,amssymb}
\usepackage{latexsym}

\DeclareMathOperator*{\argmax}{arg\,max}
\DeclareMathOperator*{\argmin}{arg\,min}

\usepackage{pgfplots}
\pgfplotsset{compat=1.16}
\usepackage{tikz}
\usetikzlibrary{trees} 
\usetikzlibrary{shapes.geometric}
\usetikzlibrary{positioning,shapes,shadows,arrows,calc,mindmap}
\usetikzlibrary{positioning,fadings,through}
\usetikzlibrary{decorations.pathreplacing}
\usetikzlibrary{intersections}
\pgfdeclarelayer{background}
\pgfdeclarelayer{foreground}
\pgfsetlayers{background,main,foreground}
\tikzstyle{activity}=[rectangle, draw=black, rounded corners, text centered, text width=8em]
\tikzstyle{data}=[rectangle, draw=black, text centered, text width=8em]
\tikzstyle{myarrow}=[->, thick, draw=black]

% Define the layers to draw the diagram
\pgfdeclarelayer{background}
\pgfdeclarelayer{foreground}
\pgfsetlayers{background,main,foreground}

\input{./latex_main_old/macros}

% Requires XeLaTeX or LuaLaTeX
\usepackage{unicode-math}

\usepackage{fontspec}
%\setsansfont{Arial}
\setsansfont{RotisSansSerifStd}[ 
Path=./latex_main/fonts/,
Extension = .otf,
UprightFont = *-Regular,  % or *-Light
BoldFont = *-ExtraBold,  % or *-Bold
ItalicFont = *-Italic
]
\setmonofont{Cascadia Mono}[
Scale=0.8
]

% scale factor adapted; mathrm font added (Benjamin Spitschan @TNT, 2021-06-01)
%\setmathfont[Scale=1.05]{Libertinus Math}
%\setmathrm[Scale=1.05]{Libertinus Math}

% other available math fonts are (not exhaustive)
% Latin Modern Math
% XITS Math
% Libertinus Math
% Asana Math
% Fira Math
% TeX Gyre Pagella Math
% TeX Gyre Bonum Math
% TeX Gyre Schola Math
% TeX Gyre Termes Math

% Literature References
% #1 = Display Name
% #2 = Url (without \href)
\newcommand{\lit}[2]{\href{#2}{\footnotesize\color{black!60}[#1]}}

%%% Beamer Customization
%----------------------------------------------------------------------
% (Don't) Show sections in frame header. Options: 'sections', 'sections light', empty
\setbeamertemplate{headline}{empty}

% Add header logo for normal frames
\setheaderimage{
	% \includegraphics[height=\logoheight]{figures/TNT_darkv4.pdf}
	\includegraphics[height=\logoheight]{./latex_main/figures/luh_logo_rgb_0_80_155.pdf}
	% \includegraphics[height=\logoheight]{figures/logo_tntluh.pdf}
}

% Header logo for title page
\settitleheaderimage{
	% \includegraphics[height=\logoheight]{figures/TNT_darkv4.pdf}
	\includegraphics[height=\logoheight]{./latex_main/figures/luh_logo_rgb_0_80_155.pdf}
	% \includegraphics[height=\logoheight]{figures/logo_tntluh.pdf}
}

% Title page: tntdefault 
\setbeamertemplate{title page}[tntdefault]  % or luhstyle
% Add optional title image here
%\addtitlepageimagedefault{\includegraphics[width=0.65\textwidth]{figures/luh_default_presentation_title_image.jpg}}

% Title page: luhstyle
% \setbeamertemplate{title page}[luhstyle]
% % Add optional title image here
% \addtitlepageimage{\includegraphics[width=0.75\textwidth]{figures/luh_default_presentation_title_image.jpg}}

\author[Lindauer]{Marius Lindauer\\[1em]
	\includegraphics[height=\logoheight]{./latex_main/figures/luh_logo_rgb_0_80_155.pdf}\qquad
\includegraphics[height=\logoheight]{./latex_main/figures/TNT_darkv4}\qquad
\includegraphics[height=\logoheight]{./latex_main/figures/L3S.jpg}	}
\date{Winter Term 2021
}


%%% Custom Packages
%----------------------------------------------------------------------
% Create dummy content
\usepackage{blindtext}

% Adds a frame with the current page layout. Just call \layout inside of a frame.
\usepackage{layout}

\title[AutoRL]{AutoRL}
\subtitle{Motivation}


\begin{document}
	
	\maketitle

%-----------------------------------------------------------------------


%----------------------------------------------------------------------
\begin{frame}[c]{Problem of Hyperparameters}
	
	\begin{itemize}
	    \item We have seen that many algorithms have important hyperparameters
	    \item Some hyperparameters are dealt with implicitly, as we have seen with the step size in TRPO
	    \begin{itemize}
	        \item but this introduced other hyperparameters; hopefully not so sensible
	    \end{itemize}
	    \item In fact, many RL algorithms are fairly brittle regarding their hyperparameter settings
	\end{itemize}
	
\end{frame}
%-----------------------------------------------------------------------
%----------------------------------------------------------------------
\begin{frame}[c]{Exemplary Hyperparameters in RL}

        \pause
	\begin{itemize}
	    \item Learning rate
	    \item Discount factor
	    \item Exploration rate / strategy
	    \item Reward shaping and reward proxies (e.g., subgoals)
	    \item DNN architecture and everything related to training DNNs
	    \item State preprocessing (e.g., images)
	    \item Experience reply
	    \item Choice of RL algorithm (DQN vs. REINFORCE vs. PPO vs. TRPO vs. ...)
	    \item \ldots
	\end{itemize}
	
\end{frame}
%-----------------------------------------------------------------------
%----------------------------------------------------------------------
\begin{frame}[c]{Objectives of AutoRL}

	\begin{itemize}
            \item There are several options for what to optimize the hyperparameters
            \item Quality of your value network, for example, TD Error on all the collected observations $(s, a, r, s')$ 
               $$ r + \gamma \max_{a'} \hat{Q}(s',a';\vect{w}) - \hat{Q}(s,a;\vect{w})$$
            \begin{itemize}
                \item Disadvantage: the policy can still be suboptimal even for small TD error
            \end{itemize}
            \smallskip
            \pause
            \item Expected (Discounted) Return
            $$\mathbb{E}[G_t \mid s_t=s]$$
            \begin{itemize}
                \item[$\leadsto$] Evaluate several episodes for the final policy
                \item Advantage: Well aligned with our objective of determining the best possible policy
                \item Disadvantage: Expensive to evaluate since it requires interaction with the environment
            \end{itemize}
        \end{itemize}
	
\end{frame}
%-----------------------------------------------------------------------
\begin{frame}[c]{Grid Search vs. Random Search}
	
\centering
Grid Search \hspace{10em} Random Search:\\
\includegraphics[width=0.4\textwidth, trim=0 0 65em 0em, clip]{w10_autoRL/images/static_grid_new.png}
\includegraphics[width=0.4\textwidth, trim=0 0 65em 0em, clip]{w10_autoRL/images/static_random_new.png}

\end{frame}
%-----------------------------------------------------------------------
%-----------------------------------------------------------------------
\begin{frame}[c]{Grid Search vs. Random Search}
	
\begin{itemize}
    \item Grid Search:
    \begin{itemize}
        \item[+] Structured study
        \item[-] Requires discretization (from an expert)
        \item[-] Doesn't scale well to high dimensions
        \item[-] Cannot be interrupted mid-way
    \end{itemize}
    \item Random Search
    \begin{itemize}
        \item[+/-] Less structured
        \item[+] Works better if effective dimensionality is low 
        \item[+] Better anytime behavior
        \item[+] Easy to add more points later on
        \item[-] Still very inefficient
    \end{itemize}
\end{itemize}
\end{frame}
%-----------------------------------------------------------------------

\end{document}
