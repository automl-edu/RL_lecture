% !TeX spellcheck = en_US

\documentclass[aspectratio=169]{tntbeamer}  % you can pass all options of the beamer class, e.g., 'handout' or 'aspectratio=43'
\usepackage{dsfont}
\usepackage{bm}
\usepackage[english]{babel}
\usepackage[T1]{fontenc}
%\usepackage[utf8]{inputenc}
\usepackage{graphicx}
\graphicspath{ {./figures/} }
\usepackage{algorithm}
\usepackage[ruled,vlined,algo2e,linesnumbered]{algorithm2e}
\usepackage{hyperref}
\usepackage{booktabs}
\usepackage{mathtools}

\usepackage{amsmath,amssymb}
\usepackage{latexsym}

\DeclareMathOperator*{\argmax}{arg\,max}
\DeclareMathOperator*{\argmin}{arg\,min}

\usepackage{pgfplots}
\pgfplotsset{compat=1.16}
\usepackage{tikz}
\usetikzlibrary{trees} 
\usetikzlibrary{shapes.geometric}
\usetikzlibrary{positioning,shapes,shadows,arrows,calc,mindmap}
\usetikzlibrary{positioning,fadings,through}
\usetikzlibrary{decorations.pathreplacing}
\usetikzlibrary{intersections}
\pgfdeclarelayer{background}
\pgfdeclarelayer{foreground}
\pgfsetlayers{background,main,foreground}
\tikzstyle{activity}=[rectangle, draw=black, rounded corners, text centered, text width=8em]
\tikzstyle{data}=[rectangle, draw=black, text centered, text width=8em]
\tikzstyle{myarrow}=[->, thick, draw=black]

% Define the layers to draw the diagram
\pgfdeclarelayer{background}
\pgfdeclarelayer{foreground}
\pgfsetlayers{background,main,foreground}

\input{./latex_main_old/macros}

% Requires XeLaTeX or LuaLaTeX
\usepackage{unicode-math}

\usepackage{fontspec}
%\setsansfont{Arial}
\setsansfont{RotisSansSerifStd}[ 
Path=./latex_main/fonts/,
Extension = .otf,
UprightFont = *-Regular,  % or *-Light
BoldFont = *-ExtraBold,  % or *-Bold
ItalicFont = *-Italic
]
\setmonofont{Cascadia Mono}[
Scale=0.8
]

% scale factor adapted; mathrm font added (Benjamin Spitschan @TNT, 2021-06-01)
%\setmathfont[Scale=1.05]{Libertinus Math}
%\setmathrm[Scale=1.05]{Libertinus Math}

% other available math fonts are (not exhaustive)
% Latin Modern Math
% XITS Math
% Libertinus Math
% Asana Math
% Fira Math
% TeX Gyre Pagella Math
% TeX Gyre Bonum Math
% TeX Gyre Schola Math
% TeX Gyre Termes Math

% Literature References
% #1 = Display Name
% #2 = Url (without \href)
\newcommand{\lit}[2]{\href{#2}{\footnotesize\color{black!60}[#1]}}

%%% Beamer Customization
%----------------------------------------------------------------------
% (Don't) Show sections in frame header. Options: 'sections', 'sections light', empty
\setbeamertemplate{headline}{empty}

% Add header logo for normal frames
\setheaderimage{
	% \includegraphics[height=\logoheight]{figures/TNT_darkv4.pdf}
	\includegraphics[height=\logoheight]{./latex_main/figures/luh_logo_rgb_0_80_155.pdf}
	% \includegraphics[height=\logoheight]{figures/logo_tntluh.pdf}
}

% Header logo for title page
\settitleheaderimage{
	% \includegraphics[height=\logoheight]{figures/TNT_darkv4.pdf}
	\includegraphics[height=\logoheight]{./latex_main/figures/luh_logo_rgb_0_80_155.pdf}
	% \includegraphics[height=\logoheight]{figures/logo_tntluh.pdf}
}

% Title page: tntdefault 
\setbeamertemplate{title page}[tntdefault]  % or luhstyle
% Add optional title image here
%\addtitlepageimagedefault{\includegraphics[width=0.65\textwidth]{figures/luh_default_presentation_title_image.jpg}}

% Title page: luhstyle
% \setbeamertemplate{title page}[luhstyle]
% % Add optional title image here
% \addtitlepageimage{\includegraphics[width=0.75\textwidth]{figures/luh_default_presentation_title_image.jpg}}

\author[Lindauer]{Marius Lindauer\\[1em]
	\includegraphics[height=\logoheight]{./latex_main/figures/luh_logo_rgb_0_80_155.pdf}\qquad
\includegraphics[height=\logoheight]{./latex_main/figures/TNT_darkv4}\qquad
\includegraphics[height=\logoheight]{./latex_main/figures/L3S.jpg}	}
\date{Winter Term 2021
}


%%% Custom Packages
%----------------------------------------------------------------------
% Create dummy content
\usepackage{blindtext}

% Adds a frame with the current page layout. Just call \layout inside of a frame.
\usepackage{layout}


\institute{Institut f\"ur Informationsverarbeitung}%\\ Leibniz Universit\"at Hannover}
%\title{TNT Beamer Template}
%\author{Suomynon A. Anonymous}
\date{}



\title[Reinforcement Learning: Model Free Control]{Model Free Control}
\subtitle{Bias Maximization and Double Q-Learning}



\begin{document}
	
	\maketitle

%----------------------------------------------------------------------
%----------------------------------------------------------------------
\begin{frame}[c]{Maximization Bias}
	
	\begin{itemize}
		\item Consider single-state MDP $(|S| = 1)$ with $2$ actions, and both actions have 0-mean random rewards: $(r \mid a = a_1 ) = (r \mid a = a_2) = 0$
		\begin{itemize}
				\item assume that reward is stochastic (e.g, $\mathcal{N}(0,1)$)
		\end{itemize}
		\item Then $Q(s,a_1) = Q(s,a_2) = 0 = V(s)$
		\item Assume there are prior samples of taking action $a_1$ and $a_2$
		\item Let $\hat{Q}(s,a_1)$, $Q(s,a_2)$ be the \alert{finite} sample estimate of Q
		\item Use an unbiased estimator for $Q$, e.g., $Q(s,a_1) = \frac{1}{N(s,a_1)} \sum_{i=1}^{N(s,a_1)} r_i(s,a_1)$
		\item Let $\hat{\pi}\in \argmax\hat{Q}(s,a)$ be the greedy policy wrt the estimated $\hat{Q}$
		\item Even though each estimate of the state-action values is unbiased, the estimate of $\hat{\pi}$'s value $\hat{V}^{\hat{\hat{\pi}}}$ can be biased:
	\end{itemize}

\vspace{-1em}
\begin{eqnarray}
\hat{V}^{\hat{\pi}}(s) &=& \mathbb{E} [\max \hat{Q}(s,a_1), \hat{Q}(s,a_2)]\nonumber\\
&\geq& \max [\mathbb{E}[\hat{Q}(s,a_1)],[\hat{Q}(s,a_2)]]\nonumber\\
&=& max[0,0] = V^\pi\nonumber
\end{eqnarray}
(where the inequality comes from Jensens' inequality.)
	
\end{frame}
%--------------------------------------------------------------
%----------------------------------------------------------------------
\begin{frame}[c]{Double Q-Learning}	
	
	\begin{itemize}
		\item The greedy policy w.r.t. estimated Q values can yield a maximization
		bias during finite-sample learning
		\item Avoid using max of estimates as estimate of max of true values
		\item Instead split samples and use to create two independent unbiased
		estimates of $Q_1(s_1, a_i)$ and $Q_2(s_1, a_i). \forall a\in A$
		\begin{itemize}
			\item Use one estimate to select max action: $a^* \in \argmax_{a \in A} Q_1(s_1, a)$
			\item Use other estimate to estimate value of $a^*$: $Q_2(s,a^*)$
			\item Yields unbiased estimate: $\mathbb{E}(Q_2(s,a^*)) = Q(s,a^*)$
		\end{itemize}
		\item[$\leadsto$] Unbiased estimate of the max state-action value because of independent samples to estimate the value
	\end{itemize}

	
\end{frame}
%--------------------------------------------------------------
%----------------------------------------------------------------------
\begin{frame}[c]{Double Q-Learning for Full MDP}	
	
	\begin{itemize}
		\item Initialization:
		\begin{itemize}
			\item $Q_1(s,a)$ and $Q_2(s,a)$ for $\forall s \in S, a\in A$
			\item $t= 0$
			\item initial state $s_t = s_0$
		\end{itemize}
		\item Loop
		\begin{itemize}
			\item Select $a_t$ using $\epsilon$-greedy $\pi(s) = \argmax_{a \in A} Q_1(s_t, a) + Q_2(s_t , a)$
			\item Observe $(r_t, s_{t+1})$
			\item With 50-50 probability either
			\begin{enumerate}
				\item $Q_1(s_t, a_t) \gets Q_1(s_t, a_t) + \alpha (r_t +\gamma \max_{a\in A} Q_2(s_{t+1}, a) - Q_1(s_t, a_t))$\\
				or
				\item $Q_2(s_t, a_t) \gets Q_2(s_t, a_t) + \alpha (r_t +\gamma \max_{a\in A} Q_1(s_{t+1}, a) - Q_2(s_t, a_t))$
			\end{enumerate}
			\item $t = t + 1 $
		\end{itemize}
		\bigskip
		\pause
		\item[$\leadsto$] Doubles the memory, same computation requirements, data requirements are subtle -- might reduce amount of exploration needed due to lower bias
	\end{itemize}
	
	
\end{frame}
%--------------------------------------------------------------
%----------------------------------------------------------------------
\begin{frame}[c]{Double Q-Learning \litw{Sutton \& Barto 2018}}	
	
\begin{center}
\includegraphics[width=0.8\textwidth]{images/double_q.png}
\end{center}

Due to the maximization bias, Q-learning spends much more time
selecting sub-optimal actions ("left") than double Q-learning.
	
	
\end{frame}
%--------------------------------------------------------------
%-----------------------------------------------------------------------
\end{document}
