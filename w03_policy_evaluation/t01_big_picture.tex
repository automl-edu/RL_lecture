\documentclass[aspectratio=169]{../latex_main/tntbeamer}  % you can pass all options of the beamer class, e.g., 'handout' or 'aspectratio=43'
\usepackage{dsfont}
\usepackage{bm}
\usepackage[english]{babel}
\usepackage[T1]{fontenc}
%\usepackage[utf8]{inputenc}
\usepackage{graphicx}
\graphicspath{ {./figures/} }
\usepackage{algorithm}
\usepackage[ruled,vlined,algo2e,linesnumbered]{algorithm2e}
\usepackage{hyperref}
\usepackage{booktabs}
\usepackage{mathtools}

\usepackage{amsmath,amssymb}
\usepackage{latexsym}

\DeclareMathOperator*{\argmax}{arg\,max}
\DeclareMathOperator*{\argmin}{arg\,min}

\usepackage{pgfplots}
\pgfplotsset{compat=1.16}
\usepackage{tikz}
\usetikzlibrary{trees} 
\usetikzlibrary{shapes.geometric}
\usetikzlibrary{positioning,shapes,shadows,arrows,calc,mindmap}
\usetikzlibrary{positioning,fadings,through}
\usetikzlibrary{decorations.pathreplacing}
\usetikzlibrary{intersections}
\pgfdeclarelayer{background}
\pgfdeclarelayer{foreground}
\pgfsetlayers{background,main,foreground}
\tikzstyle{activity}=[rectangle, draw=black, rounded corners, text centered, text width=8em]
\tikzstyle{data}=[rectangle, draw=black, text centered, text width=8em]
\tikzstyle{myarrow}=[->, thick, draw=black]

% Define the layers to draw the diagram
\pgfdeclarelayer{background}
\pgfdeclarelayer{foreground}
\pgfsetlayers{background,main,foreground}

\input{./latex_main_old/macros}

% Requires XeLaTeX or LuaLaTeX
\usepackage{unicode-math}

\usepackage{fontspec}
%\setsansfont{Arial}
\setsansfont{RotisSansSerifStd}[ 
Path=./latex_main/fonts/,
Extension = .otf,
UprightFont = *-Regular,  % or *-Light
BoldFont = *-ExtraBold,  % or *-Bold
ItalicFont = *-Italic
]
\setmonofont{Cascadia Mono}[
Scale=0.8
]

% scale factor adapted; mathrm font added (Benjamin Spitschan @TNT, 2021-06-01)
%\setmathfont[Scale=1.05]{Libertinus Math}
%\setmathrm[Scale=1.05]{Libertinus Math}

% other available math fonts are (not exhaustive)
% Latin Modern Math
% XITS Math
% Libertinus Math
% Asana Math
% Fira Math
% TeX Gyre Pagella Math
% TeX Gyre Bonum Math
% TeX Gyre Schola Math
% TeX Gyre Termes Math

% Literature References
% #1 = Display Name
% #2 = Url (without \href)
\newcommand{\lit}[2]{\href{#2}{\footnotesize\color{black!60}[#1]}}

%%% Beamer Customization
%----------------------------------------------------------------------
% (Don't) Show sections in frame header. Options: 'sections', 'sections light', empty
\setbeamertemplate{headline}{empty}

% Add header logo for normal frames
\setheaderimage{
	% \includegraphics[height=\logoheight]{figures/TNT_darkv4.pdf}
	\includegraphics[height=\logoheight]{./latex_main/figures/luh_logo_rgb_0_80_155.pdf}
	% \includegraphics[height=\logoheight]{figures/logo_tntluh.pdf}
}

% Header logo for title page
\settitleheaderimage{
	% \includegraphics[height=\logoheight]{figures/TNT_darkv4.pdf}
	\includegraphics[height=\logoheight]{./latex_main/figures/luh_logo_rgb_0_80_155.pdf}
	% \includegraphics[height=\logoheight]{figures/logo_tntluh.pdf}
}

% Title page: tntdefault 
\setbeamertemplate{title page}[tntdefault]  % or luhstyle
% Add optional title image here
%\addtitlepageimagedefault{\includegraphics[width=0.65\textwidth]{figures/luh_default_presentation_title_image.jpg}}

% Title page: luhstyle
% \setbeamertemplate{title page}[luhstyle]
% % Add optional title image here
% \addtitlepageimage{\includegraphics[width=0.75\textwidth]{figures/luh_default_presentation_title_image.jpg}}

\author[Lindauer]{Marius Lindauer\\[1em]
	\includegraphics[height=\logoheight]{./latex_main/figures/luh_logo_rgb_0_80_155.pdf}\qquad
\includegraphics[height=\logoheight]{./latex_main/figures/TNT_darkv4}\qquad
\includegraphics[height=\logoheight]{./latex_main/figures/L3S.jpg}	}
\date{Winter Term 2021
}


%%% Custom Packages
%----------------------------------------------------------------------
% Create dummy content
\usepackage{blindtext}

% Adds a frame with the current page layout. Just call \layout inside of a frame.
\usepackage{layout}

\title[Reinforcement Learning: Policy Evaluation]{Policy Evaluation}
\subtitle{The Big Picture}



\begin{document}
	
	\maketitle

%----------------------------------------------------------------------
%----------------------------------------------------------------------
\begin{frame}[c]{Recap I : Markov Decision Process (MDP)}

\begin{itemize}
	\item Markov Decision Process is Markov Reward Process + actions
	\item Definition of MDP
	\begin{itemize}
		\item $S$ is a (finite) set of Markov states $s \in S$
		\item $A$ is a (finite) set of actions $a \in A$
		\item $P$ is dynamics/transition model for each action, that specifies $P(s_{t+1} = s' \mid s_t=s, a_t=a)$
		\item $R$ is a reward function 
		$R(s_t=s, a_t=a) = \mathbb{E}[r_r \mid s_t=s, a_t=a] $
		\begin{itemize}
			\item Sometimes R is also defined based on $(s)$ or on $(s,a,s')$
		\end{itemize}
		\item Discount factor $\gamma \in [0, 1]$
	\end{itemize}
	\item MDP is tuple $(S,A,P, R, \gamma)$
\end{itemize}

\pause

\medskip
$\leadsto$ Unfortunately, we often do not have access to true MDP models

\end{frame}
%-----------------------------------------------------------------------
%----------------------------------------------------------------------
\begin{frame}[c]{Recap II}

\begin{itemize}
	\item Definition of Return $G_t$ (for a MRP)
	\begin{itemize}
		\item Discounted sum of rewards from time step t to horizon
	\end{itemize}	
$$ G_t = r_t + \gamma r_{t+1} + \gamma^2 r_{t+2} + \gamma^3 r_{t+3} + \ldots$$

	\medskip
	\pause
	\item Definition of State Value Function $V^\pi (s) $
	\begin{itemize}
		\item Expected return from starting in state $s$ under policy $\pi$
		$$V^\pi (s) = \mathbb{E} [G_t \mid s_t = s] = \mathbb{E}_\pi [r_t + \gamma r_{t+1} + \gamma^2 r_{t+2} + \gamma^3 r_{t+3} + \ldots \mid s_t = s] $$
	\end{itemize}
	\medskip
	\pause
	\item Definition of State-Action Value Function $Q^\pi (s,a)$
	\begin{itemize}
	\item Expected return from starting in state $s$, taking action $a$ and then following
	policy $\pi$
	\begin{eqnarray}
	Q^\pi (s,a) &=& \mathbb{E}_{\pi}[G_t \mid s_t = s, a_t = a]  \nonumber \\
	&=& \mathbb{E}_{\pi}  [r_t + \gamma r_{t+1} + \gamma^2 r_{t+2} + \gamma^3 r_{t+3} + \ldots \mid s_t = s, a_t = a]\nonumber
	\end{eqnarray}

	
	\end{itemize}

\end{itemize}


\end{frame}
%-----------------------------------------------------------------------
%----------------------------------------------------------------------
\begin{frame}[c]{Goal for this Week}

\begin{itemize}
	\item Assumption: We don't have the exact model of the environment\\ (i.e., model-free), but we can query the environment\\ ("playing roll-outs")
	\begin{itemize}
		\item state space and action space are in principle known
		\item we don't know the transition probabilities beforehand
		\item we don't know the reward distribution beforehand
	\end{itemize}
	\medskip
	\pause
	\item Remarks:
	\begin{itemize}
		\item If we would know the MDP, we only have to do "planning" to find the optimal policy
		\item If we first learn the MDP and then apply planning to the learned MDP, we do "model-based" RL (not today!)
	\end{itemize}
	\medskip\pause
	\item Goal for this week: We want to learn $V^\pi(s)$ or $Q^\pi(s,a)$ (depending on the RL algorithm we want to use) by only querying the unknown MDP
\end{itemize}


\end{frame}
%-----------------------------------------------------------------------

%-----------------------------------------------------------------------
\end{document}
