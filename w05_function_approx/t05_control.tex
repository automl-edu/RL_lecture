% !TeX spellcheck = en_US

\documentclass[aspectratio=169]{tntbeamer}  % you can pass all options of the beamer class, e.g., 'handout' or 'aspectratio=43'
\usepackage{dsfont}
\usepackage{bm}
\usepackage[english]{babel}
\usepackage[T1]{fontenc}
%\usepackage[utf8]{inputenc}
\usepackage{graphicx}
\graphicspath{ {./figures/} }
\usepackage{algorithm}
\usepackage[ruled,vlined,algo2e,linesnumbered]{algorithm2e}
\usepackage{hyperref}
\usepackage{booktabs}
\usepackage{mathtools}

\usepackage{amsmath,amssymb}
\usepackage{latexsym}

\DeclareMathOperator*{\argmax}{arg\,max}
\DeclareMathOperator*{\argmin}{arg\,min}

\usepackage{pgfplots}
\pgfplotsset{compat=1.16}
\usepackage{tikz}
\usetikzlibrary{trees} 
\usetikzlibrary{shapes.geometric}
\usetikzlibrary{positioning,shapes,shadows,arrows,calc,mindmap}
\usetikzlibrary{positioning,fadings,through}
\usetikzlibrary{decorations.pathreplacing}
\usetikzlibrary{intersections}
\pgfdeclarelayer{background}
\pgfdeclarelayer{foreground}
\pgfsetlayers{background,main,foreground}
\tikzstyle{activity}=[rectangle, draw=black, rounded corners, text centered, text width=8em]
\tikzstyle{data}=[rectangle, draw=black, text centered, text width=8em]
\tikzstyle{myarrow}=[->, thick, draw=black]

% Define the layers to draw the diagram
\pgfdeclarelayer{background}
\pgfdeclarelayer{foreground}
\pgfsetlayers{background,main,foreground}

\input{./latex_main_old/macros}

% Requires XeLaTeX or LuaLaTeX
\usepackage{unicode-math}

\usepackage{fontspec}
%\setsansfont{Arial}
\setsansfont{RotisSansSerifStd}[ 
Path=./latex_main/fonts/,
Extension = .otf,
UprightFont = *-Regular,  % or *-Light
BoldFont = *-ExtraBold,  % or *-Bold
ItalicFont = *-Italic
]
\setmonofont{Cascadia Mono}[
Scale=0.8
]

% scale factor adapted; mathrm font added (Benjamin Spitschan @TNT, 2021-06-01)
%\setmathfont[Scale=1.05]{Libertinus Math}
%\setmathrm[Scale=1.05]{Libertinus Math}

% other available math fonts are (not exhaustive)
% Latin Modern Math
% XITS Math
% Libertinus Math
% Asana Math
% Fira Math
% TeX Gyre Pagella Math
% TeX Gyre Bonum Math
% TeX Gyre Schola Math
% TeX Gyre Termes Math

% Literature References
% #1 = Display Name
% #2 = Url (without \href)
\newcommand{\lit}[2]{\href{#2}{\footnotesize\color{black!60}[#1]}}

%%% Beamer Customization
%----------------------------------------------------------------------
% (Don't) Show sections in frame header. Options: 'sections', 'sections light', empty
\setbeamertemplate{headline}{empty}

% Add header logo for normal frames
\setheaderimage{
	% \includegraphics[height=\logoheight]{figures/TNT_darkv4.pdf}
	\includegraphics[height=\logoheight]{./latex_main/figures/luh_logo_rgb_0_80_155.pdf}
	% \includegraphics[height=\logoheight]{figures/logo_tntluh.pdf}
}

% Header logo for title page
\settitleheaderimage{
	% \includegraphics[height=\logoheight]{figures/TNT_darkv4.pdf}
	\includegraphics[height=\logoheight]{./latex_main/figures/luh_logo_rgb_0_80_155.pdf}
	% \includegraphics[height=\logoheight]{figures/logo_tntluh.pdf}
}

% Title page: tntdefault 
\setbeamertemplate{title page}[tntdefault]  % or luhstyle
% Add optional title image here
%\addtitlepageimagedefault{\includegraphics[width=0.65\textwidth]{figures/luh_default_presentation_title_image.jpg}}

% Title page: luhstyle
% \setbeamertemplate{title page}[luhstyle]
% % Add optional title image here
% \addtitlepageimage{\includegraphics[width=0.75\textwidth]{figures/luh_default_presentation_title_image.jpg}}

\author[Lindauer]{Marius Lindauer\\[1em]
	\includegraphics[height=\logoheight]{./latex_main/figures/luh_logo_rgb_0_80_155.pdf}\qquad
\includegraphics[height=\logoheight]{./latex_main/figures/TNT_darkv4}\qquad
\includegraphics[height=\logoheight]{./latex_main/figures/L3S.jpg}	}
\date{Winter Term 2021
}


%%% Custom Packages
%----------------------------------------------------------------------
% Create dummy content
\usepackage{blindtext}

% Adds a frame with the current page layout. Just call \layout inside of a frame.
\usepackage{layout}


\institute{Institut f\"ur Informationsverarbeitung}%\\ Leibniz Universit\"at Hannover}
%\title{TNT Beamer Template}
%\author{Suomynon A. Anonymous}
\date{}



\title[Reinforcement Learning: Function Approximation]{Function Approximation}
\subtitle{Control using VFA}


\begin{document}
	
	\maketitle

%----------------------------------------------------------------------
%----------------------------------------------------------------------
\begin{frame}[c]{Control using Value Function Approximation}
	
	\begin{itemize}
		\item Use value function approximation to represent state-action values $\hat{Q}^\pi(s,a;\vec{w}) \approx Q^\pi$
		\item Interleave
		\begin{itemize}
			\item Approximate policy evaluation using value function approximation
			\item Perform $\epsilon$-greedy policy improvement
		\end{itemize}
		\item Can be unstable. Generally involves intersection of the following:
		\begin{itemize}
			\item Function approximation
			\item Bootstrapping
			\item \alert{Off-policy learning}
		\end{itemize}
	\end{itemize}

\end{frame}
%-----------------------------------------------------------------------
%----------------------------------------------------------------------
\begin{frame}[c]{Action-Value Function Approximation with an Oracle}
	
	\begin{itemize}
		\item $\hat{Q}^\pi(s,a;\vec{w}) \approx Q^\pi$
		\item Minimize the mean-squared error between the true action-value function $Q^\pi(s,a)$ and the approximate action-value function:
		$$J(\vec{w}) = \mathbb{E}_\pi [(Q^\pi(s,a) - \hat{Q}^\pi(s,a;\vec{w}))^2] $$
		\item Use stochastic gradient descent to find a local minimum
		\begin{eqnarray}
			-\frac{1}{2}\nabla_\vec{w} J(\vec{w}) &=& \mathbb{E}\left[ (Q^\pi(s,a) - \hat{Q}^\pi(s,a;\vec{w})) \nabla_\vec{w} \hat{Q}^\pi(s,a;\vec{w}) \right]\nonumber\\
			\Delta \vec{w} &=& -\frac{1}{2}\alpha\nabla_\vec{w} J(\vec{w})\nonumber
		\end{eqnarray}
		\item Stochastic gradient descent (SGD) samples the gradient
	\end{itemize}
	
\end{frame}
%-----------------------------------------------------------------------
%----------------------------------------------------------------------
\begin{frame}[c]{Linear State Action Value Function Approximation with an
		Oracle}
	
	\begin{itemize}
		\item Use features to represent both the state and action
		$$\vec{x}(s,a) = \begin{pmatrix}
		\vec{x}_1(s,a)\\
		\vec{x}_2(s,a)\\
		\ldots\\
		\vec{x}_n(s,a)
		\end{pmatrix} $$
		\item Represent state-action value function with a weighted linear
		combination of features
		$$\hat{Q}(s,a;\vec{w}) = \vec{x}(s,a)^T \vec{w} = \sum_{j=1}^n x_j(s,a)w_j $$
		\item Stochastic gradient descent update
		$$\nabla_{\vec{w}} J(\vec{w}) = \nabla_{\vec{w}} \mathbb{E}_\pi [(Q^\pi(s,a) - \hat{Q}^\pi(s,a;\vec{w}))^2] $$
	\end{itemize}
	
\end{frame}
%-----------------------------------------------------------------------
%----------------------------------------------------------------------
\begin{frame}[c]{Incremental Model-Free Control Approaches}
	
	\begin{itemize}
		\item Similar to policy evaluation, true state-action value function for a state is unknown and so substitute a target value
		\item In Monte Carlo methods, use a return $G_t$ as a substitute target
		$$\Delta \vec{w} = \alpha(G_t - \hat{Q}(s_t,a_t; \vec{w})) \nabla_{\vec{w}} \hat{Q}(s_t, a_t; \vec{w}) $$
		\item For SARSA instead use a TD target $r+ \gamma \hat{Q}(s', a'; \vec{w})$ which leverages the current function approximations value
		$$\Delta \vec{w} = \alpha (r + \gamma \hat{Q}(s',a';\vec{w}) - \hat{Q}(s,a;\vec{w})) \nabla_{\vec{w}}\hat{Q}(s,a;\vec{w}) $$
		\item For Q-learning instead use a TD target $r + \gamma \max_{a'} \hat{Q}(s',a';\vec{w})$ which leverages the max of the current function approximations value
		$$\Delta \vec{w} = \alpha (r + \gamma \max_{a'} \hat{Q}(s',a';\vec{w}) - \hat{Q}(s,a;\vec{w})) \nabla_{\vec{w}}\hat{Q}(s,a;\vec{w}) $$
	\end{itemize}
	
\end{frame}
%-----------------------------------------------------------------------
%-----------------------------------------------------------------------
\end{document}
