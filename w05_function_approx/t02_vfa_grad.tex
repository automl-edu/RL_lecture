% !TeX spellcheck = en_US

\documentclass[aspectratio=169]{tntbeamer}  % you can pass all options of the beamer class, e.g., 'handout' or 'aspectratio=43'
\usepackage{dsfont}
\usepackage{bm}
\usepackage[english]{babel}
\usepackage[T1]{fontenc}
%\usepackage[utf8]{inputenc}
\usepackage{graphicx}
\graphicspath{ {./figures/} }
\usepackage{algorithm}
\usepackage[ruled,vlined,algo2e,linesnumbered]{algorithm2e}
\usepackage{hyperref}
\usepackage{booktabs}
\usepackage{mathtools}

\usepackage{amsmath,amssymb}
\usepackage{latexsym}

\DeclareMathOperator*{\argmax}{arg\,max}
\DeclareMathOperator*{\argmin}{arg\,min}

\usepackage{pgfplots}
\pgfplotsset{compat=1.16}
\usepackage{tikz}
\usetikzlibrary{trees} 
\usetikzlibrary{shapes.geometric}
\usetikzlibrary{positioning,shapes,shadows,arrows,calc,mindmap}
\usetikzlibrary{positioning,fadings,through}
\usetikzlibrary{decorations.pathreplacing}
\usetikzlibrary{intersections}
\pgfdeclarelayer{background}
\pgfdeclarelayer{foreground}
\pgfsetlayers{background,main,foreground}
\tikzstyle{activity}=[rectangle, draw=black, rounded corners, text centered, text width=8em]
\tikzstyle{data}=[rectangle, draw=black, text centered, text width=8em]
\tikzstyle{myarrow}=[->, thick, draw=black]

% Define the layers to draw the diagram
\pgfdeclarelayer{background}
\pgfdeclarelayer{foreground}
\pgfsetlayers{background,main,foreground}

\input{./latex_main_old/macros}

% Requires XeLaTeX or LuaLaTeX
\usepackage{unicode-math}

\usepackage{fontspec}
%\setsansfont{Arial}
\setsansfont{RotisSansSerifStd}[ 
Path=./latex_main/fonts/,
Extension = .otf,
UprightFont = *-Regular,  % or *-Light
BoldFont = *-ExtraBold,  % or *-Bold
ItalicFont = *-Italic
]
\setmonofont{Cascadia Mono}[
Scale=0.8
]

% scale factor adapted; mathrm font added (Benjamin Spitschan @TNT, 2021-06-01)
%\setmathfont[Scale=1.05]{Libertinus Math}
%\setmathrm[Scale=1.05]{Libertinus Math}

% other available math fonts are (not exhaustive)
% Latin Modern Math
% XITS Math
% Libertinus Math
% Asana Math
% Fira Math
% TeX Gyre Pagella Math
% TeX Gyre Bonum Math
% TeX Gyre Schola Math
% TeX Gyre Termes Math

% Literature References
% #1 = Display Name
% #2 = Url (without \href)
\newcommand{\lit}[2]{\href{#2}{\footnotesize\color{black!60}[#1]}}

%%% Beamer Customization
%----------------------------------------------------------------------
% (Don't) Show sections in frame header. Options: 'sections', 'sections light', empty
\setbeamertemplate{headline}{empty}

% Add header logo for normal frames
\setheaderimage{
	% \includegraphics[height=\logoheight]{figures/TNT_darkv4.pdf}
	\includegraphics[height=\logoheight]{./latex_main/figures/luh_logo_rgb_0_80_155.pdf}
	% \includegraphics[height=\logoheight]{figures/logo_tntluh.pdf}
}

% Header logo for title page
\settitleheaderimage{
	% \includegraphics[height=\logoheight]{figures/TNT_darkv4.pdf}
	\includegraphics[height=\logoheight]{./latex_main/figures/luh_logo_rgb_0_80_155.pdf}
	% \includegraphics[height=\logoheight]{figures/logo_tntluh.pdf}
}

% Title page: tntdefault 
\setbeamertemplate{title page}[tntdefault]  % or luhstyle
% Add optional title image here
%\addtitlepageimagedefault{\includegraphics[width=0.65\textwidth]{figures/luh_default_presentation_title_image.jpg}}

% Title page: luhstyle
% \setbeamertemplate{title page}[luhstyle]
% % Add optional title image here
% \addtitlepageimage{\includegraphics[width=0.75\textwidth]{figures/luh_default_presentation_title_image.jpg}}

\author[Lindauer]{Marius Lindauer\\[1em]
	\includegraphics[height=\logoheight]{./latex_main/figures/luh_logo_rgb_0_80_155.pdf}\qquad
\includegraphics[height=\logoheight]{./latex_main/figures/TNT_darkv4}\qquad
\includegraphics[height=\logoheight]{./latex_main/figures/L3S.jpg}	}
\date{Winter Term 2021
}


%%% Custom Packages
%----------------------------------------------------------------------
% Create dummy content
\usepackage{blindtext}

% Adds a frame with the current page layout. Just call \layout inside of a frame.
\usepackage{layout}


\institute{Institut f\"ur Informationsverarbeitung}%\\ Leibniz Universit\"at Hannover}
%\title{TNT Beamer Template}
%\author{Suomynon A. Anonymous}
\date{}



\title[Reinforcement Learning: Function Approximation]{Function Approximation}
\subtitle{Gradient Descent and Linear Models}


\begin{document}
	
	\maketitle

%----------------------------------------------------------------------
%----------------------------------------------------------------------
\begin{frame}[c]{Overview}
	
	
\begin{itemize}
	\item Represent a (state-action/state) value function with a parameterized
	function instead of a table
\end{itemize}

\begin{center}
	\includegraphics[width=0.6\textwidth]{images/vfa.png}
\end{center}

\begin{itemize}
	\item \alert{Which function approximator}
\end{itemize}

\end{frame}
%-----------------------------------------------------------------------
%----------------------------------------------------------------------
\begin{frame}[c]{Function Approximators}
	
	
	\begin{itemize}
		\item Many possible function approximators including
		\begin{itemize}
			\item linear combinations of features
			\item Neural networks
			\item Decision trees
			\item Nearest neighbors 
			\item Fourier / wavelet bases
		\end{itemize}
		\item Focus on differentiable function approximators
		\item Let's start with linear feature representations
	\end{itemize}
	
\end{frame}
%-----------------------------------------------------------------------
%----------------------------------------------------------------------
\begin{frame}[c]{Recap: Gradient Descent}
	
	
	\begin{itemize}
		\item Consider a function $J(\vec{w})$ that is differentiable function of a parameter vector $\vec{w}$
		\item Goal is to find parameter $\vec{w}$ that minimizes $J$
		\item The gradient of $J(\vec{w})$ is 
	\end{itemize}
	$$
	\nabla J(\vec{w}) = \left[ \frac{\partial J}{\vec{w}_1} \ldots \frac{\partial J}{\vec{w}_n} \right]
	$$
	$$\vec{w}_t = \vec{w}_{t-1} - \alpha \nabla_w J(\vec{w})$$
	
	where $\alpha$ is the learning rate.

	
\end{frame}
%-----------------------------------------------------------------------
%----------------------------------------------------------------------
\begin{frame}[c]{Value Function Approximation for Policy Evaluation with
		an Oracle}
	
	
	\begin{itemize}
		\item First assume we could query any state s and an \alert{oracle} would return
		the true value for $V^\pi (s)$
		\item The objective was to find the best approximate representation of $V^\pi$
		given a particular parameterized function
	\end{itemize}
	
\end{frame}
%-----------------------------------------------------------------------
%----------------------------------------------------------------------
\begin{frame}[c]{Stochastic Gradient Descent}
	
	
	\begin{itemize}
		\item Goal: Find the parameter vector $\vec{w}$ that minimizes the loss between a
		true value function $V^\pi(s)$ and its approximation $\hat{V}^\pi(s; \vec{w})$ as
		represented with a particular function class parameterized by $\vec{w}$.
		\item Generally use mean squared error and define the loss as 
		$$ J(\vec{w}) = \mathbb{E}_\pi [(V^\pi(s) - \hat{V}^\pi(s;\vec{w}))^2]$$
		\item Use gradient descent to find a local minimum 
		$$ \Delta \vec{w} = - \frac{1}{2} \alpha \nabla_\vec{w} J(\vec{w})$$
		\item Stochastic gradient descent (SGD) uses a finite number of samples to compute an approximate gradient:
		$$ \nabla_\vec{w} J(\vec{w}) = \nabla_{\vec{w}} \mathbb{E}_\pi[V^\pi (s) - \hat{V}^\pi (s; \vec{w})]^2$$
		$$= \mathbb{E}_\pi [2 (V^\pi(s) - \hat{V}^\pi (s;\vec{w})) \nabla_\vec{w} \hat{V}(s,\vec{w})]$$
	\end{itemize}
	
\end{frame}
%-----------------------------------------------------------------------
%----------------------------------------------------------------------
\begin{frame}[c]{Model Free VFA Policy Evaluation}
	
	
	\begin{itemize}
		\item In practice, we don’t actually have access to an oracle to tell true $V^\pi(s)$ for any
		state $s$
		\item Now consider how to do model-free value function approximation for
		prediction / evaluation / policy evaluation without a model
	\end{itemize}
	
\end{frame}
%-----------------------------------------------------------------------
%----------------------------------------------------------------------
\begin{frame}[c]{Model Free VFA Prediction / Policy Evaluation}
	
	
	\begin{itemize}
		\item Recall model-free policy evaluation
		\begin{itemize}
			\item Following a fixed policy $\pi$ (or had access to prior data)
			\item Goal is to estimate $V^\pi$ and/or $Q^\pi$
		\end{itemize}
		\item Maintained a lookup table to store estimates $V^\pi$ and/or $Q^\pi$
		\item Updated these estimates after each episode (Monte Carlo methods)
		or after each step (TD methods)
		\item New: in value function approximation, change the estimate
		update step to include fitting the function approximator
	\end{itemize}
	
\end{frame}
%-----------------------------------------------------------------------
%----------------------------------------------------------------------
\begin{frame}[c]{Model Free VFA Prediction / Policy Evaluation}
	
	
	\begin{itemize}
		\item Use a feature vector to represent a state $s$
	\end{itemize}

$$\vec{x}(s) = \begin{pmatrix}
\vec{x}_1(s)\\
\vec{x}_2(s)\\
\ldots\\
\vec{x}_n(s)
\end{pmatrix} $$

\begin{itemize}
	\item For table lookups, we have not really needed that because we only needed to know which table entry to look up
\end{itemize}
	
\end{frame}
%-----------------------------------------------------------------------
%----------------------------------------------------------------------
\begin{frame}[c]{Linear Value Function Approximation for Prediction With
		An Oracle}
	
	
	\begin{itemize}
		\item Represent a value function (or state-action value function) for a
		particular policy with a weighted linear combination of features
		$$ \hat{V}(s; \vec{w}) = \sum_{j=1}^n \vec{x}_j (s) \vec{w}_j = \vec{x}(s)^T\vec{w}$$
		\item Objective function is 
		$$ J(\vec{w}) = \mathbb{E}[(V^\pi(s) - \hat{V}^\pi(s; \vec{w}))^2]$$
		\item Recall weight update:
		$$ \Delta \vec{w} = - \frac{1}{2} \alpha \nabla_{\vec{w}} J (\vec{w})$$
		\item Update (- step size $\times$ prediction error $\times$ feature value)
		$$ \Delta \vec{w} = -\frac{1}{2} \alpha(2(V^\pi(s) - \vec{x}(s)^T \vec{w})) \vec{x}(s)$$
		
	\end{itemize}
	
\end{frame}
%-----------------------------------------------------------------------
%-----------------------------------------------------------------------
\end{document}
