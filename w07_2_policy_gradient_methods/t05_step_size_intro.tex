% !TeX spellcheck = en_US
\documentclass[aspectratio=169]{../latex_main/tntbeamer}  % you can pass all options of the beamer class, e.g., 'handout' or 'aspectratio=43'
\usepackage{dsfont}
\usepackage{bm}
\usepackage[english]{babel}
\usepackage[T1]{fontenc}
%\usepackage[utf8]{inputenc}
\usepackage{graphicx}
\graphicspath{ {./figures/} }
\usepackage{algorithm}
\usepackage[ruled,vlined,algo2e,linesnumbered]{algorithm2e}
\usepackage{hyperref}
\usepackage{booktabs}
\usepackage{mathtools}

\usepackage{amsmath,amssymb}
\usepackage{latexsym}

\DeclareMathOperator*{\argmax}{arg\,max}
\DeclareMathOperator*{\argmin}{arg\,min}

\usepackage{pgfplots}
\pgfplotsset{compat=1.16}
\usepackage{tikz}
\usetikzlibrary{trees} 
\usetikzlibrary{shapes.geometric}
\usetikzlibrary{positioning,shapes,shadows,arrows,calc,mindmap}
\usetikzlibrary{positioning,fadings,through}
\usetikzlibrary{decorations.pathreplacing}
\usetikzlibrary{intersections}
\pgfdeclarelayer{background}
\pgfdeclarelayer{foreground}
\pgfsetlayers{background,main,foreground}
\tikzstyle{activity}=[rectangle, draw=black, rounded corners, text centered, text width=8em]
\tikzstyle{data}=[rectangle, draw=black, text centered, text width=8em]
\tikzstyle{myarrow}=[->, thick, draw=black]

% Define the layers to draw the diagram
\pgfdeclarelayer{background}
\pgfdeclarelayer{foreground}
\pgfsetlayers{background,main,foreground}

\newcommand{\comment}[1]{
	\noindent
	%\vspace{0.25cm}
	{\color{red}{\textbf{TODO:} #1}}
	%\vspace{0.25cm}
}
\newcommand{\notefh}[1]{\textcolor{red}{\textbf{FH:} #1}}
\renewcommand{\comment}[1]{}
\newcommand{\hide}[1]{}
\newcommand{\cemph}[2]{\emph{\textcolor{#1}{#2}}}

%\newcommand{\lit}[1]{{\footnotesize\color{black!60}[#1]}}

%\newcommand{\litw}[1]{{\footnotesize\color{blue!20}[#1]}}


\newcommand{\myframe}[2]{\begin{frame}[c]{#1}#2\end{frame}}
\newcommand{\myframetop}[2]{\begin{frame}{#1}#2\end{frame}}
\newcommand{\myit}[1]{\begin{itemize}#1\end{itemize}}
\newcommand{\myblock}[2]{\begin{block}{#1}#2\end{block}}


\newcommand{\votepurple}[1]{\textcolor{Purple}{$\bigstar$}}
\newcommand{\voteyellow}[1]{\textcolor{Goldenrod}{$\bigstar$}}
\newcommand{\voteblue}[1]{\textcolor{RoyalBlue}{$\bigstar$}}
\newcommand{\votepink}[1]{\textcolor{Pink}{$\bigstar$}}

\newcommand{\diff}{\mathop{}\!\mathrm{d}}
\newcommand{\refstyle}[1]{{\small{\textcolor{gray}{#1}}}}
\newcommand{\hands}[0]{\includegraphics[height=1.5em]{images/hands}}
\newcommand{\transpose}[0]{{\textrm{\tiny{\sf{T}}}}}
\newcommand{\norm}{{\mathcal{N}}}
\newcommand{\cutoff}[0]{\kappa}
\newcommand{\instD}[0]{\dataset}
\newcommand{\insts}[0]{\mathcal{I}}
\newcommand{\inst}[0]{i}
\newcommand{\instI}[1]{i^{(#1)}}

% Iteration specific instance of variable/function/anything
% Introduced in the BO section, but moved up here to make it available within other macros
\newcommand{\iter}[2][\bocount]{{#2}^{(#1)}}

%--------HPO parameter macros-----------

% Parameter Configuration Space
\newcommand{\pcs}[0]{\pmb{\Lambda}}

% ???
\newcommand{\bx}[0]{\conf}

% Parameter Configuration
\newcommand{\conf}[0]{\pmb{\lambda}}

% Final Configuration
\newcommand{\finconf}[0]{\pmb{\hat{\lambda}}}

% Configuration corresponding to a given iteration -- better use \iter!
\newcommand{\confI}[1]{{\conf}^{(#1)}}

% Default Configuration
\newcommand{\defconf}[0]{{\conf}_{\text{def}}}

% Incumbent Configuration
\newcommand{\incumbent}[1][\bocount]{\iter[#1]{\finconf}}

% Optimal Configuration
\newcommand{\optconf}[0]{{\conf}^*}

% Configuration Space
\newcommand{\confs}[0]{\pcs}

%----------------------------------------

%\newcommand{\vlambda}[0]{\bm{\lambda}}
%\newcommand{\vLambda}[0]{\bm{\Lambda}}
\newcommand{\dataset}[0]{\mathcal{D}}
\newcommand{\datasets}[0]{\mathbf{D}}
\newcommand{\loss}[0]{L}
\newcommand{\risk}{\mathcal{R}}
\newcommand{\riske}{\mathcal{R}_{\text{emp}}}
\newcommand{\cost}[0]{c}
\newcommand{\costI}[1]{c^{(#1)}}

% Gaussian Process
\newcommand{\gp}{\mathcal{G}}
% Family of Objective Functions
\newcommand{\objF}{F}

%---------------BO Macros------------------

% BO loop counter
\newcommand{\bocount}{t}
% BO loop counter max, the counter runs from 1 to this value
\newcommand{\bobudget}{T}
% BO loop observation
\newcommand{\obs}[1][\conf]{\cost({#1})}
% BO loop observation space
\newcommand{\obsspace}{\mathcal{Y}}
% BO loop next observation
\newcommand{\bonextobs}{\obs[\iter{\conf}]}
% Acquisition Function, no args
\newcommand{\acq}{u}
% Standard Normal PDF
\newcommand{\pdf}{\phi}
% Standard Normal CDF
\newcommand{\cdf}{\Phi}
% Mean
\newcommand{\mean}{\mu}
% Standard Deviation
\newcommand{\stddev}{\sigma}
% Variance
\newcommand{\variance}{\sigma^2}
% Noise
\newcommand{\noise}{\nu}
% BO loop next selected sample
\newcommand{\bonextsample}{\confI{\bocount}}

% Single hyperparameter
\newcommand{\hyperparam}{\lambda}

% Single hyperparameter within a hyperparameter configuration
\newcommand{\hyperparami}[1][i]{{\hyperparam}_#1}

% Full definition of final configuration
\newcommand{\finconffull}{\incumbent[\bobudget]}

% Dataset
\newcommand{\datasetHPO}{{\dataset}_{HPO}}

% Dataset definition
\newcommand{\datasetHPOdef}{{\langle \bonextsample,\,\bonextobs \rangle}_{\bocount=1}^{\bobudget}}

% Double Display Fraction, forces large displays for everything in numerator and denominator
\newcommand\ddfrac[2]{\frac{\displaystyle #1}{\displaystyle #2}}

% Conditional Probability "Given That" Relation, source:https://tex.stackexchange.com/a/141685/205886
\newcommand\given[1][]{\:#1\vert\:}

% Expectation as a math operator
\DeclareMathOperator*{\E}{\mathbb{E}}

% Citation 
\newcommand{\source}[1]{
    \begin{flushright}
    	Source: \lit{#1}
    \end{flushright}
}
%-------------------------------------------

%Real numbers set
\newcommand{\realnum}{\mathbb{R}}
%Configuration space - do not use
%\newcommand{\configspace}{\Theta}
%Instances - do not use
%\newcommand{\instances}{\mathcal{I}}
%Expected value
\newcommand{\expectation}{\mathbb{E}}
%Kernel
\newcommand{\kernel}{\kappa}
%Constraint function
\newcommand{\constraintf}{c}
%Normal distribution
\newcommand{\normaldist}{\mathcal{N}}

% \renewcommand{\vec}[1]{\mathbf{#1}}
\newcommand{\hist}[0]{\dataset_{\text{Hist}}}
\newcommand{\param}[0]{p}
\newcommand{\algo}[0]{\mathcal{A}}
\newcommand{\algos}[0]{\mathbf{A}}
%\newcommand{\nn}[0]{N}
\newcommand{\feats}[0]{\mathcal{X}_{\text{meta}}}
\newcommand{\feat}[0]{\x_{\text{meta}}}
%\newcommand{\cluster}[0]{\vec{h}}
%\newcommand{\clusters}[0]{\vec{H}}
\newcommand{\perf}[0]{\mathbb{R}}
%\newcommand{\surro}[0]{\mathcal{S}}
\newcommand{\surro}[0]{\hat{\cost}}
\newcommand{\func}[0]{f}
\newcommand{\epm}[0]{\surro}
\newcommand{\portfolio}[0]{\mathbf{P}}
\newcommand{\schedule}[0]{\mathcal{S}}

% Machine Learning
\newcommand{\mdata}[0]{\dataset_{\text{meta}}}
\newcommand{\datasettrain}[0]{\dataset_{\text{train}}}
\newcommand{\datasetval}[0]{\dataset_{\text{val}}}
\newcommand{\datasettest}[0]{\dataset_{\text{test}}}
\newcommand{\x}[0]{\mathbf{x}}
\newcommand{\y}[0]{y}
\newcommand{\xI}[1]{\mathbf{x}^{(#1)}}
\newcommand{\yI}[1]{y^{(#1)}}
\newcommand{\fx}{f(\mathbf{x})}  % f(x), continuous prediction function
\newcommand{\Hspace}{\mathcal{H}} % hypothesis space where f is from
\newcommand{\fh}{\hat{f}}       % f hat, estimated prediction function

% Deep Learning
\newcommand{\weights}[0]{\theta}
\newcommand{\metaweights}[0]{\phi}


% reinforcement learning
\newcommand{\policies}[0]{\mathbf{\Pi}}
\newcommand{\policy}[0]{\pi}
\newcommand{\actionRL}[0]{a}
\newcommand{\stateRL}[0]{s}
\newcommand{\statesRL}[0]{\mathcal{S}}
\newcommand{\rewardRL}[0]{r}
\newcommand{\rewardfuncRL}[0]{\mathcal{R}}

%\RestyleAlgo{algoruled}
%\DontPrintSemicolon
%\LinesNumbered
%\SetAlgoVlined
%\SetFuncSty{textsc}

%\SetKwInOut{Input}{Input}
%\SetKwInOut{Output}{Output}
%\SetKw{Return}{return}

%\newcommand{\changed}[1]{{\color{red}#1}}

%\newcommand{\citeN}[1]{\citeauthor{#1}~(\citeyear{#1})}

\renewcommand{\vec}[1]{\mathbf{#1}}
%\DeclareMathOperator*{\argmin}{arg\,min}
%\DeclareMathOperator*{\argmax}{arg\,max}

%\newcommand{\aqme}{\textit{AQME}}
%\newcommand{\aslib}{\textit{ASlib}}
%\newcommand{\llama}{\textit{LLAMA}}
%\newcommand{\satzilla}{\textit{SATzilla}}
%\newcommand{\satzillaY}[1]{\textit{SATzilla'{#1}}}
%\newcommand{\snnap}{\textit{SNNAP}}
%\newcommand{\claspfolioTwo}{\textit{claspfolio~2}}
%\newcommand{\flexfolio}{\textit{FlexFolio}}
%\newcommand{\claspfolioOne}{\textit{claspfolio~1}}
%\newcommand{\isac}{\textit{ISAC}}
%\newcommand{\eisac}{\textit{EISAC}}
%\newcommand{\sss}{\textit{3S}}
%\newcommand{\sunny}{\textit{Sunny}}
%\newcommand{\ssspar}{\textit{3Spar}}
%\newcommand{\cshc}{\textit{CSHC}}
%\newcommand{\cshcpar}{\textit{CSHCpar}}
%\newcommand{\measp}{\textit{ME-ASP}}
%\newcommand{\aspeed}{\textit{aspeed}}
%\newcommand{\autofolio}{\textit{AutoFolio}}
%\newcommand{\cedalion}{\textit{Cedalion}}
\newcommand{\fanova}{\textit{fANOVA}}
\newcommand{\sbs}{\textit{SB}}
\newcommand{\oracle}{\textit{VBS}}

% like approaches
\newcommand{\claspfoliolike}[1]{\texttt{claspfolio-#1-like}}
\newcommand{\satzillalike}[1]{\texttt{SATzilla'#1-like}}
\newcommand{\isaclike}{\texttt{ISAC-like}}
\newcommand{\ssslike}{\texttt{3S-like}}
\newcommand{\measplike}{\texttt{ME-ASP-like}}

\newcommand{\irace}{\textit{I/F-race}}
\newcommand{\gga}{\textit{GGA}}
\newcommand{\smac}{\textit{SMAC}}
\newcommand{\paramils}{\textit{ParamILS}}
\newcommand{\spearmint}{\textit{Spearmint}}
\newcommand{\tpe}{\textit{TPE}}


\usepackage{pifont}
\newcommand{\itarrow}{\mbox{\Pisymbol{pzd}{229}}}
\newcommand{\ithook}{\mbox{\Pisymbol{pzd}{52}}}
\newcommand{\itcross}{\mbox{\Pisymbol{pzd}{56}}}
\newcommand{\ithand}{\mbox{\raisebox{-1pt}{\Pisymbol{pzd}{43}}}}

%\DeclareMathOperator*{\argmax}{arg\,max}

\newcommand{\ie}{{\it{}i.e.\/}}
\newcommand{\eg}{{\it{}e.g.\/}}
\newcommand{\cf}{{\it{}cf.\/}}
\newcommand{\wrt}{\mbox{w.r.t.} }
\newcommand{\vs}{{\it{}vs\/}}
\newcommand{\vsp}{{\it{}vs\/}}
\newcommand{\etc}{{\copyedit{etc.}}}
\newcommand{\etal}{{\it{}et al.\/}}

\newcommand{\pscProc}{{\bf procedure}}
\newcommand{\pscBegin}{{\bf begin}}
\newcommand{\pscEnd}{{\bf end}}
\newcommand{\pscEndIf}{{\bf endif}}
\newcommand{\pscFor}{{\bf for}}
\newcommand{\pscEach}{{\bf each}}
\newcommand{\pscThen}{{\bf then}}
\newcommand{\pscElse}{{\bf else}}
\newcommand{\pscWhile}{{\bf while}}
\newcommand{\pscIf}{{\bf if}}
\newcommand{\pscRepeat}{{\bf repeat}}
\newcommand{\pscUntil}{{\bf until}}
\newcommand{\pscWithProb}{{\bf with probability}}
\newcommand{\pscOtherwise}{{\bf otherwise}}
\newcommand{\pscDo}{{\bf do}}
\newcommand{\pscTo}{{\bf to}}
\newcommand{\pscOr}{{\bf or}}
\newcommand{\pscAnd}{{\bf and}}
\newcommand{\pscNot}{{\bf not}}
\newcommand{\pscFalse}{{\bf false}}
\newcommand{\pscEachElOf}{{\bf each element of}}
\newcommand{\pscReturn}{{\bf return}}

%\newcommand{\param}[1]{{\sl{}#1}}
\newcommand{\var}[1]{{\it{}#1}}
\newcommand{\cond}[1]{{\sf{}#1}}
%\newcommand{\state}[1]{{\sf{}#1}}
%\newcommand{\func}[1]{{\sl{}#1}}
\newcommand{\set}[1]{{\Bbb #1}}
%\newcommand{\inst}[1]{{\tt{}#1}}
\newcommand{\myurl}[1]{{\small\sf #1}}

\newcommand{\Nats}{{\Bbb N}}
\newcommand{\Reals}{{\Bbb R}}
\newcommand{\extset}[2]{\{#1 \; | \; #2\}}

\newcommand{\vbar}{$\,\;|$\hspace*{-1em}\raisebox{-0.3mm}{$\,\;\;|$}}
\newcommand{\vendbar}{\raisebox{+0.4mm}{$\,\;|$}}
\newcommand{\vend}{$\,\:\lfloor$}


\newcommand{\goleft}[2][.7]{\parbox[t]{#1\linewidth}{\strut\raggedright #2\strut}}
\newcommand{\rightimage}[2][.3]{\mbox{}\hfill\raisebox{1em-\height}[0pt][0pt]{\includegraphics[width=#1\linewidth]{#2}}\vspace*{-\baselineskip}}


% Requires XeLaTeX or LuaLaTeX
\usepackage{unicode-math}

\usepackage{fontspec}
%\setsansfont{Arial}
\setsansfont{RotisSansSerifStd}[ 
Path=./latex_main/fonts/,
Extension = .otf,
UprightFont = *-Regular,  % or *-Light
BoldFont = *-ExtraBold,  % or *-Bold
ItalicFont = *-Italic
]
\setmonofont{Cascadia}[
	Path=./latex_main/fonts/,
	Scale=0.8
]

% scale factor adapted; mathrm font added (Benjamin Spitschan @TNT, 2021-06-01)
%\setmathfont[Scale=1.05]{Libertinus Math}
%\setmathrm[Scale=1.05]{Libertinus Math}

% other available math fonts are (not exhaustive)
% Latin Modern Math
% XITS Math
% Libertinus Math
% Asana Math
% Fira Math
% TeX Gyre Pagella Math
% TeX Gyre Bonum Math
% TeX Gyre Schola Math
% TeX Gyre Termes Math

% Literature References
% #1 = Display Name
% #2 = Url (without \href)
\newcommand{\lit}[2]{\href{#2}{\footnotesize\color{black!60}[#1]}}

%%% Beamer Customization
%----------------------------------------------------------------------
% (Don't) Show sections in frame header. Options: 'sections', 'sections light', empty
\setbeamertemplate{headline}{empty}

% Add header logo for normal frames
\setheaderimage{
	% \includegraphics[height=\logoheight]{figures/TNT_darkv4.pdf}
	\includegraphics[height=\logoheight]{../latex_main/figures/luh_logo_rgb_0_80_155.pdf}
	% \includegraphics[height=\logoheight]{figures/logo_tntluh.pdf}
}

% Header logo for title page
\settitleheaderimage{
	% \includegraphics[height=\logoheight]{figures/TNT_darkv4.pdf}
	\includegraphics[height=\logoheight]{../latex_main/figures/luh_logo_rgb_0_80_155.pdf}
	% \includegraphics[height=\logoheight]{figures/logo_tntluh.pdf}
}

% Title page: tntdefault 
\setbeamertemplate{title page}[tntdefault]  % or luhstyle
% Add optional title image here
%\addtitlepageimagedefault{\includegraphics[width=0.65\textwidth]{figures/luh_default_presentation_title_image.jpg}}

% Title page: luhstyle
% \setbeamertemplate{title page}[luhstyle]
% % Add optional title image here
% \addtitlepageimage{\includegraphics[width=0.75\textwidth]{figures/luh_default_presentation_title_image.jpg}}

\author[Lindauer]{Marius Lindauer\\[1em]
	\includegraphics[height=\logoheight]{../latex_main/figures/luh_logo_rgb_0_80_155.pdf}\qquad
\includegraphics[height=\logoheight]{../latex_main/figures/TNT_darkv4}\qquad
\includegraphics[height=\logoheight]{../latex_main/figures/L3S.jpg}	}
\date{Winter Term 2021
}


%%% Custom Packages
%----------------------------------------------------------------------
% Create dummy content
\usepackage{blindtext}

% Adds a frame with the current page layout. Just call \layout inside of a frame.
\usepackage{layout}

\title[RL: Policy Gradient]{RL: Policy Search}
\subtitle{Step Size and Trust Region}


\begin{document}
	
	\maketitle

%-----------------------------------------------------------------------
%----------------------------------------------------------------------
\begin{frame}[c]{Fixing Policy Gradient III: Limiting Update Size}
\begin{itemize}
    \item We can approximate the return for the gradient computation using samples
    \item We can stabilize that return to a degree using value estimation
    \item Important: value estimation does not solve the variance problem; it can only reduce variance
    \item It's still possible to sample very good or bad trajectories and thus perform extreme updates
\end{itemize}	
\bigskip
$\leadsto$ The policy can still change drastically with only one update
\end{frame}

%-----------------------------------------------------------------------
%----------------------------------------------------------------------
\begin{frame}[c]{Why are Step Sizes a Big Deal in RL?}
	
    \begin{itemize}
        \item Step size (aka learning rate) is important in any gradient-based approach
        \item Supervised learning: Step too far $\leadsto$ next updates might fix it
        \item Reinforcement learning:
        \begin{itemize}
            \item Step too far $\leadsto$ bad policy
            \item Next batch: collected under bad policy
            \item \alert{Policy is determining data collection!} 
            \begin{itemize}
                \item Essentially controlling exploration and exploitation trade-off due to particular policy parameters and the stochasticity of the policy
            \end{itemize}
            \item May not be able to recover from a bad choice $\leadsto$ collapse in performance
        \end{itemize}
    \end{itemize}

\end{frame}

%----------------------------------------------------------------------
%----------------------------------------------------------------------
\begin{frame}[c]{Policy Gradient and Step Size}
	
    \begin{itemize}
        \item Goal: Each step of policy gradient yields an updated policy $\pi'$ whose value is greater than\\ (or equal) to the prior policy $\pi$: $V^{\pi'} \geq V^\pi$
        \begin{itemize}
            \item Monotonic improvement
            \item Important in some applications
            \item PG often more stable than DQN
        \end{itemize} 
        \item Gradient descent approaches update the weights a small step in direction of the gradient
        \item First order / linear approximation of the value function's dependence on the policy parameterization
        \item Locally a good approximation; further away less good
    \end{itemize}

\end{frame}

%-----------------------------------------------------------------------
%----------------------------------------------------------------------
\begin{frame}[c]{Policy Gradient Methods with Auto-Step-Size Selection}
	
    \begin{itemize}
        \item Can we automatically ensure the updated policy $\pi'$ has a value greater than (or equal to) the prior policy $V^{\pi'} \geq V^\pi$?
        \item We will:
        \begin{itemize}
            \item Consider this question for the policy gradient setting
            \item Try to address this by modifying step size
            \item Introduce an algorithm implementing such an auto-adaption approach
        \end{itemize}
    \end{itemize}

\end{frame}
%-----------------------------------------------------------------------
%----------------------------------------------------------------------
\begin{frame}[c]{Objective Function}
	
    \begin{itemize}
        \item Goal: find policy parameters that maximize the value function
        $$ V(\theta) = \mathbb{E}_{\pi_\theta} \left[ \sum_{t=0}^{\infty} \gamma^t r(s_t, a_t) \right]$$
        \item Having access to samples from the current policy $\pi_\theta$ 
        \item But what if we want to predict the value of a different policy ($\leadsto$ off-policy learning)?
    \end{itemize}

\end{frame}
%-----------------------------------------------------------------------
%----------------------------------------------------------------------
\begin{frame}[c]{Objective Function}
	\vspace{-1em}
    \begin{itemize}
        \item Goal: find policy parameters that maximize value function
        $$ V(\theta) = \mathbb{E}_{\pi_\theta} \left[ \sum_{t=0}^{\infty} \gamma^t r(s_t, a_t)\right]$$
        \item Express value of a second policy $\Tilde{\pi}$ in terms of advantage of original policy $\pi$
        \begin{eqnarray}
        V(\Tilde{\theta}) &=& V(\theta) + \mathbb{E}_{\pi_{\Tilde{\theta}}} \left[ \sum_{t=0}^{\infty} \gamma^t A_\pi(s_t, a_t) \right] \nonumber\\
        &=& V(\theta) + \sum_{s} \mu_{\Tilde{\pi}}(s) \sum_{a} \Tilde{\pi}(a \mid s) A_\pi(s_t, a_t) \nonumber
        \end{eqnarray}
        %\item $ \mu_{\Tilde{\pi}}(s)$ is the discounted weighted frequency of state $s$ under policy $\Tilde{\pi}$
        \item Assume for now that we know the true advantage $A_\pi$ and $\Tilde{\pi}$
        \item But we cannot compute the above if we don't know $\mu(\Tilde{\pi})$,\\ i.e., the state distribution under the newly proposed policy (a.k.a. state visitation frequency)
    \end{itemize}

\end{frame}
%-----------------------------------------------------------------------
%----------------------------------------------------------------------
\begin{frame}[c]{Local Approximation}
	
    \begin{itemize}
        \item Can we remove the dependency on the discounted visitation frequencies under the new policy?
        \item Substitute in the state distribution under the \alert{current} policy to define a new objective function to \alert{maximize}:
        $$ L_\pi(\Tilde{\pi}) = V(\theta) + \sum_{s} \alert{\mu_\pi (s)} \sum_{a} \Tilde{\pi}(a \mid s) A_\pi(s,a)$$
        \item Note that $L_{\pi_{\theta}}(\pi_{\theta}) = V(\theta)$ since the unweighted sum of $A_\pi$ is zero
        \item Thus, the gradient of $L$ is identical to the gradient of the value function at policy parameterized by $\theta_0$: 
        $$\nabla_{\theta} L_{\pi_{\theta'}}(\pi_{\theta'})\mid_{\theta = \theta'} = \nabla_\theta V(\theta)\mid_{\theta = \theta'}$$
    \end{itemize}

\end{frame}
%-----------------------------------------------------------------------
%----------------------------------------------------------------------

\end{document}
