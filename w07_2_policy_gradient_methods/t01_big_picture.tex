% !TeX spellcheck = en_US
\documentclass[aspectratio=169]{../latex_main/tntbeamer}  % you can pass all options of the beamer class, e.g., 'handout' or 'aspectratio=43'
\usepackage{dsfont}
\usepackage{bm}
\usepackage[english]{babel}
\usepackage[T1]{fontenc}
%\usepackage[utf8]{inputenc}
\usepackage{graphicx}
\graphicspath{ {./figures/} }
\usepackage{algorithm}
\usepackage[ruled,vlined,algo2e,linesnumbered]{algorithm2e}
\usepackage{hyperref}
\usepackage{booktabs}
\usepackage{mathtools}

\usepackage{amsmath,amssymb}
\usepackage{latexsym}

\DeclareMathOperator*{\argmax}{arg\,max}
\DeclareMathOperator*{\argmin}{arg\,min}

\usepackage{pgfplots}
\pgfplotsset{compat=1.16}
\usepackage{tikz}
\usetikzlibrary{trees} 
\usetikzlibrary{shapes.geometric}
\usetikzlibrary{positioning,shapes,shadows,arrows,calc,mindmap}
\usetikzlibrary{positioning,fadings,through}
\usetikzlibrary{decorations.pathreplacing}
\usetikzlibrary{intersections}
\pgfdeclarelayer{background}
\pgfdeclarelayer{foreground}
\pgfsetlayers{background,main,foreground}
\tikzstyle{activity}=[rectangle, draw=black, rounded corners, text centered, text width=8em]
\tikzstyle{data}=[rectangle, draw=black, text centered, text width=8em]
\tikzstyle{myarrow}=[->, thick, draw=black]

% Define the layers to draw the diagram
\pgfdeclarelayer{background}
\pgfdeclarelayer{foreground}
\pgfsetlayers{background,main,foreground}

\input{./latex_main_old/macros}

% Requires XeLaTeX or LuaLaTeX
\usepackage{unicode-math}

\usepackage{fontspec}
%\setsansfont{Arial}
\setsansfont{RotisSansSerifStd}[ 
Path=./latex_main/fonts/,
Extension = .otf,
UprightFont = *-Regular,  % or *-Light
BoldFont = *-ExtraBold,  % or *-Bold
ItalicFont = *-Italic
]
\setmonofont{Cascadia Mono}[
Scale=0.8
]

% scale factor adapted; mathrm font added (Benjamin Spitschan @TNT, 2021-06-01)
%\setmathfont[Scale=1.05]{Libertinus Math}
%\setmathrm[Scale=1.05]{Libertinus Math}

% other available math fonts are (not exhaustive)
% Latin Modern Math
% XITS Math
% Libertinus Math
% Asana Math
% Fira Math
% TeX Gyre Pagella Math
% TeX Gyre Bonum Math
% TeX Gyre Schola Math
% TeX Gyre Termes Math

% Literature References
% #1 = Display Name
% #2 = Url (without \href)
\newcommand{\lit}[2]{\href{#2}{\footnotesize\color{black!60}[#1]}}

%%% Beamer Customization
%----------------------------------------------------------------------
% (Don't) Show sections in frame header. Options: 'sections', 'sections light', empty
\setbeamertemplate{headline}{empty}

% Add header logo for normal frames
\setheaderimage{
	% \includegraphics[height=\logoheight]{figures/TNT_darkv4.pdf}
	\includegraphics[height=\logoheight]{./latex_main/figures/luh_logo_rgb_0_80_155.pdf}
	% \includegraphics[height=\logoheight]{figures/logo_tntluh.pdf}
}

% Header logo for title page
\settitleheaderimage{
	% \includegraphics[height=\logoheight]{figures/TNT_darkv4.pdf}
	\includegraphics[height=\logoheight]{./latex_main/figures/luh_logo_rgb_0_80_155.pdf}
	% \includegraphics[height=\logoheight]{figures/logo_tntluh.pdf}
}

% Title page: tntdefault 
\setbeamertemplate{title page}[tntdefault]  % or luhstyle
% Add optional title image here
%\addtitlepageimagedefault{\includegraphics[width=0.65\textwidth]{figures/luh_default_presentation_title_image.jpg}}

% Title page: luhstyle
% \setbeamertemplate{title page}[luhstyle]
% % Add optional title image here
% \addtitlepageimage{\includegraphics[width=0.75\textwidth]{figures/luh_default_presentation_title_image.jpg}}

\author[Lindauer]{Marius Lindauer\\[1em]
	\includegraphics[height=\logoheight]{./latex_main/figures/luh_logo_rgb_0_80_155.pdf}\qquad
\includegraphics[height=\logoheight]{./latex_main/figures/TNT_darkv4}\qquad
\includegraphics[height=\logoheight]{./latex_main/figures/L3S.jpg}	}
\date{Winter Term 2021
}


%%% Custom Packages
%----------------------------------------------------------------------
% Create dummy content
\usepackage{blindtext}

% Adds a frame with the current page layout. Just call \layout inside of a frame.
\usepackage{layout}

\title[RL: Big Picture]{RL: Policy Gradient Methods}
\subtitle{The Big Picture}



\begin{document}
	
	\maketitle

%----------------------------------------------------------------------
%----------------------------------------------------------------------
\begin{frame}[c]{Policy Gradient Methods}

\begin{itemize}
	\item Before, we showed how to find a good policy without value approximation\\ by directly optimizing the policy gradient
	\item But:
        \begin{itemize}
            \item scalability of black-box optimizers might be limited
            \item the finite difference method is noisy and inefficient 
        \end{itemize}
	\item We showed the policy gradient can be computed analytically\\
        $\leadsto$ How exactly can we leverage this in RL algorithms?
\end{itemize}

\end{frame}
%-----------------------------------------------------------------------
%----------------------------------------------------------------------
\begin{frame}[c]{Recap: Policy Gradient Theorem}

	\begin{itemize}
		\item For any differentiable policy $\pi_\theta$ and objective function $J$, the policy gradient is $$\nabla_\theta J_\theta= \mathbb{E}_{\pi_\theta} [Q^{\pi_\theta}(s,a) \nabla_\theta \log \pi_\theta(s,a) ] $$
		\item To compute this gradient, we only need trajectories from $\pi_\theta$ and its Q-function $Q^{\pi_\theta}$
        \item A trajectory is a tuple $\tau = (s_0, a_0, r_1, \ldots, s_{T-1}, a_{T-1}, r_{T-1}, s_T)$
	\end{itemize}		
\bigskip
	$\leadsto$ The data we collect through environment interactions is enough to compute the gradient
    \newline
    \textbf{Problem}: the gradient estimate is unbiased but very noisy (i.e., high variance)
\end{frame}
%-----------------------------------------------------------------------
%-----------------------------------------------------------------------
\begin{frame}[c]{A Simple Policy Gradient Update}
\begin{enumerate}
	\item for each episode $\{s_0, a_0, r_1, \ldots, s_{T-1}, a_{T-1}, r_T \} \sim \pi_\theta $ do
	\begin{itemize}
		\item for $t=0$ to $T - 1$ do
		\begin{itemize}
			\item $\theta := \theta + \alpha Q^{\pi_\theta}(s,a) \nabla_\theta \log \pi_\theta (s_t, a_t)  $ 
		\end{itemize}
	\end{itemize}
	\item return $\theta$
\end{enumerate}

\medskip
Note: This is gradient ascent ($+$) and not gradient descent

\pause
\medskip
\textbf{Questions:} How do we estimate $Q^{\pi_\theta}(s,a)$ efficiently? Can we make the gradient estimation less noisy?
\end{frame}
%-----------------------------------------------------------------------
%-----------------------------------------------------------------------
\begin{frame}[c]{Policy Gradient Algorithms}
\begin{itemize}
    \item We can construct algorithms using the policy gradient to update a policy function iteratively
    \item The policy can be modeled in different ways; though DNNs are the most common
    \item We can use algorithmic improvements to stabilize the gradient estimation
    \pause
    \item In this module, we will introduce some of them:
    \begin{itemize}
        \item Estimation via sampling: REINFORCE
        \item Utilizing value baselines: Actor-Critic
        \item Limiting the size of each policy update: PPO
    \end{itemize}
\end{itemize}
\end{frame}
%-----------------------------------------------------------------------
%-----------------------------------------------------------------------
\end{document}
