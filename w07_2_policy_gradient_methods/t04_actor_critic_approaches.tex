% !TeX spellcheck = en_US
\documentclass[aspectratio=169]{../latex_main/tntbeamer}  % you can pass all options of the beamer class, e.g., 'handout' or 'aspectratio=43'
\usepackage{dsfont}
\usepackage{bm}
\usepackage[english]{babel}
\usepackage[T1]{fontenc}
%\usepackage[utf8]{inputenc}
\usepackage{graphicx}
\graphicspath{ {./figures/} }
\usepackage{algorithm}
\usepackage[ruled,vlined,algo2e,linesnumbered]{algorithm2e}
\usepackage{hyperref}
\usepackage{booktabs}
\usepackage{mathtools}

\usepackage{amsmath,amssymb}
\usepackage{latexsym}

\DeclareMathOperator*{\argmax}{arg\,max}
\DeclareMathOperator*{\argmin}{arg\,min}

\usepackage{pgfplots}
\pgfplotsset{compat=1.16}
\usepackage{tikz}
\usetikzlibrary{trees} 
\usetikzlibrary{shapes.geometric}
\usetikzlibrary{positioning,shapes,shadows,arrows,calc,mindmap}
\usetikzlibrary{positioning,fadings,through}
\usetikzlibrary{decorations.pathreplacing}
\usetikzlibrary{intersections}
\pgfdeclarelayer{background}
\pgfdeclarelayer{foreground}
\pgfsetlayers{background,main,foreground}
\tikzstyle{activity}=[rectangle, draw=black, rounded corners, text centered, text width=8em]
\tikzstyle{data}=[rectangle, draw=black, text centered, text width=8em]
\tikzstyle{myarrow}=[->, thick, draw=black]

% Define the layers to draw the diagram
\pgfdeclarelayer{background}
\pgfdeclarelayer{foreground}
\pgfsetlayers{background,main,foreground}

\input{./latex_main_old/macros}

% Requires XeLaTeX or LuaLaTeX
\usepackage{unicode-math}

\usepackage{fontspec}
%\setsansfont{Arial}
\setsansfont{RotisSansSerifStd}[ 
Path=./latex_main/fonts/,
Extension = .otf,
UprightFont = *-Regular,  % or *-Light
BoldFont = *-ExtraBold,  % or *-Bold
ItalicFont = *-Italic
]
\setmonofont{Cascadia Mono}[
Scale=0.8
]

% scale factor adapted; mathrm font added (Benjamin Spitschan @TNT, 2021-06-01)
%\setmathfont[Scale=1.05]{Libertinus Math}
%\setmathrm[Scale=1.05]{Libertinus Math}

% other available math fonts are (not exhaustive)
% Latin Modern Math
% XITS Math
% Libertinus Math
% Asana Math
% Fira Math
% TeX Gyre Pagella Math
% TeX Gyre Bonum Math
% TeX Gyre Schola Math
% TeX Gyre Termes Math

% Literature References
% #1 = Display Name
% #2 = Url (without \href)
\newcommand{\lit}[2]{\href{#2}{\footnotesize\color{black!60}[#1]}}

%%% Beamer Customization
%----------------------------------------------------------------------
% (Don't) Show sections in frame header. Options: 'sections', 'sections light', empty
\setbeamertemplate{headline}{empty}

% Add header logo for normal frames
\setheaderimage{
	% \includegraphics[height=\logoheight]{figures/TNT_darkv4.pdf}
	\includegraphics[height=\logoheight]{./latex_main/figures/luh_logo_rgb_0_80_155.pdf}
	% \includegraphics[height=\logoheight]{figures/logo_tntluh.pdf}
}

% Header logo for title page
\settitleheaderimage{
	% \includegraphics[height=\logoheight]{figures/TNT_darkv4.pdf}
	\includegraphics[height=\logoheight]{./latex_main/figures/luh_logo_rgb_0_80_155.pdf}
	% \includegraphics[height=\logoheight]{figures/logo_tntluh.pdf}
}

% Title page: tntdefault 
\setbeamertemplate{title page}[tntdefault]  % or luhstyle
% Add optional title image here
%\addtitlepageimagedefault{\includegraphics[width=0.65\textwidth]{figures/luh_default_presentation_title_image.jpg}}

% Title page: luhstyle
% \setbeamertemplate{title page}[luhstyle]
% % Add optional title image here
% \addtitlepageimage{\includegraphics[width=0.75\textwidth]{figures/luh_default_presentation_title_image.jpg}}

\author[Lindauer]{Marius Lindauer\\[1em]
	\includegraphics[height=\logoheight]{./latex_main/figures/luh_logo_rgb_0_80_155.pdf}\qquad
\includegraphics[height=\logoheight]{./latex_main/figures/TNT_darkv4}\qquad
\includegraphics[height=\logoheight]{./latex_main/figures/L3S.jpg}	}
\date{Winter Term 2021
}


%%% Custom Packages
%----------------------------------------------------------------------
% Create dummy content
\usepackage{blindtext}

% Adds a frame with the current page layout. Just call \layout inside of a frame.
\usepackage{layout}

\title[RL: Policy Gradient]{RL: Policy Search}
\subtitle{Policy Gradient Algorithms: Actor-Critic}



\begin{document}
	
	\maketitle
%-----------------------------------------------------------------------
%----------------------------------------------------------------------
\begin{frame}[c]{Incorporating Baselines: Actor-Critic Methods}

\begin{itemize}
	\item We know the near-optimal baseline is:
 $$b(s_t) \approx \mathbb{E} [r_t + r_{t+1} +\ldots + r_{T-1}] = G_t$$
    \item That should look very familiar since we know: 
    $$ V(s)  = \mathbb{E}[G_t \mid s_t=s] $$
\end{itemize}
$\leadsto$ We can use value estimation to approximate the optimal baseline
\end{frame}
%-----------------------------------------------------------------------
% %----------------------------------------------------------------------
\begin{frame}[c]{Choosing the Baseline: Value Functions}
	
	\begin{itemize}
		\item Recall $Q$-function:
		$$Q^\pi(s,a) = \mathbb{E}_\pi [r_0 +  \gamma r_1 + \gamma^2 r_2 + \ldots \mid s_0 = s, a_0 = a ]$$
		\item State-value function can serve as a great baseline since it's often easier to estimate:
		\begin{eqnarray}
		V^\pi (s) &=& \mathbb{E}_\pi [r_0 +  \gamma r_1 + \gamma^2 r_2 + \ldots \mid s_0 = s]\nonumber\\	
		&=& \mathbb{E}_{a\sim\pi} [Q^\pi(s,a)]\nonumber
		\end{eqnarray}
		\item Advantage function: estimating the advantage of a given action over the state value:
		$$A^\pi(s,a) = Q^\pi(s,a) - V^\pi(s) $$
	\end{itemize}
	
\end{frame}
%-----------------------------------------------------------------------
% %----------------------------------------------------------------------
\begin{frame}[c]{”Vanilla” Policy Gradient Algorithm With Advantage Baseline}
	
	\begin{itemize}
		\item Initialize policy parameters $\theta$ and baseline $b$
		\item for iteration$=1,2,\ldots$ do
		\begin{itemize}
			\item Collect a set of $K$ trajectories by executing the current policy
			\item At each time step $t$ in each trajectory $\tau^k$, compute
			\begin{itemize}
				\item Return $G_t^k= \sum_{t'=t}^{T-1} r_{t'}^k$ and
				\item Advantage estimate (actual return - general return estimate) $\hat{A}^k_t = G^k_t - b(s^k_t) = G^k_t - G^t$ 
			\end{itemize}
			\item Re-fit the baseline by minimizing $b := \sum_k \sum_t || b(s^k_t) - G^k_t||^2$ (MSE)
			\item Update the policy, using a policy gradient estimate $\hat{g}$
			\begin{itemize}
				\item which is a sum over k: $\hat{A}_t \nabla_\theta \log \pi(a^k_t \mid s^k_t; \theta) $
				\item Apply gradient $\hat{g}$ by any DL-optimizer (e.g., SGD or ADAM)
			\end{itemize}
		\end{itemize}
		
	\end{itemize}
 
\end{frame}

\begin{frame}[c]{Actor-Critic Methods}
	
	\begin{itemize}
		\item Methods combining policy approximation (an "Actor") with value approximation (a "Critic") are called "Actor-Critic" approaches
        \item Many different configurations depending on the exact combination of policy and value learning methods -- previous slide shows a basic version
        \item Most modern policy gradient algorithms are Actor-Critic algorithms
        \item A big difference to REINFORCE: we now update over a set of trajectories at once instead of updating at each step
	\end{itemize}
	\bigskip
 $\leadsto$ More stable, highly parallelizable policy gradient algorithms
\end{frame}

% %-----------------------------------------------------------------------
%-----------------------------------------------------------------------
\end{document}