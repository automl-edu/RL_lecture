% !TeX spellcheck = en_US
\documentclass[aspectratio=169]{../latex_main/tntbeamer}  % you can pass all options of the beamer class, e.g., 'handout' or 'aspectratio=43'
\usepackage{dsfont}
\usepackage{bm}
\usepackage[english]{babel}
\usepackage[T1]{fontenc}
%\usepackage[utf8]{inputenc}
\usepackage{graphicx}
\graphicspath{ {./figures/} }
\usepackage{algorithm}
\usepackage[ruled,vlined,algo2e,linesnumbered]{algorithm2e}
\usepackage{hyperref}
\usepackage{booktabs}
\usepackage{mathtools}

\usepackage{amsmath,amssymb}
\usepackage{latexsym}

\DeclareMathOperator*{\argmax}{arg\,max}
\DeclareMathOperator*{\argmin}{arg\,min}

\usepackage{pgfplots}
\pgfplotsset{compat=1.16}
\usepackage{tikz}
\usetikzlibrary{trees} 
\usetikzlibrary{shapes.geometric}
\usetikzlibrary{positioning,shapes,shadows,arrows,calc,mindmap}
\usetikzlibrary{positioning,fadings,through}
\usetikzlibrary{decorations.pathreplacing}
\usetikzlibrary{intersections}
\pgfdeclarelayer{background}
\pgfdeclarelayer{foreground}
\pgfsetlayers{background,main,foreground}
\tikzstyle{activity}=[rectangle, draw=black, rounded corners, text centered, text width=8em]
\tikzstyle{data}=[rectangle, draw=black, text centered, text width=8em]
\tikzstyle{myarrow}=[->, thick, draw=black]

% Define the layers to draw the diagram
\pgfdeclarelayer{background}
\pgfdeclarelayer{foreground}
\pgfsetlayers{background,main,foreground}

\input{./latex_main_old/macros}

% Requires XeLaTeX or LuaLaTeX
\usepackage{unicode-math}

\usepackage{fontspec}
%\setsansfont{Arial}
\setsansfont{RotisSansSerifStd}[ 
Path=./latex_main/fonts/,
Extension = .otf,
UprightFont = *-Regular,  % or *-Light
BoldFont = *-ExtraBold,  % or *-Bold
ItalicFont = *-Italic
]
\setmonofont{Cascadia Mono}[
Scale=0.8
]

% scale factor adapted; mathrm font added (Benjamin Spitschan @TNT, 2021-06-01)
%\setmathfont[Scale=1.05]{Libertinus Math}
%\setmathrm[Scale=1.05]{Libertinus Math}

% other available math fonts are (not exhaustive)
% Latin Modern Math
% XITS Math
% Libertinus Math
% Asana Math
% Fira Math
% TeX Gyre Pagella Math
% TeX Gyre Bonum Math
% TeX Gyre Schola Math
% TeX Gyre Termes Math

% Literature References
% #1 = Display Name
% #2 = Url (without \href)
\newcommand{\lit}[2]{\href{#2}{\footnotesize\color{black!60}[#1]}}

%%% Beamer Customization
%----------------------------------------------------------------------
% (Don't) Show sections in frame header. Options: 'sections', 'sections light', empty
\setbeamertemplate{headline}{empty}

% Add header logo for normal frames
\setheaderimage{
	% \includegraphics[height=\logoheight]{figures/TNT_darkv4.pdf}
	\includegraphics[height=\logoheight]{./latex_main/figures/luh_logo_rgb_0_80_155.pdf}
	% \includegraphics[height=\logoheight]{figures/logo_tntluh.pdf}
}

% Header logo for title page
\settitleheaderimage{
	% \includegraphics[height=\logoheight]{figures/TNT_darkv4.pdf}
	\includegraphics[height=\logoheight]{./latex_main/figures/luh_logo_rgb_0_80_155.pdf}
	% \includegraphics[height=\logoheight]{figures/logo_tntluh.pdf}
}

% Title page: tntdefault 
\setbeamertemplate{title page}[tntdefault]  % or luhstyle
% Add optional title image here
%\addtitlepageimagedefault{\includegraphics[width=0.65\textwidth]{figures/luh_default_presentation_title_image.jpg}}

% Title page: luhstyle
% \setbeamertemplate{title page}[luhstyle]
% % Add optional title image here
% \addtitlepageimage{\includegraphics[width=0.75\textwidth]{figures/luh_default_presentation_title_image.jpg}}

\author[Lindauer]{Marius Lindauer\\[1em]
	\includegraphics[height=\logoheight]{./latex_main/figures/luh_logo_rgb_0_80_155.pdf}\qquad
\includegraphics[height=\logoheight]{./latex_main/figures/TNT_darkv4}\qquad
\includegraphics[height=\logoheight]{./latex_main/figures/L3S.jpg}	}
\date{Winter Term 2021
}


%%% Custom Packages
%----------------------------------------------------------------------
% Create dummy content
\usepackage{blindtext}

% Adds a frame with the current page layout. Just call \layout inside of a frame.
\usepackage{layout}

\title[RL: Policy Gradient]{RL: Policy Search}
\subtitle{Step Size and Trust Region: PPO}


\begin{document}
	
	\maketitle

%-----------------------------------------------------------------------
%----------------------------------------------------------------------
\begin{frame}{The State Of The Art in Policy Gradient Methods}
We combine all improvements from this lecture into one algorithm:
    \begin{itemize}
        \item Policy gradient estimation from Monte Carlo samples
        \item Actor-Critic algorithm with value estimation
        \item Limited updates to improve continuously
    \end{itemize}
\end{frame}
%-----------------------------------------------------------------------
%----------------------------------------------------------------------
\begin{frame}[c]{Common Template of Policy Gradient Algorithms}

	\begin{itemize}
	    \item For $i= 1,2,\ldots$ do
	    \begin{enumerate}
	        \item Run policy for $T$ timesteps or $N$ trajectories
	        \item At each timestep in each trajectory, compute baseline $b(s_t)$
	        \item Compute surrogate objective $L$
	        \item Update the policy to optimize $L$ $\leadsto$ constrain the update to ensure policy improvement
	    \end{enumerate}
	\end{itemize}

\end{frame}
%-----------------------------------------------------------------------
%----------------------------------------------------------------------
\begin{frame}[c]{The Current Standard: PPO Algorithm~\lit{Schulman et al. 2017}{https://arxiv.org/pdf/1707.06347.pdf}}

	\begin{itemize}
	    \item For $i= 1,2,\ldots$ do
	    \begin{enumerate}
	        \item Run policy for $T$ timesteps or $N$ trajectories
            \item At each timestep in each trajectory, \alert{compute advantage using value function $V$}
	        \item Compute surrogate objective $L$
            \item \alert{Clip $L$} for large updates 
            \item Add the \alert{value loss} term to $L$ 
            \item Add a term to $L$ that provides a \alert{bonus for entropy}
            \item Update the policy to optimize $L$
            \item \alert{Update target value function $V^{target}$}
	    \end{enumerate}
	\end{itemize}

\end{frame}
%-----------------------------------------------------------------------
%----------------------------------------------------------------------
\begin{frame}[c]{Update I: The CLIP Objective}

	\begin{itemize}
	    \item Recall our surrogate objective $L$:
     $$\mathbb{E}_{s\sim \mu_{\pi_{old}}, a \sim \pi_{old}} \left[ \frac{\pi_{new}(a \mid s_n)}{\pi_{old}(a\mid s_n)} A_{\pi_{old}} (s_n, a) \right]$$
        \item We want to ensure similarity between the old and new policy and thus improvement
        \item Instead of trying to approximate the distance between the old and new policy explicitly with a KL divergence (which can be hard to compute), we simply clip the importance sampling term:
        $$L = L^{CLIP} = \mathbb{E}_{s\sim \mu_{\pi_{old}}, a \sim \pi_{old}} \left[ min( r(\pi_{new}) A_{\pi_{old}} (s_n, a), clip(r(\pi_{new}), 1-\epsilon, 1+\epsilon) A_{\pi_{old}} ) \right]$$
        with $r(\pi) = \frac{\pi_new(a \mid s_n)}{\pi_{old}(a\mid s_n)}$
        \item This penalizes large changes away from the old policy by limiting the advantage from them
	\end{itemize}

\end{frame}
%-----------------------------------------------------------------------
%----------------------------------------------------------------------
\begin{frame}[c]{Update II: The Value Error}

\begin{columns}

    \column{0.3\textwidth}

\centering
        \includegraphics[scale=0.2]{w07_2_policy_gradient_methods/images/ac-arch.pdf}

    \column{0.7\textwidth}

\begin{itemize}
	    \item A second change to our PG template is the addition of a value error loss term for learning $V$
        \item This term is updating the value function $V$ (used for advantage estimation)
        \item This is part of the overall objective since in practice a single network predicts policy and values
        \item We add the hyperparameter $c_1$ to weigh it against the policy improvement term
        \item Our updated objective is:
        $$L = L^{CLIP} + c_1  \mathbb{E}_{s\sim \mu_{\pi_{old}}, a \sim \pi_{old}} \left[ (V(s_n) - V^{target}(s_n))^2 \right] $$
        $$= L^{CLIP} + c_1 L^V$$
	\end{itemize}

\end{columns}

        
	



\end{frame}
%-----------------------------------------------------------------------
%----------------------------------------------------------------------
\begin{frame}[c]{Update III: The Entropy Bonus}

	\begin{itemize}
	    \item The entropy bonus is added to ensure sufficient exploration ($\to$ maximize entropy!)
        \item It measures the entropy $H$ of our current policy across the states of our sampled trajectories
        \item We introduce the hyperparameter $c_2$ to weigh it against the other loss terms
        \item Now we have the full PPO objective:
        $$L^{PPO} = L^{CLIP} + c_1 L^V + c_2 \mathbb{E}_{s\sim \mu_{\pi_{old}}, a \sim \pi_{old}} \left[ H(\pi_{old}, s_n) \right] $$
        $$= L^{CLIP} + c_1 L^V + c_2 L^{ENT}$$
	\end{itemize}

\end{frame}
%-----------------------------------------------------------------------
%----------------------------------------------------------------------
\begin{frame}[c]{Policy Gradient Summary}

\begin{itemize}
    \item Extremely popular and useful set of approaches
    \item Can input prior knowledge in the form of specifying policy parameterization
    \item You should be familiar with REINFORCE and the policy gradient template
    \item Understand where different estimators can be slotted in 
    \item (You don't have to be able to derive or remember the specific formulas in PPO for approximating the objectives and constraints)
\end{itemize}

\end{frame}
%-----------------------------------------------------------------------
%-----------------------------------------------------------------------
\end{document}