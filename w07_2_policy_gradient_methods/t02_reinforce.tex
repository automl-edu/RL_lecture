% !TeX spellcheck = en_US
\documentclass[aspectratio=169]{../latex_main/tntbeamer}  % you can pass all options of the beamer class, e.g., 'handout' or 'aspectratio=43'
\usepackage{dsfont}
\usepackage{bm}
\usepackage[english]{babel}
\usepackage[T1]{fontenc}
%\usepackage[utf8]{inputenc}
\usepackage{graphicx}
\graphicspath{ {./figures/} }
\usepackage{algorithm}
\usepackage[ruled,vlined,algo2e,linesnumbered]{algorithm2e}
\usepackage{hyperref}
\usepackage{booktabs}
\usepackage{mathtools}

\usepackage{amsmath,amssymb}
\usepackage{latexsym}

\DeclareMathOperator*{\argmax}{arg\,max}
\DeclareMathOperator*{\argmin}{arg\,min}

\usepackage{pgfplots}
\pgfplotsset{compat=1.16}
\usepackage{tikz}
\usetikzlibrary{trees} 
\usetikzlibrary{shapes.geometric}
\usetikzlibrary{positioning,shapes,shadows,arrows,calc,mindmap}
\usetikzlibrary{positioning,fadings,through}
\usetikzlibrary{decorations.pathreplacing}
\usetikzlibrary{intersections}
\pgfdeclarelayer{background}
\pgfdeclarelayer{foreground}
\pgfsetlayers{background,main,foreground}
\tikzstyle{activity}=[rectangle, draw=black, rounded corners, text centered, text width=8em]
\tikzstyle{data}=[rectangle, draw=black, text centered, text width=8em]
\tikzstyle{myarrow}=[->, thick, draw=black]

% Define the layers to draw the diagram
\pgfdeclarelayer{background}
\pgfdeclarelayer{foreground}
\pgfsetlayers{background,main,foreground}

\input{./latex_main_old/macros}

% Requires XeLaTeX or LuaLaTeX
\usepackage{unicode-math}

\usepackage{fontspec}
%\setsansfont{Arial}
\setsansfont{RotisSansSerifStd}[ 
Path=./latex_main/fonts/,
Extension = .otf,
UprightFont = *-Regular,  % or *-Light
BoldFont = *-ExtraBold,  % or *-Bold
ItalicFont = *-Italic
]
\setmonofont{Cascadia Mono}[
Scale=0.8
]

% scale factor adapted; mathrm font added (Benjamin Spitschan @TNT, 2021-06-01)
%\setmathfont[Scale=1.05]{Libertinus Math}
%\setmathrm[Scale=1.05]{Libertinus Math}

% other available math fonts are (not exhaustive)
% Latin Modern Math
% XITS Math
% Libertinus Math
% Asana Math
% Fira Math
% TeX Gyre Pagella Math
% TeX Gyre Bonum Math
% TeX Gyre Schola Math
% TeX Gyre Termes Math

% Literature References
% #1 = Display Name
% #2 = Url (without \href)
\newcommand{\lit}[2]{\href{#2}{\footnotesize\color{black!60}[#1]}}

%%% Beamer Customization
%----------------------------------------------------------------------
% (Don't) Show sections in frame header. Options: 'sections', 'sections light', empty
\setbeamertemplate{headline}{empty}

% Add header logo for normal frames
\setheaderimage{
	% \includegraphics[height=\logoheight]{figures/TNT_darkv4.pdf}
	\includegraphics[height=\logoheight]{./latex_main/figures/luh_logo_rgb_0_80_155.pdf}
	% \includegraphics[height=\logoheight]{figures/logo_tntluh.pdf}
}

% Header logo for title page
\settitleheaderimage{
	% \includegraphics[height=\logoheight]{figures/TNT_darkv4.pdf}
	\includegraphics[height=\logoheight]{./latex_main/figures/luh_logo_rgb_0_80_155.pdf}
	% \includegraphics[height=\logoheight]{figures/logo_tntluh.pdf}
}

% Title page: tntdefault 
\setbeamertemplate{title page}[tntdefault]  % or luhstyle
% Add optional title image here
%\addtitlepageimagedefault{\includegraphics[width=0.65\textwidth]{figures/luh_default_presentation_title_image.jpg}}

% Title page: luhstyle
% \setbeamertemplate{title page}[luhstyle]
% % Add optional title image here
% \addtitlepageimage{\includegraphics[width=0.75\textwidth]{figures/luh_default_presentation_title_image.jpg}}

\author[Lindauer]{Marius Lindauer\\[1em]
	\includegraphics[height=\logoheight]{./latex_main/figures/luh_logo_rgb_0_80_155.pdf}\qquad
\includegraphics[height=\logoheight]{./latex_main/figures/TNT_darkv4}\qquad
\includegraphics[height=\logoheight]{./latex_main/figures/L3S.jpg}	}
\date{Winter Term 2021
}


%%% Custom Packages
%----------------------------------------------------------------------
% Create dummy content
\usepackage{blindtext}

% Adds a frame with the current page layout. Just call \layout inside of a frame.
\usepackage{layout}

\title[RL: Policy Gradient]{RL: Policy Gradient}
\subtitle{Policy Gradient Algorithm: REINFORCE}



\begin{document}
	
	\maketitle

%-----------------------------------------------------------------------
%----------------------------------------------------------------------
\begin{frame}[c]{Fixing Policy Gradient I: Use Temporal Structure}
\vspace{-1.5em}
We know:
$$ \nabla_\theta V(\theta) = \nabla_\theta \mathbb{E}_\tau [R(\tau)] = \mathbb{E}_\tau \left[ R(\tau) \left( \sum_{t=0}^{T-1} \nabla_\theta \log \pi_\theta(a_t \mid s_t) \right) \right]$$
	
And since $R(\tau) = \sum_{t'=0}^{T-1} r_{t'}$:

$$
     \nabla_\theta \mathbb{E}[R(\tau)] = \mathbb{E} \left[  \bigg ( \sum_{t'=0}^{T-1} r_{t'} \bigg ) \bigg ( \sum^{t'}_{t=0} \nabla_\theta \log \pi_\theta (a_t \mid s_t) \bigg )  \right]
$$

We can distribute $\sum_{t'=0}^{T-1} r_{t'}$ inside the second sum to get

$$
    \nabla_\theta V(\theta) = \nabla_\theta \mathbb{E}[R(\tau)] = \mathbb{E} \bigg [ \sum_{t=0}^{T-1}  \nabla_\theta \log \pi_\theta (a_t \mid s_t) \sum^{T-1}_{t'=0} r_{t'}  \bigg ]
$$

	
\end{frame}
%-----------------------------------------------------------------------
%----------------------------------------------------------------------
\begin{frame}[c]{Policy Gradient: Use Temporal Structure}
	
\begin{itemize}
        \item Note that the return at timestep $t$ cannot impact returns at timesteps $t' < t$. So, we can change the sum $\sum_{t'=0}^{T-1} r_{t'}$ to $\sum_{t'=t}^{T-1} r_{t'}$. These are \textbf{future rewards}.

        $$
             \nabla_\theta \mathbb{E}[R(\tau)] = \mathbb{E} \bigg [ \sum_{t=0}^{T-1}  \nabla_\theta \log \pi_\theta (a_t \mid s_t) \underbrace{\sum^{T-1}_{t'=t} r_{t'} }_{\text{future rewards}} \bigg ]
        $$

        \item[$\leadsto$] If we consider less sampled rewards, the estimate will have less variance.
	\item Recall for a particular trajectory $\tau^{(i)}$, $\sum_{t'=t}^{T-1} r_{t'}^{(i)}$ is the return $G_t^{(i)}$
    \item We estimate the value of the expectation using $m$ trajectories:
\end{itemize}

$$\nabla_\theta \mathbb{E}[R(\tau)] \approx \frac{1}{m} \sum_{i=1}^m \sum_{t=0}^{T-1} \nabla_\theta \log \pi_\theta (a_t,s_t) G_t^{(i)} $$

	
\end{frame}
%-----------------------------------------------------------------------
%----------------------------------------------------------------------
\begin{frame}[c]{Monte-Carlo Policy Gradient (REINFORCE)}
	
REINFORCE is an algorithm that approximates the policy gradient using sampled trajectories in this manner:

$$ \Delta \theta_t = \alpha \nabla_\theta \log \pi_\theta (s_t, a_t) G_t $$
	
	
REINFORCE:
\begin{enumerate}
	\item Initialize policy parameters $\theta$ arbitrarily
	\item for each episode $\{s_0, a_0, r_1, \ldots, s_{T-1}, a_{T-1}, r_T \} \sim \pi_\theta $ do
	\begin{itemize}
		\item for $t=0$ to $T - 1$ do
		\begin{itemize}
			\item $\theta := \theta + \alpha \nabla_\theta \log \pi_\theta (s_t, a_t) G_t $
		\end{itemize}
	\end{itemize}
	\item return $\theta$
\end{enumerate}

\begin{itemize}
    \item Policy gradient via Monte Carlo sampling
    \item[$\leadsto$] still high variance
\end{itemize}
	
\end{frame}
%-----------------------------------------------------------------------
%----------------------------------------------------------------------
\begin{frame}[c]{Where Does The Variance Come From?}
	
\begin{itemize}
	\item Sampling trajectories will not necessarily lead to similar results from non-deterministic policies and environments
    \item Example: 3 steps in a Gridworld
\end{itemize}

 \begin{figure}
     \centering
\includegraphics[width=0.3\textwidth]{w07_2_policy_gradient_methods/images/grid1.pdf}
\includegraphics[width=0.3\textwidth]{w07_2_policy_gradient_methods/images/grid2.pdf}
\includegraphics[width=0.3\textwidth]{w07_2_policy_gradient_methods/images/grid3.pdf}
     \label{fig:enter-label}
 \end{figure}

	$\leadsto$ three very different gradient estimates
    \pause 
    \newline
    $\leadsto$ \alert{three very different updates}
\end{frame}
%-----------------------------------------------------------------------
\end{document}
