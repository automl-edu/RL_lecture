% !TeX spellcheck = en_US
\documentclass[aspectratio=169]{../latex_main/tntbeamer}  % you can pass all options of the beamer class, e.g., 'handout' or 'aspectratio=43'
\usepackage{dsfont}
\usepackage{bm}
\usepackage[english]{babel}
\usepackage[T1]{fontenc}
%\usepackage[utf8]{inputenc}
\usepackage{graphicx}
\graphicspath{ {./figures/} }
\usepackage{algorithm}
\usepackage[ruled,vlined,algo2e,linesnumbered]{algorithm2e}
\usepackage{hyperref}
\usepackage{booktabs}
\usepackage{mathtools}

\usepackage{amsmath,amssymb}
\usepackage{latexsym}

\DeclareMathOperator*{\argmax}{arg\,max}
\DeclareMathOperator*{\argmin}{arg\,min}

\usepackage{pgfplots}
\pgfplotsset{compat=1.16}
\usepackage{tikz}
\usetikzlibrary{trees} 
\usetikzlibrary{shapes.geometric}
\usetikzlibrary{positioning,shapes,shadows,arrows,calc,mindmap}
\usetikzlibrary{positioning,fadings,through}
\usetikzlibrary{decorations.pathreplacing}
\usetikzlibrary{intersections}
\pgfdeclarelayer{background}
\pgfdeclarelayer{foreground}
\pgfsetlayers{background,main,foreground}
\tikzstyle{activity}=[rectangle, draw=black, rounded corners, text centered, text width=8em]
\tikzstyle{data}=[rectangle, draw=black, text centered, text width=8em]
\tikzstyle{myarrow}=[->, thick, draw=black]

% Define the layers to draw the diagram
\pgfdeclarelayer{background}
\pgfdeclarelayer{foreground}
\pgfsetlayers{background,main,foreground}

\input{./latex_main_old/macros}

% Requires XeLaTeX or LuaLaTeX
\usepackage{unicode-math}

\usepackage{fontspec}
%\setsansfont{Arial}
\setsansfont{RotisSansSerifStd}[ 
Path=./latex_main/fonts/,
Extension = .otf,
UprightFont = *-Regular,  % or *-Light
BoldFont = *-ExtraBold,  % or *-Bold
ItalicFont = *-Italic
]
\setmonofont{Cascadia Mono}[
Scale=0.8
]

% scale factor adapted; mathrm font added (Benjamin Spitschan @TNT, 2021-06-01)
%\setmathfont[Scale=1.05]{Libertinus Math}
%\setmathrm[Scale=1.05]{Libertinus Math}

% other available math fonts are (not exhaustive)
% Latin Modern Math
% XITS Math
% Libertinus Math
% Asana Math
% Fira Math
% TeX Gyre Pagella Math
% TeX Gyre Bonum Math
% TeX Gyre Schola Math
% TeX Gyre Termes Math

% Literature References
% #1 = Display Name
% #2 = Url (without \href)
\newcommand{\lit}[2]{\href{#2}{\footnotesize\color{black!60}[#1]}}

%%% Beamer Customization
%----------------------------------------------------------------------
% (Don't) Show sections in frame header. Options: 'sections', 'sections light', empty
\setbeamertemplate{headline}{empty}

% Add header logo for normal frames
\setheaderimage{
	% \includegraphics[height=\logoheight]{figures/TNT_darkv4.pdf}
	\includegraphics[height=\logoheight]{./latex_main/figures/luh_logo_rgb_0_80_155.pdf}
	% \includegraphics[height=\logoheight]{figures/logo_tntluh.pdf}
}

% Header logo for title page
\settitleheaderimage{
	% \includegraphics[height=\logoheight]{figures/TNT_darkv4.pdf}
	\includegraphics[height=\logoheight]{./latex_main/figures/luh_logo_rgb_0_80_155.pdf}
	% \includegraphics[height=\logoheight]{figures/logo_tntluh.pdf}
}

% Title page: tntdefault 
\setbeamertemplate{title page}[tntdefault]  % or luhstyle
% Add optional title image here
%\addtitlepageimagedefault{\includegraphics[width=0.65\textwidth]{figures/luh_default_presentation_title_image.jpg}}

% Title page: luhstyle
% \setbeamertemplate{title page}[luhstyle]
% % Add optional title image here
% \addtitlepageimage{\includegraphics[width=0.75\textwidth]{figures/luh_default_presentation_title_image.jpg}}

\author[Lindauer]{Marius Lindauer\\[1em]
	\includegraphics[height=\logoheight]{./latex_main/figures/luh_logo_rgb_0_80_155.pdf}\qquad
\includegraphics[height=\logoheight]{./latex_main/figures/TNT_darkv4}\qquad
\includegraphics[height=\logoheight]{./latex_main/figures/L3S.jpg}	}
\date{Winter Term 2021
}


%%% Custom Packages
%----------------------------------------------------------------------
% Create dummy content
\usepackage{blindtext}

% Adds a frame with the current page layout. Just call \layout inside of a frame.
\usepackage{layout}

\title[RL: Policy Gradient]{RL: Policy Search}
\subtitle{Step Size and Trust Region}


\begin{document}
	
	\maketitle

%-----------------------------------------------------------------------
%----------------------------------------------------------------------
\begin{frame}[c]{Optimization of Parameterized Policies}
	
    \begin{itemize}
        \item Goal is to optimize
        $$ L_{\pi_{old}} (\pi_{new}) -  \frac{4\epsilon\gamma}{(1-\gamma)^2} D_{KL}^{\max}(\pi_{old}, \pi_{new}) = L_{\pi_{old}} (\pi_{new}) -  C \cdot D_{KL}^{\max}(\pi_{old}, \pi_{new}) $$
        where $C$ is the penalty coefficient (i.e. hyperparameter)
        \item The penalty  would be fairly small in practice (as we know from theory)
        \medskip
        \item Since the penalty depends on the KL divergence between old and new policy, we can constrain the update to keep divergence low:
        \begin{eqnarray}
        \max_{\theta} L_{\pi_{old}} (\pi_{new})\\
        \text{subject to } D_{KL}^{s \sim \mu_{\pi_{old}}} (\pi_{old}, \pi_{new}) \leq \delta
        \end{eqnarray}
        \item This uses the average KL instead of max (the max requires the KL to be bounded at all states and yields an impractical number of constraints)
    \end{itemize}
    
\end{frame}
%-----------------------------------------------------------------------
%----------------------------------------------------------------------
\begin{frame}[c]{From Theory to Practice}
	
    \begin{itemize}
        \item Current objective:
        $$ \max_{\theta} L_{\pi_{old}} (\pi_{new})$$
        {\centering subject to $ D_{KL}^{s \sim \alert{\mu_{\pi_{old}}}} (\pi_{old}, \pi_{new}) \leq \delta$\\}
        where $ L_{\pi_{old}}(\pi_{new}) = V(\pi_{old}) + \sum_{s} \alert{\mu_{\pi_{old}}(s)} \sum_{a} \pi_{new}(a \mid s) \alert{A_{\pi_{old}}(s,a)}$ 
        \item But: in practice we don't know the visitation weights nor true advantage function
        \pause
        \item Step 1: estimate $\mu_\pi$ using samples we have already collected:
        $$ \sum_{s} \mu_\pi (s) \to \frac{1}{1- \gamma} \mathbb{E}_{s\sim \mu_{\pi_{old}}} [\ldots]$$
        \item[$\leadsto$] estimate $\mu_\pi (s)$ wrt the sampled trajectories we have already
    \end{itemize}

\end{frame}
%-----------------------------------------------------------------------
%----------------------------------------------------------------------
\begin{frame}[c]{From Theory to Practice II}
	
    \begin{itemize}
        \item Step 2: use \alert{importance sampling} to estimate the desired sum:
        $$\sum_{a} \pi_{new} (a\mid s_n) A_{\pi_{old}} (s_n, a) \to \mathbb{E}\left[ \frac{\pi_{new}(a \mid s_n)}{q(a\mid s_n)} A_{\pi_{old}} (s_n, a) \right] $$
        \item where $q$ is some sampling distribution over actions and $s_n$ is a particular sampled state
        \item Now we are able to use samples from $q$ (instead of the new policy $\pi_{new}$)
        \pause
        \item Step 3:
        $$ A_{\theta_{old}} \to Q_{\theta_{old}}$$
        \item Note that these substitutions do not change the solution to the optimization problem
    \end{itemize}

\end{frame}
%-----------------------------------------------------------------------
%----------------------------------------------------------------------
\begin{frame}[c]{Policy Gradient with Step Size Adaption}
	
    \begin{itemize}
        \item Optimize new surrogate objective $L$:
        $$ \mathbb{E}_{s\sim \mu_{\pi_{old}}, a \sim q} \left[ \frac{\pi_{new}(a \mid s_n)}{q(a\mid s_n)} Q_{\pi_{old}} (s_n, a) \right]$$
        {\centering subject to $\mathbb{E}_{s\sim \mu_{\pi_{old}}} D_{KL}(\pi_{old}(\cdot \mid s), \pi_{new} (\cdot \mid s)) \leq \delta$ \\}
        \item Standard approach: sampling distribution $q(a\mid s)$ is simply $\pi_{old}(a \mid s)$
        \item Then we can compute this update using existing samples and don't need to evaluate $\pi_{new}$
        %\item For the vine procedure see the paper
    \end{itemize}
    
    % \centering
    % \includegraphics[width=0.5\textwidth]{images/trpo.PNG}\\
    % \lit{Schulman et al. 2015}{https://arxiv.org/abs/1502.05477}

\end{frame}


\end{document}