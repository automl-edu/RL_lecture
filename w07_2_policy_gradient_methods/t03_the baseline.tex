% !TeX spellcheck = en_US
\documentclass[aspectratio=169]{../latex_main/tntbeamer}  % you can pass all options of the beamer class, e.g., 'handout' or 'aspectratio=43'
\usepackage{dsfont}
\usepackage{bm}
\usepackage[english]{babel}
\usepackage[T1]{fontenc}
%\usepackage[utf8]{inputenc}
\usepackage{graphicx}
\graphicspath{ {./figures/} }
\usepackage{algorithm}
\usepackage[ruled,vlined,algo2e,linesnumbered]{algorithm2e}
\usepackage{hyperref}
\usepackage{booktabs}
\usepackage{mathtools}

\usepackage{amsmath,amssymb}
\usepackage{latexsym}

\DeclareMathOperator*{\argmax}{arg\,max}
\DeclareMathOperator*{\argmin}{arg\,min}

\usepackage{pgfplots}
\pgfplotsset{compat=1.16}
\usepackage{tikz}
\usetikzlibrary{trees} 
\usetikzlibrary{shapes.geometric}
\usetikzlibrary{positioning,shapes,shadows,arrows,calc,mindmap}
\usetikzlibrary{positioning,fadings,through}
\usetikzlibrary{decorations.pathreplacing}
\usetikzlibrary{intersections}
\pgfdeclarelayer{background}
\pgfdeclarelayer{foreground}
\pgfsetlayers{background,main,foreground}
\tikzstyle{activity}=[rectangle, draw=black, rounded corners, text centered, text width=8em]
\tikzstyle{data}=[rectangle, draw=black, text centered, text width=8em]
\tikzstyle{myarrow}=[->, thick, draw=black]

% Define the layers to draw the diagram
\pgfdeclarelayer{background}
\pgfdeclarelayer{foreground}
\pgfsetlayers{background,main,foreground}

\input{./latex_main_old/macros}

% Requires XeLaTeX or LuaLaTeX
\usepackage{unicode-math}

\usepackage{fontspec}
%\setsansfont{Arial}
\setsansfont{RotisSansSerifStd}[ 
Path=./latex_main/fonts/,
Extension = .otf,
UprightFont = *-Regular,  % or *-Light
BoldFont = *-ExtraBold,  % or *-Bold
ItalicFont = *-Italic
]
\setmonofont{Cascadia Mono}[
Scale=0.8
]

% scale factor adapted; mathrm font added (Benjamin Spitschan @TNT, 2021-06-01)
%\setmathfont[Scale=1.05]{Libertinus Math}
%\setmathrm[Scale=1.05]{Libertinus Math}

% other available math fonts are (not exhaustive)
% Latin Modern Math
% XITS Math
% Libertinus Math
% Asana Math
% Fira Math
% TeX Gyre Pagella Math
% TeX Gyre Bonum Math
% TeX Gyre Schola Math
% TeX Gyre Termes Math

% Literature References
% #1 = Display Name
% #2 = Url (without \href)
\newcommand{\lit}[2]{\href{#2}{\footnotesize\color{black!60}[#1]}}

%%% Beamer Customization
%----------------------------------------------------------------------
% (Don't) Show sections in frame header. Options: 'sections', 'sections light', empty
\setbeamertemplate{headline}{empty}

% Add header logo for normal frames
\setheaderimage{
	% \includegraphics[height=\logoheight]{figures/TNT_darkv4.pdf}
	\includegraphics[height=\logoheight]{./latex_main/figures/luh_logo_rgb_0_80_155.pdf}
	% \includegraphics[height=\logoheight]{figures/logo_tntluh.pdf}
}

% Header logo for title page
\settitleheaderimage{
	% \includegraphics[height=\logoheight]{figures/TNT_darkv4.pdf}
	\includegraphics[height=\logoheight]{./latex_main/figures/luh_logo_rgb_0_80_155.pdf}
	% \includegraphics[height=\logoheight]{figures/logo_tntluh.pdf}
}

% Title page: tntdefault 
\setbeamertemplate{title page}[tntdefault]  % or luhstyle
% Add optional title image here
%\addtitlepageimagedefault{\includegraphics[width=0.65\textwidth]{figures/luh_default_presentation_title_image.jpg}}

% Title page: luhstyle
% \setbeamertemplate{title page}[luhstyle]
% % Add optional title image here
% \addtitlepageimage{\includegraphics[width=0.75\textwidth]{figures/luh_default_presentation_title_image.jpg}}

\author[Lindauer]{Marius Lindauer\\[1em]
	\includegraphics[height=\logoheight]{./latex_main/figures/luh_logo_rgb_0_80_155.pdf}\qquad
\includegraphics[height=\logoheight]{./latex_main/figures/TNT_darkv4}\qquad
\includegraphics[height=\logoheight]{./latex_main/figures/L3S.jpg}	}
\date{Winter Term 2021
}


%%% Custom Packages
%----------------------------------------------------------------------
% Create dummy content
\usepackage{blindtext}

% Adds a frame with the current page layout. Just call \layout inside of a frame.
\usepackage{layout}

\title[RL: Policy Gradient]{RL: Policy Search}
\subtitle{Policy Gradient Algorithms}



\begin{document}
	
	\maketitle
%-----------------------------------------------------------------------
%----------------------------------------------------------------------
\begin{frame}[c]{Fixing Policy Gradient II: Value Baselines}

\begin{itemize}
	\item We identified a problem: using the sampled return introduces a lot of variance in our gradient estimation:
 $$\theta := \theta + \alpha \nabla_\theta \log \pi_\theta (s_t, a_t) G_t $$
    \item So: we need to offset the noise we incur from using the return samples
    \item Ideally, we want to keep the gradient unbiased
\end{itemize}
	$\leadsto$ \alert{Idea}: We use a baseline for the return estimation to reduce variance without introducing bias
\end{frame}
%-----------------------------------------------------------------------
%----------------------------------------------------------------------
\begin{frame}[c]{Fixing Policy Gradient II: Value Baselines}

\begin{itemize}
	\item Reduce variance by introducing a baseline $b(s)$ to remove the influence of future actions
	
	$$\nabla_\theta \mathbb{E}_\tau [R] = \mathbb{E}_\tau \left[ \sum_{t=0}^{T-1} \nabla_\theta \log \pi(a_t \mid s_t; \theta) \left( \sum_{t'=t}^{T-1} r_{t'} - \alert{b(s_t)} \right) \right] $$
	
	\item For any choice of $b$, gradient estimator is unbiased
	\item Near-optimal choice is our approximation of the expected return
	$$b(s_t) \approx \mathbb{E} [r_t + r_{t+1} +\ldots + r_{T-1}] $$
	
	\item Interpretation: increase logprob of action $a_t$ proportionally to how much returns $\sum_{t'=t}^{T-1} r_{t'}$ are better than expected
	
\end{itemize}
\end{frame}
%-----------------------------------------------------------------------
%----------------------------------------------------------------------
\begin{frame}[c]{Why Does This Not Introduce Bias?}

\begin{itemize}
	\item Mathematically, when we distribute the sum on the right side we get two terms: the original policy gradient term, and an \alert{expectation that includes the baseline}: 
    $$ \mathbb{E}_\tau \bigg [ \nabla_\theta \log \pi(a_t \mid s_t; \theta) \cdot b(s_t) \bigg ]$$
	
	$$= \mathbb{E}_{s_{0;t},a_{0;t-1}} \left[ b(s_t)  \cdot \mathbb{E}_{a_t} \left[ \nabla_\theta \log \pi(a_t \mid s_t; \theta) \right] \right] $$
    
    $$=\mathbb{E}_\tau \left[ b(s_t) \cdot 0 \right] $$
	
	\item The expectation goes to zero because of the log derivative trick \href{https://danieltakeshi.github.io/2017/03/28/going-deeper-into-reinforcement-learning-fundamentals-of-policy-gradients/}{\alert{(click here for more details)}}

	\item More intuitively: there's no additional bias because we only re-weigh the action $a_t$ at timestep $t$ with a function $b$ that does not depend on the action at all
	
\end{itemize}
	
\end{frame}
% %-----------------------------------------------------------------------
\begin{frame}[c]{Intuition: Why Does It Help?}

\begin{itemize}
	\item Usually we would assume not changing the bias will also not change the variance
	
	\item In this case, it does since we offset the effect of our sampling procedure
 
	\item While the expectations in our gradient are technically constant, in practice, we approximate them with samples

    \item As we have seen, samples can diverge greatly over the course of a single training run
	
	\item With the baseline, we can get better updates on diverse samples since now we can remove the influence of future actions on each return step
 \newline
 $\leadsto$ Bringing the approximation of the expectation closer to the theoretical true value
\end{itemize}
	
\end{frame}
%----------------------------------------------------------------------
% \begin{frame}[c]{”Vanilla” Policy Gradient Algorithm}
	
% 	\begin{itemize}
% 		\item Initialize policy parameters $\theta$ and baseline $b$
% 		\item for iteration$=1,2,\ldots$ do
% 		\begin{itemize}
% 			\item Collect a set of trajectories by executing the current policy
% 			\item At each time step $t$ in each trajectory $\tau^i$, compute
% 			\begin{itemize}
% 				\item Return $G_t^i= \sum_{t'=t}^{T-1} r_{t'}^i$ and
% 				\item Advantage estimate $\hat{A}^i_t = G^i_t - b(s_t)$
% 			\end{itemize}
% 			\item Re-fit the baseline by minimizing $\sum_i \sum_t || b(s_t) - G^i_t||^2$
% 			\item Update the policy, using a policy gradient estimate $\hat{g}$
% 			\begin{itemize}
% 				\item which is a sum of terms $\nabla_\theta \log \pi(a_t \mid s_t; \theta) \hat{A}_t$
% 				\item Apply gradient $\hat{g}$ by any DL-optimizer (e.g., SGD or ADAM)
% 			\end{itemize}
% 		\end{itemize}
		
% 	\end{itemize}

% \bigskip

% $\leadsto$ Actor-Critic method
 
% \end{frame}
%-----------------------------------------------------------------------
% %----------------------------------------------------------------------
% \begin{frame}[c]{Choosing the Baseline: Value Functions}
	
% 	\begin{itemize}
% 		\item Recall $Q$-function:
% 		$$Q^\pi(s,a) = \mathbb{E}_\pi [r_0 +  \gamma r_1 + \gamma^2 r_2 + \ldots \mid s_0 = s, a_0 = a ]$$
% 		\item State-value function can serve as a great baseline
% 		\begin{eqnarray}
% 		V^\pi (s) &=& \mathbb{E}_\pi [r_0 +  \gamma r_1 + \gamma^2 r_2 + \ldots \mid s_0 = s]\nonumber\\	
% 		&=& \mathbb{E}_{a\sim\pi} [Q^\pi(s,a)]\nonumber
% 		\end{eqnarray}
% 		\item Advantage function: Combining $Q$ with baseline $V$:
% 		$$A^\pi(s,a) = Q^\pi(s,a) - V^\pi(s) $$
% 	\end{itemize}
	
% \end{frame}
% %-----------------------------------------------------------------------
%-----------------------------------------------------------------------
\end{document}