% !TeX spellcheck = en_US
\documentclass[aspectratio=169]{../latex_main/tntbeamer}  % you can pass all options of the beamer class, e.g., 'handout' or 'aspectratio=43'
\usepackage{dsfont}
\usepackage{bm}
\usepackage[english]{babel}
\usepackage[T1]{fontenc}
%\usepackage[utf8]{inputenc}
\usepackage{graphicx}
\graphicspath{ {./figures/} }
\usepackage{algorithm}
\usepackage[ruled,vlined,algo2e,linesnumbered]{algorithm2e}
\usepackage{hyperref}
\usepackage{booktabs}
\usepackage{mathtools}

\usepackage{amsmath,amssymb}
\usepackage{latexsym}

\DeclareMathOperator*{\argmax}{arg\,max}
\DeclareMathOperator*{\argmin}{arg\,min}

\usepackage{pgfplots}
\pgfplotsset{compat=1.16}
\usepackage{tikz}
\usetikzlibrary{trees} 
\usetikzlibrary{shapes.geometric}
\usetikzlibrary{positioning,shapes,shadows,arrows,calc,mindmap}
\usetikzlibrary{positioning,fadings,through}
\usetikzlibrary{decorations.pathreplacing}
\usetikzlibrary{intersections}
\pgfdeclarelayer{background}
\pgfdeclarelayer{foreground}
\pgfsetlayers{background,main,foreground}
\tikzstyle{activity}=[rectangle, draw=black, rounded corners, text centered, text width=8em]
\tikzstyle{data}=[rectangle, draw=black, text centered, text width=8em]
\tikzstyle{myarrow}=[->, thick, draw=black]

% Define the layers to draw the diagram
\pgfdeclarelayer{background}
\pgfdeclarelayer{foreground}
\pgfsetlayers{background,main,foreground}

\input{./latex_main_old/macros}

% Requires XeLaTeX or LuaLaTeX
\usepackage{unicode-math}

\usepackage{fontspec}
%\setsansfont{Arial}
\setsansfont{RotisSansSerifStd}[ 
Path=./latex_main/fonts/,
Extension = .otf,
UprightFont = *-Regular,  % or *-Light
BoldFont = *-ExtraBold,  % or *-Bold
ItalicFont = *-Italic
]
\setmonofont{Cascadia Mono}[
Scale=0.8
]

% scale factor adapted; mathrm font added (Benjamin Spitschan @TNT, 2021-06-01)
%\setmathfont[Scale=1.05]{Libertinus Math}
%\setmathrm[Scale=1.05]{Libertinus Math}

% other available math fonts are (not exhaustive)
% Latin Modern Math
% XITS Math
% Libertinus Math
% Asana Math
% Fira Math
% TeX Gyre Pagella Math
% TeX Gyre Bonum Math
% TeX Gyre Schola Math
% TeX Gyre Termes Math

% Literature References
% #1 = Display Name
% #2 = Url (without \href)
\newcommand{\lit}[2]{\href{#2}{\footnotesize\color{black!60}[#1]}}

%%% Beamer Customization
%----------------------------------------------------------------------
% (Don't) Show sections in frame header. Options: 'sections', 'sections light', empty
\setbeamertemplate{headline}{empty}

% Add header logo for normal frames
\setheaderimage{
	% \includegraphics[height=\logoheight]{figures/TNT_darkv4.pdf}
	\includegraphics[height=\logoheight]{./latex_main/figures/luh_logo_rgb_0_80_155.pdf}
	% \includegraphics[height=\logoheight]{figures/logo_tntluh.pdf}
}

% Header logo for title page
\settitleheaderimage{
	% \includegraphics[height=\logoheight]{figures/TNT_darkv4.pdf}
	\includegraphics[height=\logoheight]{./latex_main/figures/luh_logo_rgb_0_80_155.pdf}
	% \includegraphics[height=\logoheight]{figures/logo_tntluh.pdf}
}

% Title page: tntdefault 
\setbeamertemplate{title page}[tntdefault]  % or luhstyle
% Add optional title image here
%\addtitlepageimagedefault{\includegraphics[width=0.65\textwidth]{figures/luh_default_presentation_title_image.jpg}}

% Title page: luhstyle
% \setbeamertemplate{title page}[luhstyle]
% % Add optional title image here
% \addtitlepageimage{\includegraphics[width=0.75\textwidth]{figures/luh_default_presentation_title_image.jpg}}

\author[Lindauer]{Marius Lindauer\\[1em]
	\includegraphics[height=\logoheight]{./latex_main/figures/luh_logo_rgb_0_80_155.pdf}\qquad
\includegraphics[height=\logoheight]{./latex_main/figures/TNT_darkv4}\qquad
\includegraphics[height=\logoheight]{./latex_main/figures/L3S.jpg}	}
\date{Winter Term 2021
}


%%% Custom Packages
%----------------------------------------------------------------------
% Create dummy content
\usepackage{blindtext}

% Adds a frame with the current page layout. Just call \layout inside of a frame.
\usepackage{layout}

\title[RL: Exploration]{Exploration in RL}
\subtitle{Traditional Exploration Strategies for MDPs}



\begin{document}
	
	\maketitle

%----------------------------------------------------------------------
%----------------------------------------------------------------------
\begin{frame}[c]{Recap: Bandit Exploration}

\begin{itemize}
	\item Optimistic initialization
	\item Optimism in the face of uncertainty (Upper Confidence bounds)
	\item Probability matching (Thompson Sampling)
\end{itemize}

\end{frame}
%-----------------------------------------------------------------------
%----------------------------------------------------------------------
\begin{frame}[c]{Optimistic Initialization: Model-free RL}
	
	\begin{itemize}
		\item Initialize action-value function $Q(s,a)$ to $\frac{r_{max}}{1-\gamma}$
		\item Run favorite model-free RL algorithm
		\begin{itemize}
			\item Monte-carlo method
			\item Sarsa
			\item Q-Learning
		\end{itemize}
		\item Encourages systematic exploration of states and actions
	\end{itemize}
	
\end{frame}
%-----------------------------------------------------------------------
%----------------------------------------------------------------------
\begin{frame}[c]{Upper Confidence Bounds: Model-free RL}
	
	\begin{itemize}
		\item Maximize UCB on action-value function $Q^\pi(s,a)$
		$$a_t \in \argmax_{a \in A} Q(s_t, a) + U(s_t, a) $$
		
		\begin{itemize}
			\item Estimate uncertainty in policy evaluation (easy)
			\item Ignores uncertainty from policy improvement
		\end{itemize}
		\item Maximize UCB on optimal action-value function $Q^*(s,a)$
		$$a_t \in \argmax_{a \in A} Q(s_t, a) + U_1(s_t, a) + U_2(s_t, a)$$
		\begin{itemize}
			\item Estimate uncertainty in policy evaluation (easy)
			\item plus uncertainty from policy improvement (hard)
		\end{itemize}
		
	\end{itemize}
	
\end{frame}
%-----------------------------------------------------------------------
%----------------------------------------------------------------------
\begin{frame}[c]{Bayesian Model-based RL}
	
	\begin{itemize}
		\item Maintain posterior distribution over MDP models
		\item Estimate both transitions and rewards $ \mathbb{P}[P, R \mid h_t]$
		\begin{itemize}
			\item where $h_t = s_1, a_1, r_2, \ldots, s_t$ is the history
		\end{itemize}
		\item Use posterior to guide exploration
		\begin{itemize}
			\item Upper confidence bounds (Bayesian UCB)
			\item Probability matching (Thompson sampling)
		\end{itemize}
		
	\end{itemize}
	
\end{frame}
%-----------------------------------------------------------------------
%----------------------------------------------------------------------
\begin{frame}[c]{Thompson Sampling: Model-based RL}
	
	\begin{itemize}
		\item Thompson sampling implements probability matching
	\begin{eqnarray}
	\pi(s, a \mid h_t) &=& \mathbb{P} [Q^*(s,a) > Q^*(s,a'), \forall a' \neq a \mid h_t]\nonumber \\
	&=& \mathbb{E}_{P, R \mid h_t} \left[\mathbf{1}(a \in \argmax_{a \in A} Q^*(s,a))\right]		\nonumber
	\end{eqnarray}
	\end{itemize}
	
	\begin{enumerate}		
		\item Use Bayes law to compute posterior $ \mathbb{P}[P, R \mid h_t]$
		\item Sample an MDP $P, R$ from posterior
		\item Solve MDP using favorite planning algorithm to get $Q^*(s,a)$
		\item Select optimal action for sample MDP: $a_t \in \argmax_{a \in A}Q^*(s_t,a)$
	\end{enumerate}
	
\end{frame}
%-----------------------------------------------------------------------

%-----------------------------------------------------------------------
\end{document}
