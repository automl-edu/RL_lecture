% !TeX spellcheck = en_US

\documentclass[aspectratio=169]{tntbeamer}  % you can pass all options of the beamer class, e.g., 'handout' or 'aspectratio=43'
\usepackage{dsfont}
\usepackage{bm}
\usepackage[english]{babel}
\usepackage[T1]{fontenc}
%\usepackage[utf8]{inputenc}
\usepackage{graphicx}
\graphicspath{ {./figures/} }
\usepackage{algorithm}
\usepackage[ruled,vlined,algo2e,linesnumbered]{algorithm2e}
\usepackage{hyperref}
\usepackage{booktabs}
\usepackage{mathtools}

\usepackage{amsmath,amssymb}
\usepackage{latexsym}

\DeclareMathOperator*{\argmax}{arg\,max}
\DeclareMathOperator*{\argmin}{arg\,min}

\usepackage{pgfplots}
\pgfplotsset{compat=1.16}
\usepackage{tikz}
\usetikzlibrary{trees} 
\usetikzlibrary{shapes.geometric}
\usetikzlibrary{positioning,shapes,shadows,arrows,calc,mindmap}
\usetikzlibrary{positioning,fadings,through}
\usetikzlibrary{decorations.pathreplacing}
\usetikzlibrary{intersections}
\pgfdeclarelayer{background}
\pgfdeclarelayer{foreground}
\pgfsetlayers{background,main,foreground}
\tikzstyle{activity}=[rectangle, draw=black, rounded corners, text centered, text width=8em]
\tikzstyle{data}=[rectangle, draw=black, text centered, text width=8em]
\tikzstyle{myarrow}=[->, thick, draw=black]

% Define the layers to draw the diagram
\pgfdeclarelayer{background}
\pgfdeclarelayer{foreground}
\pgfsetlayers{background,main,foreground}

\input{./latex_main_old/macros}

% Requires XeLaTeX or LuaLaTeX
\usepackage{unicode-math}

\usepackage{fontspec}
%\setsansfont{Arial}
\setsansfont{RotisSansSerifStd}[ 
Path=./latex_main/fonts/,
Extension = .otf,
UprightFont = *-Regular,  % or *-Light
BoldFont = *-ExtraBold,  % or *-Bold
ItalicFont = *-Italic
]
\setmonofont{Cascadia Mono}[
Scale=0.8
]

% scale factor adapted; mathrm font added (Benjamin Spitschan @TNT, 2021-06-01)
%\setmathfont[Scale=1.05]{Libertinus Math}
%\setmathrm[Scale=1.05]{Libertinus Math}

% other available math fonts are (not exhaustive)
% Latin Modern Math
% XITS Math
% Libertinus Math
% Asana Math
% Fira Math
% TeX Gyre Pagella Math
% TeX Gyre Bonum Math
% TeX Gyre Schola Math
% TeX Gyre Termes Math

% Literature References
% #1 = Display Name
% #2 = Url (without \href)
\newcommand{\lit}[2]{\href{#2}{\footnotesize\color{black!60}[#1]}}

%%% Beamer Customization
%----------------------------------------------------------------------
% (Don't) Show sections in frame header. Options: 'sections', 'sections light', empty
\setbeamertemplate{headline}{empty}

% Add header logo for normal frames
\setheaderimage{
	% \includegraphics[height=\logoheight]{figures/TNT_darkv4.pdf}
	\includegraphics[height=\logoheight]{./latex_main/figures/luh_logo_rgb_0_80_155.pdf}
	% \includegraphics[height=\logoheight]{figures/logo_tntluh.pdf}
}

% Header logo for title page
\settitleheaderimage{
	% \includegraphics[height=\logoheight]{figures/TNT_darkv4.pdf}
	\includegraphics[height=\logoheight]{./latex_main/figures/luh_logo_rgb_0_80_155.pdf}
	% \includegraphics[height=\logoheight]{figures/logo_tntluh.pdf}
}

% Title page: tntdefault 
\setbeamertemplate{title page}[tntdefault]  % or luhstyle
% Add optional title image here
%\addtitlepageimagedefault{\includegraphics[width=0.65\textwidth]{figures/luh_default_presentation_title_image.jpg}}

% Title page: luhstyle
% \setbeamertemplate{title page}[luhstyle]
% % Add optional title image here
% \addtitlepageimage{\includegraphics[width=0.75\textwidth]{figures/luh_default_presentation_title_image.jpg}}

\author[Lindauer]{Marius Lindauer\\[1em]
	\includegraphics[height=\logoheight]{./latex_main/figures/luh_logo_rgb_0_80_155.pdf}\qquad
\includegraphics[height=\logoheight]{./latex_main/figures/TNT_darkv4}\qquad
\includegraphics[height=\logoheight]{./latex_main/figures/L3S.jpg}	}
\date{Winter Term 2021
}


%%% Custom Packages
%----------------------------------------------------------------------
% Create dummy content
\usepackage{blindtext}

% Adds a frame with the current page layout. Just call \layout inside of a frame.
\usepackage{layout}


\institute{Institut f\"ur Informationsverarbeitung}%\\ Leibniz Universit\"at Hannover}
%\title{TNT Beamer Template}
%\author{Suomynon A. Anonymous}
\date{}



\title[Reinforcement Learning: Big Picture]{Exploration in RL}
\subtitle{Traditional Exploration Strategies for Bandits}



\begin{document}
	
	\maketitle

%----------------------------------------------------------------------
%----------------------------------------------------------------------
\begin{frame}[c]{Recap: The Bandit Problem}

\begin{itemize}
	\item Simplified RL setting with no states
	\item Simply try to identify which action $a^* \in \mathcal{A}$ is the best one
	\begin{itemize}
		\item of course, we want to be efficient in doing that!
	\end{itemize}
	\item Reward is drawn from some unknown distribution
	\bigskip
	\pause
	\item[$\leadsto$] That's exactly the problem you face in every state $s$ again.\\ Let's assume that we fix $s$ for the moment
\end{itemize}

\end{frame}
%-----------------------------------------------------------------------
%----------------------------------------------------------------------
\begin{frame}[c]{$\epsilon$-greedy}
	
	\begin{itemize}
		\item play best known action $\hat{a}$ with probability $1-\epsilon$
		\item play a random action $a \in \mathcal{A}$ with probability $\epsilon$ 
		\smallskip
		\item Question: Is this a zero-regret strategy?
		\pause
		\begin{itemize}
			\item No, since with probability $\epsilon$ we will obtain a non-zero regret and therefore, even in the limit the regret will not go to zero, but obtain a linear regret.
		\end{itemize}
		\smallskip\pause
		\item Solution: Anneal $\epsilon$ over time
		\item If we linearly anneal $\epsilon$ over time to $0$, do we have a zero-regret strategy?
		\pause
		\begin{itemize}
			\item No, because of the linear annealing, we have only a finite amount of observations which might not suffice to identify the best action
		\end{itemize}
		\smallskip
		\item[$\leadsto$] Anneal $\epsilon$ proportional $\sqrt{t}$ or $1/t$
	\end{itemize}
	
\end{frame}
%-----------------------------------------------------------------------
%----------------------------------------------------------------------
\begin{frame}[c]{Upper Confidence Bounds}
	
	\begin{itemize}
		\item Track all rewards you obtained by playing each action $a_k$ and compute mean $\mu(a_k)$ and standard deviation $\sigma(a_k)$ to estimate the underlying distribution
		\item Optimistic in face of uncertainty by upper confidence bound:
		$$\mu(a_k) + \kappa \cdot \sigma(a_k)$$
		\item Idea: Over time, we get more and more evidence for the best actions until we are sure that the best known is really the best
		\medskip
		\pause
		\item To prevent premature convergence: Use optimistic initialization of each action s.t. all actions are played in the beginning
	\end{itemize}
	
\end{frame}
%-----------------------------------------------------------------------
%----------------------------------------------------------------------
\begin{frame}[c]{Thompson Sampling}
	
	\begin{itemize}
		\item Track all rewards you obtained by playing each action $a_k$ and compute mean $\mu(a_k)$ and standard deviation $\sigma(a_k)$ to estimate the underlying distribution
		\item Draw from each estimated distribution a single realization and simply play the action with the best one
		$$ s_k \sim \mathcal{N}(\mu(a_k), \sigma(a_k))$$
		\item In the limit, only the best performing action will be played with high probability
	\end{itemize}
	
\end{frame}
%-----------------------------------------------------------------------
%-----------------------------------------------------------------------
\end{document}
