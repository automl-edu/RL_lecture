% !TeX spellcheck = en_US
\documentclass[aspectratio=169]{../latex_main/tntbeamer}  % you can pass all options of the beamer class, e.g., 'handout' or 'aspectratio=43'
\usepackage{dsfont}
\usepackage{bm}
\usepackage[english]{babel}
\usepackage[T1]{fontenc}
%\usepackage[utf8]{inputenc}
\usepackage{graphicx}
\graphicspath{ {./figures/} }
\usepackage{algorithm}
\usepackage[ruled,vlined,algo2e,linesnumbered]{algorithm2e}
\usepackage{hyperref}
\usepackage{booktabs}
\usepackage{mathtools}

\usepackage{amsmath,amssymb}
\usepackage{latexsym}

\DeclareMathOperator*{\argmax}{arg\,max}
\DeclareMathOperator*{\argmin}{arg\,min}

\usepackage{pgfplots}
\pgfplotsset{compat=1.16}
\usepackage{tikz}
\usetikzlibrary{trees} 
\usetikzlibrary{shapes.geometric}
\usetikzlibrary{positioning,shapes,shadows,arrows,calc,mindmap}
\usetikzlibrary{positioning,fadings,through}
\usetikzlibrary{decorations.pathreplacing}
\usetikzlibrary{intersections}
\pgfdeclarelayer{background}
\pgfdeclarelayer{foreground}
\pgfsetlayers{background,main,foreground}
\tikzstyle{activity}=[rectangle, draw=black, rounded corners, text centered, text width=8em]
\tikzstyle{data}=[rectangle, draw=black, text centered, text width=8em]
\tikzstyle{myarrow}=[->, thick, draw=black]

% Define the layers to draw the diagram
\pgfdeclarelayer{background}
\pgfdeclarelayer{foreground}
\pgfsetlayers{background,main,foreground}

\input{./latex_main_old/macros}

% Requires XeLaTeX or LuaLaTeX
\usepackage{unicode-math}

\usepackage{fontspec}
%\setsansfont{Arial}
\setsansfont{RotisSansSerifStd}[ 
Path=./latex_main/fonts/,
Extension = .otf,
UprightFont = *-Regular,  % or *-Light
BoldFont = *-ExtraBold,  % or *-Bold
ItalicFont = *-Italic
]
\setmonofont{Cascadia Mono}[
Scale=0.8
]

% scale factor adapted; mathrm font added (Benjamin Spitschan @TNT, 2021-06-01)
%\setmathfont[Scale=1.05]{Libertinus Math}
%\setmathrm[Scale=1.05]{Libertinus Math}

% other available math fonts are (not exhaustive)
% Latin Modern Math
% XITS Math
% Libertinus Math
% Asana Math
% Fira Math
% TeX Gyre Pagella Math
% TeX Gyre Bonum Math
% TeX Gyre Schola Math
% TeX Gyre Termes Math

% Literature References
% #1 = Display Name
% #2 = Url (without \href)
\newcommand{\lit}[2]{\href{#2}{\footnotesize\color{black!60}[#1]}}

%%% Beamer Customization
%----------------------------------------------------------------------
% (Don't) Show sections in frame header. Options: 'sections', 'sections light', empty
\setbeamertemplate{headline}{empty}

% Add header logo for normal frames
\setheaderimage{
	% \includegraphics[height=\logoheight]{figures/TNT_darkv4.pdf}
	\includegraphics[height=\logoheight]{./latex_main/figures/luh_logo_rgb_0_80_155.pdf}
	% \includegraphics[height=\logoheight]{figures/logo_tntluh.pdf}
}

% Header logo for title page
\settitleheaderimage{
	% \includegraphics[height=\logoheight]{figures/TNT_darkv4.pdf}
	\includegraphics[height=\logoheight]{./latex_main/figures/luh_logo_rgb_0_80_155.pdf}
	% \includegraphics[height=\logoheight]{figures/logo_tntluh.pdf}
}

% Title page: tntdefault 
\setbeamertemplate{title page}[tntdefault]  % or luhstyle
% Add optional title image here
%\addtitlepageimagedefault{\includegraphics[width=0.65\textwidth]{figures/luh_default_presentation_title_image.jpg}}

% Title page: luhstyle
% \setbeamertemplate{title page}[luhstyle]
% % Add optional title image here
% \addtitlepageimage{\includegraphics[width=0.75\textwidth]{figures/luh_default_presentation_title_image.jpg}}

\author[Lindauer]{Marius Lindauer\\[1em]
	\includegraphics[height=\logoheight]{./latex_main/figures/luh_logo_rgb_0_80_155.pdf}\qquad
\includegraphics[height=\logoheight]{./latex_main/figures/TNT_darkv4}\qquad
\includegraphics[height=\logoheight]{./latex_main/figures/L3S.jpg}	}
\date{Winter Term 2021
}


%%% Custom Packages
%----------------------------------------------------------------------
% Create dummy content
\usepackage{blindtext}

% Adds a frame with the current page layout. Just call \layout inside of a frame.
\usepackage{layout}

\title[RL: Exploration]{Exploration in RL}
\subtitle{Intrinsic Exploration\footnote{based on \href{https://lilianweng.github.io/lil-log/2020/06/07/exploration-strategies-in-deep-reinforcement-learning.html}{Blog by Lilian Weng}}}



\begin{document}
	
	\maketitle

%----------------------------------------------------------------------
%----------------------------------------------------------------------
\begin{frame}[c]{The Hard Exploration Problem}
	
	\begin{itemize}
		\item hard-exploration problems:
		\begin{itemize}
			\item very sparse rewards 
			\item or even deceptive rewards
		\end{itemize}
		\medskip
		\pause
		\item Examples
		\begin{itemize}
			\item Montezuma's Revenge (Atari): long sequence of steps needed to figure out that ``key'' is needed to open ``door''
			\item Noisy-TV problem: 
			\begin{itemize}
				\item Assumption: Agent gets explicit reward for seeking novel experience
				\item Agent discovers TV that only shows random images
				\item Agent will watch TV forever (without solving the real task)!
			\end{itemize}
		\end{itemize}
	\end{itemize}

\end{frame}
%-----------------------------------------------------------------------
%----------------------------------------------------------------------
\begin{frame}[c]{Intrinsic Rewards as Exploration Bonus}
	
	\begin{itemize}
		\item Augment reward by reward external reward $r^e$ and intrinsic reward $r^i$
		$$r_t = r^e_t + \beta r_t^i $$
		\item Inspired by intrinsic motivation in psychology
		\begin{itemize}
			\item Children are driven by curiosity which helps to learn
			\item Intrinsic rewards could be correlated with curiosity, surprise, familiarity of the state and more
		\end{itemize}
		\item Two main ideas for RL
		\begin{itemize}
			\item Discovery of novel states
			\item Improvement of the agent's knowledge about the environment
		\end{itemize}
	\end{itemize}
	
\end{frame}
%-----------------------------------------------------------------------
%----------------------------------------------------------------------
\begin{frame}[c]{Count-based Exploration}
	
	\begin{itemize}
		\item What does it mean that the agent is surprised that it discovered something new?
		\item[$\leadsto$] Measure whether the state is novel or appeared often
		\item Count how many times a state was encountered and assign bonus to rarely encountered states
		\begin{itemize}
			\item Count-based exploration
			\item $N_n(s)$: number of visits of state $s$ in the sequence $s_{1:n}$
			\item Problem: Most $N(s)$ will be zero for non-trivial environments
		\end{itemize}
	\end{itemize}
	
\end{frame}
%-----------------------------------------------------------------------
%----------------------------------------------------------------------
\begin{frame}[c]{Counting by Density Model~\lit{Bellemare et al. 2016}{https://arxiv.org/abs/1606.01868}}
	
	\begin{itemize}
		\item Use a density model to approximate the frequency of state visits
		\item $p_n(s) = p(s \mid s_{1:n})$ is the probability of the $(n+1)$-th state being $s$
		\begin{itemize}
			\item empirically: $p_n(s) = N_n(s) / n$
		\end{itemize}
		\item $p'_n(s) = p(s \mid s_{1:n} s)$: probability assigned by the density model to $s$ after observing a new occurrence of $s$
		$$p_n(s) =  \frac{\hat{N}_n(s)}{\hat{n}} \leq \frac{\hat{N}_n(s) + 1}{\hat{n} + 1} = p'_n(s)$$
		\begin{itemize}
			\item where $\hat{N}_n(s)$ is a pseudo-count function and $\hat{n}$ a pseudo-count total which regulates the density function.
			\item learning-positive of density function is required since visiting $s$ again ($p'_n(s)$) should increase probability
		\end{itemize}
	\end{itemize}
	
\end{frame}
%-----------------------------------------------------------------------
%----------------------------------------------------------------------
\begin{frame}[c]{Count-based Intrinsic Bonus}
	
	\begin{itemize}
		\item Common choice \lit{Strehl and Littmann. 2008}:
		$$r_t^i = N(s_t, a_t)^{-1/2}$$
		\item For pseudo-count based exploration, very similar:
		$$r_t^i = (\hat{N}_n (s_t, a_t) + 0.01)^{-1/2} $$
	\end{itemize}
	
\end{frame}
%-----------------------------------------------------------------------
%-----------------------------------------------------------------------
\end{document}
