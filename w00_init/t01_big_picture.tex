\documentclass[aspectratio=169]{../latex_main/tntbeamer}  % you can pass all options of the beamer class, e.g., 'handout' or 'aspectratio=43'
\usepackage{dsfont}
\usepackage{bm}
\usepackage[english]{babel}
\usepackage[T1]{fontenc}
%\usepackage[utf8]{inputenc}
\usepackage{graphicx}
\graphicspath{ {./figures/} }
\usepackage{algorithm}
\usepackage[ruled,vlined,algo2e,linesnumbered]{algorithm2e}
\usepackage{hyperref}
\usepackage{booktabs}
\usepackage{mathtools}

\usepackage{amsmath,amssymb}
\usepackage{latexsym}

\DeclareMathOperator*{\argmax}{arg\,max}
\DeclareMathOperator*{\argmin}{arg\,min}

\usepackage{pgfplots}
\pgfplotsset{compat=1.16}
\usepackage{tikz}
\usetikzlibrary{trees} 
\usetikzlibrary{shapes.geometric}
\usetikzlibrary{positioning,shapes,shadows,arrows,calc,mindmap}
\usetikzlibrary{positioning,fadings,through}
\usetikzlibrary{decorations.pathreplacing}
\usetikzlibrary{intersections}
\pgfdeclarelayer{background}
\pgfdeclarelayer{foreground}
\pgfsetlayers{background,main,foreground}
\tikzstyle{activity}=[rectangle, draw=black, rounded corners, text centered, text width=8em]
\tikzstyle{data}=[rectangle, draw=black, text centered, text width=8em]
\tikzstyle{myarrow}=[->, thick, draw=black]

% Define the layers to draw the diagram
\pgfdeclarelayer{background}
\pgfdeclarelayer{foreground}
\pgfsetlayers{background,main,foreground}

\input{./latex_main_old/macros}

% Requires XeLaTeX or LuaLaTeX
\usepackage{unicode-math}

\usepackage{fontspec}
%\setsansfont{Arial}
\setsansfont{RotisSansSerifStd}[ 
Path=./latex_main/fonts/,
Extension = .otf,
UprightFont = *-Regular,  % or *-Light
BoldFont = *-ExtraBold,  % or *-Bold
ItalicFont = *-Italic
]
\setmonofont{Cascadia Mono}[
Scale=0.8
]

% scale factor adapted; mathrm font added (Benjamin Spitschan @TNT, 2021-06-01)
%\setmathfont[Scale=1.05]{Libertinus Math}
%\setmathrm[Scale=1.05]{Libertinus Math}

% other available math fonts are (not exhaustive)
% Latin Modern Math
% XITS Math
% Libertinus Math
% Asana Math
% Fira Math
% TeX Gyre Pagella Math
% TeX Gyre Bonum Math
% TeX Gyre Schola Math
% TeX Gyre Termes Math

% Literature References
% #1 = Display Name
% #2 = Url (without \href)
\newcommand{\lit}[2]{\href{#2}{\footnotesize\color{black!60}[#1]}}

%%% Beamer Customization
%----------------------------------------------------------------------
% (Don't) Show sections in frame header. Options: 'sections', 'sections light', empty
\setbeamertemplate{headline}{empty}

% Add header logo for normal frames
\setheaderimage{
	% \includegraphics[height=\logoheight]{figures/TNT_darkv4.pdf}
	\includegraphics[height=\logoheight]{./latex_main/figures/luh_logo_rgb_0_80_155.pdf}
	% \includegraphics[height=\logoheight]{figures/logo_tntluh.pdf}
}

% Header logo for title page
\settitleheaderimage{
	% \includegraphics[height=\logoheight]{figures/TNT_darkv4.pdf}
	\includegraphics[height=\logoheight]{./latex_main/figures/luh_logo_rgb_0_80_155.pdf}
	% \includegraphics[height=\logoheight]{figures/logo_tntluh.pdf}
}

% Title page: tntdefault 
\setbeamertemplate{title page}[tntdefault]  % or luhstyle
% Add optional title image here
%\addtitlepageimagedefault{\includegraphics[width=0.65\textwidth]{figures/luh_default_presentation_title_image.jpg}}

% Title page: luhstyle
% \setbeamertemplate{title page}[luhstyle]
% % Add optional title image here
% \addtitlepageimage{\includegraphics[width=0.75\textwidth]{figures/luh_default_presentation_title_image.jpg}}

\author[Lindauer]{Marius Lindauer\\[1em]
	\includegraphics[height=\logoheight]{./latex_main/figures/luh_logo_rgb_0_80_155.pdf}\qquad
\includegraphics[height=\logoheight]{./latex_main/figures/TNT_darkv4}\qquad
\includegraphics[height=\logoheight]{./latex_main/figures/L3S.jpg}	}
\date{Winter Term 2021
}


%%% Custom Packages
%----------------------------------------------------------------------
% Create dummy content
\usepackage{blindtext}

% Adds a frame with the current page layout. Just call \layout inside of a frame.
\usepackage{layout}


\title[ML-RL: Big Picture]{Welcome to the RL Lecture}
\subtitle{Brief Motivation and Orga}




\begin{document}
	
	\maketitle

%----------------------------------------------------------------------
%----------------------------------------------------------------------
\begin{frame}[c]{Team}
	
	\begin{columns}[T]
		
		\column{0.3\textwidth}
		\centering
		\includegraphics[height=7em]{images/marius}
		
		Prof. Dr.\\ Marius Lindauer
		
		\column{0.3\textwidth}
		\centering
		\includegraphics[height=7em]{images/caro}
		
		Carolin Benjamins\\
		
		\column{0.3\textwidth}
		\centering
		\includegraphics[height=7em]{images/aditya}
		
		Aditya Mohan \\
		
		
		
% 		\column{0.3\textwidth}
% 		\centering
% 		\includegraphics[height=10em]{images/eimer}
		
% 		Theresa Eimer\\
	\end{columns}
	
	
\end{frame}
%----------------------------------------------------------------------
%----------------------------------------------------------------------
\begin{frame}[c]{Question}
	
	\centering
	{
	\huge
	Why are you interested to learn more about reinforcement learning (RL)?
	}

	%\bigskip	\bigskip
	%$\to$ use the chat to answer!
	
\end{frame}
%-----------------------------------------------------------------------
%----------------------------------------------------------------------
\begin{frame}[c]{Machine Learning}

\centering
\textit{``Machine learning is the science of getting computers to act\\
 without being explicitly programmed.''}

\hfill by Andrew Ng (inspired by Arthur Samuel)

\end{frame}
%-----------------------------------------------------------------------
%----------------------------------------------------------------------
\begin{frame}[c]{Reinforcement Learning}
	
	\centering
	\includegraphics[width=0.55\textwidth]{images/rl_comic.png}\footnote{Image source: Morning Brew and Marius Haakestad on Unsplash}
	
	\bigskip
	
	\begin{itemize}
		\item Data: Self-acquired observations + rewards
		\item Task: Learn how to behave s.t. reward is maximized
		\item[$\leadsto$] Not a single decision, but a sequence of good decisions
	\end{itemize}	
	
	
\end{frame}
%-----------------------------------------------------------------------

%----------------------------------------------------------------------
\begin{frame}[c]{Goals of the Lecture}
	
	You will be able to \ldots
	\begin{enumerate}
		\item \alert{understand} and \alert{explain} the basic algorithms in RL
		\smallskip
		\item \alert{discuss} the assumptions and limitations of RL and its algorithms
		\smallskip
		\item  \alert{decide} which RL algorithm to use on a given environments
		\smallskip
		\item \alert{do} research on RL yourself
		\begin{itemize}
			\item perfect opportunity to do a master project or thesis with us afterwards
		\end{itemize}
	\end{enumerate}
	
\end{frame}
%-----------------------------------------------------------------------
%----------------------------------------------------------------------
\begin{frame}[c]{Course Overview (tentative)}
	
	\begin{enumerate}
		\item Big Picture (Introduction) $\xleftarrow{}$ Today
		\item MDP, Policy, Value Iteration
		\item Policy Evaluation
		\item Model Free Control
		\item Linear Function Approximation
		\item Deep RL
		\item Policy Gradient
		\item Exploration
		\item Meta-RL
		\item Reproducibility in RL
		\item Auto-RL
		\item Project
	\end{enumerate}
	
	\pause
	$\leadsto$ \alert{More an introduction into RL!}
	
\end{frame}
%----------------------------------------------------------------------
%----------------------------------------------------------------------
\begin{frame}[c]{Course Format}
	
	\begin{itemize}
		\item Concepts \& Details
		\begin{itemize}
			\item We provide references and links to papers\\ s.t. you can read up more details! 
			\item Nevertheless, we cover all the basic math for RL
		\end{itemize}
		\smallskip
		\item Interactive lecture and exercise sessions
		\begin{itemize}
		    \item Watch lecture at home!
		    \item Discuss it with us on-site
			\item Interactive quizzes in exercise sessions to reinforce your knowledge and understanding
			\item[$\leadsto$] The success of it depends on whether you are willing to talk to us! 
		\end{itemize}
		\smallskip
		\item (Mostly) Practical exercises
		\begin{itemize}
			\item Implement it, use it and play with it!
			\item We will answer questions for the current exercise and partially discuss solutions from the last week in the on-site sessions
		\end{itemize}
	\end{itemize}
	
\end{frame}
%----------------------------------------------------------------------

%----------------------------------------------------------------------
\begin{frame}[c]{Organization (Lectures)}
	
	\begin{itemize}
		\item Lectures are pre-recorded
		\item We will make them available on StudIP via Flowcast
		\item[$\leadsto$] Flipped classroom lecture
		\begin{itemize}
		    \item You watch the lectures whenever it suits you
		    \item We will discuss the lecture content in our on-site sessions
		    \item[$\leadsto$] each week Thursday at 2pm (s.t).
		    \item The meetings will not be recorded
		\end{itemize}
            \item If you are sick (or have symptoms), please send me an email in advance and we will set up a hybrid meeting via Zoom
	\end{itemize}
	
\end{frame}
%-----------------------------------------------------------------------
%----------------------------------------------------------------------
\begin{frame}[c]{Why Videos?}

\begin{itemize}
  \item Advantages of videos:
  \begin{itemize}
      \item Neither YOUTUBE nor NETFLIX VIDEOS!
      \item Watch it whenever (wherever) you want
      \item Watch it at your own speed
      \begin{itemize}
          \item[$\leadsto$] Stop it if you need time to think about it
      \end{itemize}
      \item Go back and watch it again, if you missed or forgot something
      \item Annotate questions on the fly %(e.g., using the Miro boards)
      \item Take notes of what you have learned $\leadsto$ Learning Diary
      \item After each video ($\sim$10-20min), you can take a break and\\ think about what you learned in this video (and whether you understood it)
  \end{itemize}
  \medskip
  \pause
  \item Risks and challenges:
  \begin{itemize}
      \item You have to be self-disciplined 
      \item You have to wait with your questions until our meetings
      \begin{itemize}
          \item[$\leadsto$] Use the StudIP forum to discuss with your peers
      \end{itemize}
  \end{itemize}
\end{itemize}

\end{frame}
%-----------------------------------------------------------------------
%----------------------------------------------------------------------
\begin{frame}[c]{Details on On-Site Meetings}
	
	\begin{enumerate}
	    \item Answering your questions regarding the videos
	    \item Breakout groups discussing some more in-depth questions
		\item Kahoot-Quiz
		\begin{itemize}
		    \item Helps you to check whether you really understand the topics
		    \item Gamified quiz 
		\end{itemize}
		\item Feedback to exercise sheet
		\begin{itemize}
			\item You don't need to achieve any point threshold
			%\item But you need to submit something every week
                \item There is no extra exercise slot
                \item[$\leadsto$] Consultation hour: 
                \begin{itemize}
                    \item You can meet with Aditya and/or Carolin Fridays from 13:00 - 14:30 
                    \item They will clarify all your detailed questions regarding the exercise
                    \item Send them an email beforehand to make an appointment
                \end{itemize}
		\end{itemize}
	\end{enumerate}
	
\end{frame}
%-----------------------------------------------------------------------
%----------------------------------------------------------------------
\begin{frame}[c]{Organization (Exercise Assignments)}
	
	\begin{itemize}
	\item Every week, a new exercise sheet
		\item Exercise focus is aligned with the week's lecture topic
		\begin{itemize}
		    \item \alert{Deadline} is always Fridays at 4pm
		\end{itemize}
		\item Most exercises will be practical, i.e., you have to implement something
		\item Teamwork highly recommended, team size at most 3! 
		\pause
		\item Build upon GitHub classroom $\leadsto$ enables auto-grading
		\begin{itemize}
			\item There will be an invitation link each week on each exercise sheet.
			\item You will have to click on the link on exercise sheet one to be able to form groups.
			\item Submit solutions via git
		\end{itemize}
		\pause
            \item Don't cheat (incl. plagiarism)
            \begin{itemize}
              \item First time cheating: $0$ points for exercise
              \item Second time cheating: failing the course
            \end{itemize}
            \pause
            \item Exercises are not mandatory\\ \alert{BUT: quite unlikely that you will pass the course without doing them}
	\end{itemize}
	
\end{frame}
%-----------------------------------------------------------------------
%----------------------------------------------------------------------
\begin{frame}[c]{Bonus Points}

\begin{itemize}

  \item Up to \alert{$8$ bonus points} for final grade
  \begin{itemize}
      \item Solve at least $80\%$ of an exercise sheet to obtain a bonus point
      \item $4$ bonus points $\leadsto$ grading boost by 0.3/0.4
  \end{itemize}
  \item We will (sometimes) ask someone of you to present your solution 
  \begin{itemize}
      \item We will only ask students who got at least 80\% correct
      \item If you refuse or fail to present your solution, you lose 2 bonus points
  \end{itemize}
\end{itemize}

\end{frame}
%-----------------------------------------------------------------------
%----------------------------------------------------------------------
\begin{frame}[c]{You need help?}
	
Priority list:
	\begin{enumerate}
		\item Ask your friends and peers
		\item Stud.IP Forum
		\begin{itemize}
		    %\item[$\leadsto$] Channel ``2022 RL Lecture''
		    \item Only use the lecture forum
			\item You can also answer the questions of your peers! 
			\item We will only reply if we have the feeling that it is necessary.
		\end{itemize}
		\item If there are organizational questions, contact Carolin or Aditya directly 
		\begin{itemize}
		    \item Don't write them emails about how to solve an exercise or that you found a bug in the exercise
		    \item[$\leadsto$] all of these belong into the Stud.IP forum
		\end{itemize}
		\item Only as the very last option, contact me ;-)
	\end{enumerate}
	
\end{frame}
%-----------------------------------------------------------------------
\begin{frame}{AI Courses at LUH}
    
    \centering
    \vspace{-2em}
    \includegraphics[width=0.38\textwidth]{images/aicources22.png}

\end{frame}
%----------------------------------------------------------------------

\begin{frame}{Leibniz AI Academy}
    
    \centering
    \vspace{-2em}
    \includegraphics[width=0.6\textwidth]{images/laa.png}

\end{frame}
%----------------------------------------------------------------------
\begin{frame}{Leibniz AI Academy}
    
    \centering
   % \vspace{-2em}
    \includegraphics[width=0.7\textwidth]{images/LAI-New-Blue.png}

    $\leadsto$ more details at \url{https://www.ai-academy.uni-hannover.de/}

\end{frame}
%----------------------------------------------------------------------
\begin{frame}[c]{Requirements for Attending}

        \vspace{-1em}
	\begin{itemize}
		\item Basics of \alert{AI} (recommended)
		\begin{itemize}
			\item Search, planning, optimization \ldots, expectations, \ldots
		\end{itemize}
		\pause
		\item Basics of \alert{Machine Learning} (mandatory)
		\begin{itemize}
			\item Classification, regression, clustering, decision tree, training-test split, cross validation, pre-processing \ldots
			\item For example, ML lecture by Prof. Rosenhahn
			\item to catch up (if nec.):
			\url{https://www.coursera.org/learn/machine-learning} 
		\end{itemize}
		\pause
		\item Knowledge and hands-on exp. in \alert{Deep Learning} (PyTorch)\\ (mandatory)
		\begin{itemize}
			\item feed-forward network, recurrent network, convolutions, learning rates, regularization, \ldots 
			\item For example, DL lecture by Prof. Anand
			\item to catch up (if nec.): \url{https://course.fast.ai/}
		\end{itemize}
		\pause
		\item Experience in \alert{Python and git} (mandatory)
		\begin{itemize}
			\item nearly all exercises will require 
			that you implement something in~Python and submit the solution to a git repo
		\end{itemize}
	\end{itemize}
	\pause
	$\leadsto$ If you solved the self-assessment test, you should be ready.
	
\end{frame}
%-----------------------------------------------------------------------
%----------------------------------------------------------------------
\begin{frame}[c]{Final Grading}
	
	\begin{itemize}
		\item Implement a larger project (worth $1-2$ weeks full time)
		\begin{itemize}
			\item You can propose your own project idea!
			\begin{itemize}
				\item Hand-in a short summary of the idea (half a page) and we will provide feedback regarding feasibility
			\end{itemize}
			\item Teamwork (at most 3) again possible
			\begin{itemize}
				\item Larger team $\to$ larger scope of the project
			\end{itemize}
		\end{itemize}
		\item ``Exam''
		\begin{itemize}
			\item First $15$ minutes: Present your project idea and results
			% \begin{itemize}
			% 	\item Of course, everyone will present the project on their own
			% \end{itemize}
			\item Second 15+min: We will ask further questions about your project and how it relates to stuff you learned in the lecture.
		\end{itemize}	
            \item \alert{Tentative dates}: Feb 27th - March 3rd (mornings)
	%	\item You will have the choice between a virtual and on-site exam.
	\end{itemize}
	
\end{frame}
%----------------------------------------------------------------------
%----------------------------------------------------------------------
\begin{frame}[c,fragile]{Material}
	
	\begin{itemize}
		\item Slides: \url{https://github.com/automl-edu/RL_lecture}
		\item Additional Material:
		\begin{itemize}
    		\item To get a deep understanding of RL,\\ you should also read some papers 
    		\item RL book by Sutton and Barto: 
    		\url{https://www.andrew.cmu.edu/course/10-703/textbook/BartoSutton.pdf}
    		\item Video lectures -- click on it!
    		\begin{itemize}
    			\item \lit{Emma Brunskill (2019-20)}{https://www.youtube.com/watch?v=FgzM3zpZ55o&list=PLRQmQC3wIq9yxKVK1qc0r2nPuInn92LmK&index=1}
    			\item \lit{Sergey Levine (2020)}{https://www.youtube.com/playlist?list=PL_iWQOsE6TfURIIhCrlt-wj9ByIVpbfGc}
    			\item \lit{David Silver (2015)}{https://www.youtube.com/playlist?list=PLbWDNovNB5mqFBgq7i3MY6Ui4zudcvNFJ}
    			\item \lit{Robot Learning by Jan Peters (2021)}{https://learn.ki-campus.org/courses/moocrobot-tud2021}
		\end{itemize}
	\end{itemize}
	\end{itemize}
	
\end{frame}
%----------------------------------------------------------------------
%----------------------------------------------------------------------
\begin{frame}[c]{Opportunities and Risks}
	
	\vspace{-1em}
	RL is an advanced lecture and we present it for the third time
	
	\medskip
	\pause
	
	Opportunities:
	\begin{itemize}
		\item RL is a very hot topic these days
		\item We will start with the basics and go step by step to the more advanced (research) topics
		\item The course will provide a solid background for doing a master project/thesis in our group 
	\end{itemize}
	
	\pause
	
	Challenges:
	\begin{itemize}
		\item The research on RL is very active and there is so much progress\newline $\leadsto$ impossible to catch up with state of the art with one course
		\item The origins of RL go back to robotics, control, theory on bandits and computer science\\ $\leadsto$ different notations
		\item You will find some typos and issues in the slides\newline $\leadsto$ please tell us if you find something
	\end{itemize}
	
	\pause
	$\to$ Give us some feedback and we will improve the course!
	
	
\end{frame}
%-----------------------------------------------------------------------
%----------------------------------------------------------------------
\begin{frame}[c]{Additional Bonus Point}

\begin{itemize}
    \item GitHub repo:
    \begin{itemize}
        \item Slides: \url{https://github.com/automl-edu/RL_lecture}
    \end{itemize}
    \item If you find bugs in the slides or exercises,\\ you can obtain bonus points:
    \begin{itemize}
        \item 0.5 point for every major bug in an equation
        \item 0.1 point for every typo in the slides
        \item 0.5 point for every code bug in the exercise
    \end{itemize}
    \item At most 4 points $\leadsto$ 1 grading boost
    \item Submit a PR to our repos and ensure that we can decipher your real name
\end{itemize}


\end{frame}
%-----------------------------------------------------------------------
\begin{frame}[c]{}
	
	\centering
	\huge
	Questions?
	
\end{frame}


%-----------------------------------------------------------------------
\end{document}
