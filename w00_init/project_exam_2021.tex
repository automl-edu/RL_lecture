\documentclass[aspectratio=169]{../latex_main/tntbeamer}  % you can pass all options of the beamer class, e.g., 'handout' or 'aspectratio=43'
\usepackage{dsfont}
\usepackage{bm}
\usepackage[english]{babel}
\usepackage[T1]{fontenc}
%\usepackage[utf8]{inputenc}
\usepackage{graphicx}
\graphicspath{ {./figures/} }
\usepackage{algorithm}
\usepackage[ruled,vlined,algo2e,linesnumbered]{algorithm2e}
\usepackage{hyperref}
\usepackage{booktabs}
\usepackage{mathtools}

\usepackage{amsmath,amssymb}
\usepackage{latexsym}

\DeclareMathOperator*{\argmax}{arg\,max}
\DeclareMathOperator*{\argmin}{arg\,min}

\usepackage{pgfplots}
\pgfplotsset{compat=1.16}
\usepackage{tikz}
\usetikzlibrary{trees} 
\usetikzlibrary{shapes.geometric}
\usetikzlibrary{positioning,shapes,shadows,arrows,calc,mindmap}
\usetikzlibrary{positioning,fadings,through}
\usetikzlibrary{decorations.pathreplacing}
\usetikzlibrary{intersections}
\pgfdeclarelayer{background}
\pgfdeclarelayer{foreground}
\pgfsetlayers{background,main,foreground}
\tikzstyle{activity}=[rectangle, draw=black, rounded corners, text centered, text width=8em]
\tikzstyle{data}=[rectangle, draw=black, text centered, text width=8em]
\tikzstyle{myarrow}=[->, thick, draw=black]

% Define the layers to draw the diagram
\pgfdeclarelayer{background}
\pgfdeclarelayer{foreground}
\pgfsetlayers{background,main,foreground}

\input{./latex_main_old/macros}

% Requires XeLaTeX or LuaLaTeX
\usepackage{unicode-math}

\usepackage{fontspec}
%\setsansfont{Arial}
\setsansfont{RotisSansSerifStd}[ 
Path=./latex_main/fonts/,
Extension = .otf,
UprightFont = *-Regular,  % or *-Light
BoldFont = *-ExtraBold,  % or *-Bold
ItalicFont = *-Italic
]
\setmonofont{Cascadia Mono}[
Scale=0.8
]

% scale factor adapted; mathrm font added (Benjamin Spitschan @TNT, 2021-06-01)
%\setmathfont[Scale=1.05]{Libertinus Math}
%\setmathrm[Scale=1.05]{Libertinus Math}

% other available math fonts are (not exhaustive)
% Latin Modern Math
% XITS Math
% Libertinus Math
% Asana Math
% Fira Math
% TeX Gyre Pagella Math
% TeX Gyre Bonum Math
% TeX Gyre Schola Math
% TeX Gyre Termes Math

% Literature References
% #1 = Display Name
% #2 = Url (without \href)
\newcommand{\lit}[2]{\href{#2}{\footnotesize\color{black!60}[#1]}}

%%% Beamer Customization
%----------------------------------------------------------------------
% (Don't) Show sections in frame header. Options: 'sections', 'sections light', empty
\setbeamertemplate{headline}{empty}

% Add header logo for normal frames
\setheaderimage{
	% \includegraphics[height=\logoheight]{figures/TNT_darkv4.pdf}
	\includegraphics[height=\logoheight]{./latex_main/figures/luh_logo_rgb_0_80_155.pdf}
	% \includegraphics[height=\logoheight]{figures/logo_tntluh.pdf}
}

% Header logo for title page
\settitleheaderimage{
	% \includegraphics[height=\logoheight]{figures/TNT_darkv4.pdf}
	\includegraphics[height=\logoheight]{./latex_main/figures/luh_logo_rgb_0_80_155.pdf}
	% \includegraphics[height=\logoheight]{figures/logo_tntluh.pdf}
}

% Title page: tntdefault 
\setbeamertemplate{title page}[tntdefault]  % or luhstyle
% Add optional title image here
%\addtitlepageimagedefault{\includegraphics[width=0.65\textwidth]{figures/luh_default_presentation_title_image.jpg}}

% Title page: luhstyle
% \setbeamertemplate{title page}[luhstyle]
% % Add optional title image here
% \addtitlepageimage{\includegraphics[width=0.75\textwidth]{figures/luh_default_presentation_title_image.jpg}}

\author[Lindauer]{Marius Lindauer\\[1em]
	\includegraphics[height=\logoheight]{./latex_main/figures/luh_logo_rgb_0_80_155.pdf}\qquad
\includegraphics[height=\logoheight]{./latex_main/figures/TNT_darkv4}\qquad
\includegraphics[height=\logoheight]{./latex_main/figures/L3S.jpg}	}
\date{Winter Term 2021
}


%%% Custom Packages
%----------------------------------------------------------------------
% Create dummy content
\usepackage{blindtext}

% Adds a frame with the current page layout. Just call \layout inside of a frame.
\usepackage{layout}

\title[RL: Exam]{Final Project for Exam}
%\subtitle{Final Project for Exam}



\begin{document}
	
	\maketitle

%----------------------------------------------------------------------
%----------------------------------------------------------------------
\begin{frame}[c]{Reminder: Final Grading}
	
	\begin{itemize}
		\item Implement a larger project (worth $1-2$ weeks full time)
		\begin{itemize}
			\item You can propose your own project idea!
			\begin{itemize}
				\item Hand-in a short summary of the idea and we will provide feedback regarding feasibility
			\end{itemize}
			\item Teamwork (at most 3) again possible
			\begin{itemize}
				\item Larger team $\to$ larger scope of the project
			\end{itemize}
		\end{itemize}
		\item ``Exam''
		\begin{itemize}
			\item First $\sim 15$ min: Present your project idea and results 
			\begin{itemize}
				\item Of course, everyone will present the project on their own
			\end{itemize}
			\item Second 15min: We will ask further questions about your project and how it relates to stuff you learned in the lecture.
		\end{itemize}	
		\item You will have the choice between a virtual and on-site exam.
	\end{itemize}
	
\end{frame}
%----------------------------------------------------------------------
%----------------------------------------------------------------------
\begin{frame}[c]{Requirements}
	\vspace{-1em}
	\begin{itemize}
		\item 2 different options for the scope of the project 
		\begin{itemize}
			\item see next slides
		\end{itemize}
		\pause
		\item Proposal should include:
			\begin{itemize}
				\item Scope and main objective(s)
				\item Which open-source frameworks you plan to use
				\begin{itemize}
					\item both for RL agents and for benchmark envs
					\item some presented in the next (last) exercise
				\end{itemize}
				\item Amount for compute resources you will use
				\item Rough time frame / milestones
				\item[$\leadsto$] At most 1 page
			\end{itemize}
		\pause
		\item Sound scientific  workflow
		\begin{itemize}
			\item Fill out reproducibility check list (see studip)
		\end{itemize}
		\pause
		\item Hand in: source code, ML reproducibility check list and PDF presentation with at most 7 slides (excl. title slide)
		\begin{itemize}
			\item Proper code documentation; PEP8
			\item Send us an invitation to a GitHub Repo
			\item To both: eimer@tnt.uni-hannover.de, schubert@tnt.uni-hannover.de
		\end{itemize}
	\end{itemize}
	
\end{frame}
%----------------------------------------------------------------------
%----------------------------------------------------------------------
\begin{frame}[c]{Option I: New Env}
	
	\begin{itemize}
		\item Propose a new \& interesting benchmark environment / application for RL (incl. state, action, reward, transition, $\ldots$)
		\begin{itemize}
			\item Use OpenAI-Gym env API
			\item Should be somehow related to preventing climate change
		\end{itemize}
		\item Minimal requirement: An RL agent (of your choice) can learn something reasonable
		\begin{itemize}
			\item That is, it performs better than a static or random policy
			\item You can use already implemented RL algorithms
			\item We will provide a list of recommended RL agents
		\end{itemize}
		\item Further goals could include:
		\begin{itemize}
			\item Impact of state or action space (e.g., size or encoding)
			\item Reward signal or reward shaping
			\item Does the Markov assumption hold true?
		\end{itemize}
		\item[$\leadsto$] Environment should not be too trivial (e.g., a small maze)
	\end{itemize}
	
\end{frame}
%----------------------------------------------------------------------
%----------------------------------------------------------------------
\begin{frame}[c]{Option II: RL Agent}
	
	\begin{itemize}
		\item Implement an RL agent from scratch
		\begin{itemize}
		    \item Pick a (recent) RL paper and re-implement it
			\item Don't use existing implementations!
		\end{itemize}
		\item Minimal requirement: Your RL agent can learn a reasonable policy\\ on a RL benchmark of your choice
		\begin{itemize}
			\item That is, it performs better than a static or random policy
			\item You can use already implemented RL benchmark envs
			\item We will provide a list of recommended RL envs in the next exercise session
		\end{itemize}
		\item Further goals could include:
		\begin{itemize}
			\item Variants of different algorithm components (e.g., experience replay)
			\item Hyperparameter sensitivity study
			\item Comparison against other baselines
		\end{itemize}
	\end{itemize}
	
\end{frame}
%----------------------------------------------------------------------
%----------------------------------------------------------------------
\begin{frame}[c]{Score Distribution\footnote{Not the same scoring as in iML}}
	\vspace{-1em}
	\begin{itemize}
		\item Idea, Topic \& Results: at most 25\\
        (sufficiently challenging, interesting beyond lecture, gained insights, results, ...)
        \item Implementation:  at most 25\\
        (correctness, reproducibility, code documentation, code quality, ...)
        \item Presentation: at most 25\\
        (concise slides, clear message, structure, citations, plots, ...)
        \item Answering of questions: at most 25\\
        (correct, concise, knowledge, own ideas, ...)
	\end{itemize}
	\pause
	\begin{itemize}
	    \item 100-90: 1.0; 89 - 85: 1.3; 84 - 80: 1.7
	    \item 79 - 75: 2.0; 74 - 70: 2.3; 69 - 65: 2.7
	    \item 64 - 60: 3.0; 59 - 55: 3.3; 54 - 50: 3.7
	    \item 49 - 45: 4.0; $<$ 45 : failed
	\end{itemize}
	
\end{frame}
%----------------------------------------------------------------------
%----------------------------------------------------------------------
\begin{frame}[c]{Deadlines}
	
	\begin{itemize}
		\item Proposal Deadline: Jan 27th (AoE\footnote{Anywhere on Earth})
		\begin{itemize}
			\item Submit earlier to get feedback sooner!
		\end{itemize}
		\item Feedback for proposal: latest Feb 3rd
		\item Results Deadline: March 4th (AoE) 
		\item Exam: March 14th - March 18th (in the mornings)
		\begin{itemize}
			\item We will send you a link for registration to one of the possible slots
		\end{itemize}
	\end{itemize}
	
\end{frame}
%----------------------------------------------------------------------


\begin{frame}[c]{}
	
	\centering
	\huge
	Questions?
	
\end{frame}


%-----------------------------------------------------------------------
\end{document}
