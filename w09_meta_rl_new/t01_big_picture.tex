% !TeX spellcheck = en_US
\documentclass[aspectratio=169]{../latex_main/tntbeamer}  % you can pass all options of the beamer class, e.g., 'handout' or 'aspectratio=43'
\input{../latex_main/preamble_adi}
\title[Meta-RL]{Meta Reinforcement Learning}
\subtitle{The Big Picture}


\begin{document}
	
	\maketitle

%-----------------------------------------------------------------------
%----------------------------------------------------------------------
\begin{frame}{What do we want to achieve today ...}

    \centering
    \includegraphics[width=0.5\textwidth]{w09_meta_rl_new/images/t01/Buzz.png}

    \begin{itemize}
        \item To put all of these words into a framework
        \item A bunch of perspectives through which we can look and understand techniques in Meta-RL
    \end{itemize}

\end{frame}



%-----------------------------------------------------------------------
%----------------------------------------------------------------------
\begin{frame}{An Ode to Prior Knowledge}

    \begin{columns}
    \column{0.5\textwidth} 
        \begin{center}
        \textbf{Breakout}\\ \\
        \includegraphics[width=0.5\textwidth]{images/t01/breakout.jpg}\\
        
        \end{center}
        
        \column{0.5\textwidth}
        \begin{center}
        \textbf{Montezuma's Revenge} \\
        \includegraphics[width=0.5\textwidth]{images/t01/montezuma.jpg}\\

    \end{center}
    
    \end{columns}
        
        
    \pause

    \begin{columns}
        \column{0.5\textwidth} 
        \centering
        Pretty Easy for a standard DQN

        \column{0.5\textwidth}
        \centering
        Almost Impossible for a standard DQN
    \end{columns}
    
    \footnote{Image source: \url{https://paperswithcode.com/task/atari-games}}

\end{frame}


%-----------------------------------------------------------------------
%----------------------------------------------------------------------
\begin{frame}{An ode to Prior Knowledge}

    Breakout:
    \begin{itemize}
        \item \textbf{Actions:} FIRE, RIGHT, LEFT
        \item \textbf{Rewards:} Score points by destroying bricks in the wall
    \end{itemize}
    
    \vfill
    
    Montezuma's Revenge:
    \begin{itemize}
        \item \textbf{Actions:} Set of 17 actions including movement and firing the gun
        \item \textbf{Rewards:} Getting a Key, Opening a Door
        \item \textbf{Ambiguity:} Getting killed by a skull gives no rewards (good? bad?)
    \end{itemize}
    
    \vfill
    \pause
    \noindent
    \textbf{Key point:} We know what to do because we \textbf{understand} what most sprites mean

\end{frame}

%-----------------------------------------------------------------------
%----------------------------------------------------------------------
\begin{frame}{Transferring skills}

    Prior knowledge helps us incrementally acquire new skills
    \vfill
    
    \begin{center}
        \includegraphics[width=0.8\textwidth]{images/t01/transfer.png} % made by Aditya using CC-sources
    \end{center}
    
    \vfill
    \pause
    \textbf{Transfer Learning:} Can we train our agents to learn new skills better by leveraging prior knowledge?
    

\end{frame}

%-----------------------------------------------------------------------
%----------------------------------------------------------------------
\begin{frame}{Transfer Learning}

Using experience from one (set of) tasks for faster learning and better adaptation to a new (set of) tasks

\begin{itemize}
    \item \textbf{Shots:} Number of attempts of learning in the target domain
    \item \textbf{Zero-Shot:} When we don't perform any learning in the target domain
    \item \textbf{Few-Shot:} We perform few attempts
\end{itemize}

\begin{center}
    \includegraphics[width=0.5\textwidth]{images/t01/transfer-learning.png}
\end{center}


\end{frame}

%-----------------------------------------------------------------------
%----------------------------------------------------------------------
\begin{frame}{Multi-Task Learning}

Train on many tasks and then transfer to a new task



\begin{center}
    \includegraphics[width=0.6\textwidth]{images/t01/multi-task-learning.png}
\end{center}


\end{frame}

%-----------------------------------------------------------------------
%----------------------------------------------------------------------
\begin{frame}{Meta-Learning}

Train on many tasks to figure out how to learn more efficiently!
\begin{itemize}
    \item Learn how to Learn! (L2L)
\end{itemize}

\vfill
\pause
What can we meta-learn?
\begin{itemize}
    \item Hyperparameters: Optimizer configurations
    \item Optimization methods
    \item Training Dynamics: credit assignment, models
    \item Representations ...
\end{itemize}

\end{frame}

%-----------------------------------------------------------------------
%----------------------------------------------------------------------
\begin{frame}{Benefits of Meta-Learning}

\begin{itemize}
    \item Improved sample efficiency
    \item A faster learner for new tasks i.e. quicker adaptation
    \item Intelligent exploration 
    \item Acquiring the right features more quickly
    \item Robustness to changes in the environment
    \item Generalization to different kinds of tasks
\end{itemize}    
\end{frame}

%-----------------------------------------------------------------------
%----------------------------------------------------------------------
\begin{frame}{Storing Prior Knowledge}

\begin{itemize}
    \item \textbf{Q-Function:} tells us which actions or states are good
    \item \textbf{Policy:} tells us which actions (or states) are good
    \item \textbf{Models of the Environment:} What are the laws of physics that govern the world?
    \item \textbf{Features:} Provide us with good representations of states


\end{itemize}
\vfill

\end{frame}


\end{document}
