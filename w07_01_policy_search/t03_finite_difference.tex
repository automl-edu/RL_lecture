% !TeX spellcheck = en_US
\documentclass[aspectratio=169]{../latex_main/tntbeamer}  % you can pass all options of the beamer class, e.g., 'handout' or 'aspectratio=43'
\usepackage{dsfont}
\usepackage{bm}
\usepackage[english]{babel}
\usepackage[T1]{fontenc}
%\usepackage[utf8]{inputenc}
\usepackage{graphicx}
\graphicspath{ {./figures/} }
\usepackage{algorithm}
\usepackage[ruled,vlined,algo2e,linesnumbered]{algorithm2e}
\usepackage{hyperref}
\usepackage{booktabs}
\usepackage{mathtools}

\usepackage{amsmath,amssymb}
\usepackage{latexsym}

\DeclareMathOperator*{\argmax}{arg\,max}
\DeclareMathOperator*{\argmin}{arg\,min}

\usepackage{pgfplots}
\pgfplotsset{compat=1.16}
\usepackage{tikz}
\usetikzlibrary{trees} 
\usetikzlibrary{shapes.geometric}
\usetikzlibrary{positioning,shapes,shadows,arrows,calc,mindmap}
\usetikzlibrary{positioning,fadings,through}
\usetikzlibrary{decorations.pathreplacing}
\usetikzlibrary{intersections}
\pgfdeclarelayer{background}
\pgfdeclarelayer{foreground}
\pgfsetlayers{background,main,foreground}
\tikzstyle{activity}=[rectangle, draw=black, rounded corners, text centered, text width=8em]
\tikzstyle{data}=[rectangle, draw=black, text centered, text width=8em]
\tikzstyle{myarrow}=[->, thick, draw=black]

% Define the layers to draw the diagram
\pgfdeclarelayer{background}
\pgfdeclarelayer{foreground}
\pgfsetlayers{background,main,foreground}

\input{./latex_main_old/macros}

% Requires XeLaTeX or LuaLaTeX
\usepackage{unicode-math}

\usepackage{fontspec}
%\setsansfont{Arial}
\setsansfont{RotisSansSerifStd}[ 
Path=./latex_main/fonts/,
Extension = .otf,
UprightFont = *-Regular,  % or *-Light
BoldFont = *-ExtraBold,  % or *-Bold
ItalicFont = *-Italic
]
\setmonofont{Cascadia Mono}[
Scale=0.8
]

% scale factor adapted; mathrm font added (Benjamin Spitschan @TNT, 2021-06-01)
%\setmathfont[Scale=1.05]{Libertinus Math}
%\setmathrm[Scale=1.05]{Libertinus Math}

% other available math fonts are (not exhaustive)
% Latin Modern Math
% XITS Math
% Libertinus Math
% Asana Math
% Fira Math
% TeX Gyre Pagella Math
% TeX Gyre Bonum Math
% TeX Gyre Schola Math
% TeX Gyre Termes Math

% Literature References
% #1 = Display Name
% #2 = Url (without \href)
\newcommand{\lit}[2]{\href{#2}{\footnotesize\color{black!60}[#1]}}

%%% Beamer Customization
%----------------------------------------------------------------------
% (Don't) Show sections in frame header. Options: 'sections', 'sections light', empty
\setbeamertemplate{headline}{empty}

% Add header logo for normal frames
\setheaderimage{
	% \includegraphics[height=\logoheight]{figures/TNT_darkv4.pdf}
	\includegraphics[height=\logoheight]{./latex_main/figures/luh_logo_rgb_0_80_155.pdf}
	% \includegraphics[height=\logoheight]{figures/logo_tntluh.pdf}
}

% Header logo for title page
\settitleheaderimage{
	% \includegraphics[height=\logoheight]{figures/TNT_darkv4.pdf}
	\includegraphics[height=\logoheight]{./latex_main/figures/luh_logo_rgb_0_80_155.pdf}
	% \includegraphics[height=\logoheight]{figures/logo_tntluh.pdf}
}

% Title page: tntdefault 
\setbeamertemplate{title page}[tntdefault]  % or luhstyle
% Add optional title image here
%\addtitlepageimagedefault{\includegraphics[width=0.65\textwidth]{figures/luh_default_presentation_title_image.jpg}}

% Title page: luhstyle
% \setbeamertemplate{title page}[luhstyle]
% % Add optional title image here
% \addtitlepageimage{\includegraphics[width=0.75\textwidth]{figures/luh_default_presentation_title_image.jpg}}

\author[Lindauer]{Marius Lindauer\\[1em]
	\includegraphics[height=\logoheight]{./latex_main/figures/luh_logo_rgb_0_80_155.pdf}\qquad
\includegraphics[height=\logoheight]{./latex_main/figures/TNT_darkv4}\qquad
\includegraphics[height=\logoheight]{./latex_main/figures/L3S.jpg}	}
\date{Winter Term 2021
}


%%% Custom Packages
%----------------------------------------------------------------------
% Create dummy content
\usepackage{blindtext}

% Adds a frame with the current page layout. Just call \layout inside of a frame.
\usepackage{layout}

\title[RL: Finite Difference]{RL: Policy Search}
\subtitle{Finite Difference}



\begin{document}
	
	\maketitle

%----------------------------------------------------------------------
%----------------------------------------------------------------------
\begin{frame}[c]{Policy Gradient}

\begin{itemize}
%	\item Define $V(\theta) = V(s_0, \theta)$ to make explicit the dependence of the value on the policy parameters [but don't confuse with value function approximation, where parameterized value function]
	\item Assume episodic MDPs $\leadsto$ Only interested in gradients w.r.t. $s_0$
	\item Policy gradient algorithms search for a \alert{local} maximum in $V(s_0;\theta)$ by ascending the gradient of the policy, w.r.t parameters $\theta$
	$$\Delta \theta = \alpha \nabla_\theta V(s_0; \theta) $$
	where $\alpha$ is the learning rate (step-size) and\\ $\nabla_\theta V(s_0; \theta)$ is the policy gradient
		$$\nabla_\theta V(s_0, \theta) = \begin{pmatrix}
	\frac{\partial V(s_0; \theta)}{\partial \theta_1}\\
	\vdots\\
	\frac{\partial V(s_0; \theta)}{\partial \theta_n}
	\end{pmatrix} $$
	\medskip
	\item Remark: $V(s;\theta)$ is a short form for $V^{\pi_{\theta}}(s)$ 
\end{itemize}

\end{frame}
%-----------------------------------------------------------------------
%----------------------------------------------------------------------
\begin{frame}[c]{Simple Approach: Compute Gradients by Finite Differences}

        \vspace{-1em}
	\begin{itemize}
		\item To evaluate the gradient of $\nabla_\theta V(s_0; \theta)$
		\item For each dimension $k\in [1,n]$
		\begin{itemize}
			\item Estimate $k$-th partial derivative of objective function wrt $\theta$
			\item Perturb $\theta$ by small amount $\epsilon$ in $k$-th dimension
			$$\frac{\partial V(s_0; \theta)}{\partial \theta_k} \approx \frac{V(s_0; \theta + \epsilon u_k) - V(s_0; \theta)}{\epsilon} $$
			where $u_k$ is a unit vector with $1$ in $k$-th component and $0$ elsewhere
		\end{itemize}
		\pause
		\item $\epsilon$ should be
		\begin{itemize}
			\item large enough to observe a change in the policy and value function
			\item small enough to have a good gradient approximation
		\end{itemize}
		\pause
		\item Uses $\geq n$ evaluations to compute policy gradient in $n$ dimensions
		\begin{itemize}
			\item[$\leadsto$] fairly inefficient for doing a single update!
			\item[$\leadsto$] weight space should be small --- no computer vision model sizes
		\end{itemize}
		\pause
		\item Simple, noisy, inefficient -- but sometimes effective
		\item Works for arbitrary policies, even if policy is not differentiable
	\end{itemize}
	
\end{frame}
%-----------------------------------------------------------------------
\end{document}
